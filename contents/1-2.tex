\chapter{卤素}\label{chp:halogen}
我们在初中化学里,已经知道氟原子和氯原子的电子层结构,它们的最外电子层都有 7 个电子。
在 107 种元素的原子里,还有溴、碘、砹的原子结构跟氟和氯相似,在最外层都有 7 个电子。
氟、氯、溴、碘、砹具有相似的化学性质,成为一族,称为卤族元素,简称卤素。
砹在自然界里含量很少。
在这章里,主要介绍氯,并在认识氯的基础上,学习氟、溴、碘。

\section{氯气}
\subsection{氯气的性质}
\medskip\noindent
\begin{minipage}{0.52\linewidth}\parindent2em
氯气(\ce{Cl2})的分子是由两个氯原子\footnotemark[1]组成的双原子分子(\cref{fig:2-1})。 
氯分子也象氢分子一样,分子里共用电子对处在两个原子核的中间。 
氯气是一种非金属单质。
在通常情况下,氯气呈黄绿色,\qty{1}{atm} 下,冷却到 \qty{-34.6}{\celsius},变成液氯,液氯继续冷却到 \qty{-101}{\celsius},变成固态氯。
\end{minipage}\hfill
\begin{minipage}{0.43\linewidth}\centering
\begin{figurehere}
  \includegraphics{2-1.pdf}
  \caption{氯气分子}\label{fig:2-1}
\end{figurehere}
\end{minipage}
\footnotetext[1]{氯原子很小,它的原子半径,即氯分子中两个原子核间距离的一半,是 \qty{0.99e-10}{m}。}

\begin{Experiment}
  展示一瓶氯气,瓶后衬一张白纸,以便清晰地观察到氯气的颜色。
\end{Experiment}

氯气有毒,有剧烈的刺激性,吸入少量氯气会使鼻和喉头的粘膜受到刺激,引起胸部疼痛和咳嗽;吸入大量氯气会中毒致死。
实验室里,闻氯气的时候,必须十分小心,应该用手轻轻地在瓶口扇动,仅使极少量的氯气飘进鼻孔(\cref{fig:2-2})。

我们已经知道氯原子的最外电子层上有 7 个电子,因而在化学反应中容易结合一个电子,使最外电子层达到 8 个电子的稳定结构。
氯气的化学性质很活泼,它是一种活泼的非金属。
\begin{figure}
  \begin{minipage}[b]{0.48\linewidth}\centering
    \caption{闻氯气的方法}\label{fig:2-2}
  \end{minipage}
  \begin{minipage}[b]{0.48\linewidth}\centering
    \caption{铜在氯气里燃烧}\label{fig:2-3}
  \end{minipage}
\end{figure}

\subsubsection{氯气跟金属的反应}
氯气跟金属钠的反应很剧烈,这在初中化学里已经观察过。
氯气不但跟钠等活泼金属直接化合,而且还能跟受热的铜等某些不活泼的金属起反应。
\begin{Experiment}
  把一束细铜丝灼热后,立刻放进盛有氯 气的集气瓶里(\cref{fig:2-3}),观察发生的现象。把少量的水注入集气瓶里,用毛玻璃片把瓶口盖住,振荡。观察溶液的颜色。
\end{Experiment}
可以看到红热的铜丝在氯气里燃烧起来,集气瓶里充满棕色的烟,这是氯化铜晶体颗粒。
这个反应可以用化学方程式表示如下:
\[ \ce{ Cu + Cl2 \xlongequal{\triangle} CuCl2 } \]

氯化铜溶解在水里,成为绿色的氯化铜溶液。

\subsubsection{氯气跟非金属的反应}
\begin{Experiment}*
把新收集的一瓶氯气和一瓶氢气(氢气和氯气可以分别收集在透明或半透明的塑料制的集气瓶里),口对口地对着,抽去瓶口间的较璃片,上下颠倒几次,使氯气和氢气充分混和。
拿一瓶氯、氢混和气体作试验;用塑料片盖好,在离瓶约 \qty{10}{cm} 处点燃镁条,当发生的强光照射混和气体时,可以观察到因瓶里的氯气跟氢气迅速化合而发生的爆炸,把塑料片向上弹起(\cref{fig:2-4})。
\tcblower
\begin{figurehere}
  \caption{氯气和氢气化合}\label{fig:2-4}
\end{figurehere}
\end{Experiment}
氯气跟氢气起反应,生成氯化氢气体,放出大量的热,以致发生爆炸。
\[ \ce{H2}\,\text{(气)} + \ce{Cl2}\,\text{(气)} \xlongequal{\text{光照}} \ce{2HCl}\,\text{(气)} +\qty{44.1}{kCal} \]

\begin{Experiment}*
把红磷放在燃烧匙里,点燃后插入盛有氯气的集气瓶里,磷就燃烧起来(\cref{fig:2-5})。观察发生的现象。
\tcblower
\begin{figurehere}
  \caption{磷在氯气里燃烧}\label{fig:2-5}
\end{figurehere}
\end{Experiment}
氯气跟磷起反应,生成三氯化磷和五氯化磷。出现的白色烟雾是三氯化磷和五氯化磷的混和物。
\begin{gather*}
  \ce{ 2P +3Cl2 \xlongequal{\text{点燃}} 2PCl3 }\\
  \ce{ PCl3 +Cl2 \xlongequal{\quad} PCl5 }
\end{gather*}

三氯化磷是无色液体,是重要的化工原料,用来制造许多磷的化合物,如敌百虫等多种农药。

\subsubsection{氯气跟水的反应}
氯气溶解于水,在常温下,1 体积的水能够溶解约 2 体积的氯气。
氯气的水溶液叫做 “氯水”。
溶解的氯气能够跟水起反应,生成盐酸和次氯酸(\ce{HClO})。
\[ \ce{Cl2 + H2O \xlongequal{\quad} HCl + HClO} \]

次氯酸不稳定,容易分解,放出氧气。当氯水受自光照射时,次氯酸的分解加速了。

\begin{Experiment}*[righthand ratio=0.6]
  当日光照射到如\cref{fig:2-6} 盛有氯水的装置时,不久就见到有气泡逸出。
  \tcblower
  \begin{figurehere}
    \caption{氯水被分解}\label{fig:2-6}
  \end{figurehere}
\end{Experiment}
次氯酸是一种强氧化剂,能杀死水里的病菌,所以自来水常用氯气(\qty{1}{L} 水里约通入 \qty{0.002}{g} 氯气)来杀菌消毒。次氯酸能使染料和有机色质褪色,可用作漂白剂。

\begin{Experiment}
  取干燥的和湿润的有色布条各一条,放在如\cref{fig:2-7} 的装置里,观察发生的现象。
  \tcblower
  \begin{figurehere}
    \caption{次氯酸使色布褪色}\label{fig:2-7}
  \end{figurehere}
\end{Experiment}
\subsubsection{氯气跟碱的反应}
氯气跟氢氧化钠等碱类都能较快地发生反应,所以制氯气时可以用碱液吸收剩余的氯气。
\[ \ce{ 2NaOH + Cl2 \xlongequal{\quad} NaCl + NaClO + H2O } \]

由于次氯酸盐比次氯酸稳定,容易保存。
工业上就用氯气和消石灰制成漂白粉,漂白粉的有效成分是次氯酸钙。
制漂白粉的反应可以用化学方程式简单表示如下:
\[ \ce{ 2Ca(OH)2 + 2Cl2 \xlongequal{\quad} Ca(ClO)2 +CaCl2 +2H2O } \]

漂白粉应用于漂白的时候,使次氯酸钙跟稀酸或空气里的二氧化碳和水蒸气起反应,就生成次氯酸。
\begin{gather*}
  \ce{ Ca(ClO)2 + 2HCl \xlongequal{\quad} CaCl2 +2HClO } \\
  \ce{ Ca(ClO)2 + CO2 +H2O \xlongequal{\quad} CaCO3 v + 2 HClO }
\end{gather*}
\subsection{氯气的用途}
氯气除用于消毒、制造盐酸和漂白粉外,还用于制造多种农药,制造氯仿等有机溶剂,所以氯气是一种重要的化工原料。

\subsection{氯气的试验室制法}
在实验室里,氯气可以用浓盐酸跟二氧化锰起反应来制取。
\begin{Experiment}
  象\cref{fig:2-8} 所示那样把装置连接好,检查气密性。在烧瓶里加入少量二氧化锰粉末,从分液漏斗慢慢地注入密度为 \qty{1.19}{g/cm^3} 的浓盐酸。缓缓加热,使反应加速,氯气就均匀地放出。用向上排空气法收集,多余的氧气用氢氧化钠吸收。
  \tcblower
  \begin{figurehere}
    \caption{实验室制取氯气}\label{fig:2-8}
  \end{figurehere}
\end{Experiment}

这个反应可以用化学方程式表示如下:
\[ \ce{ 4HCl + MnO2 \xlongequal{\triangle} MnCl2 + 2H2O + Cl2 ^}\]

\begin{Practice}[习题]
  \begin{question}
    \item 下列说法里哪一条是正确的?
    \begin{tasks}
      \task 氯原子跟氯离子的性质是一样的。
      \task 氯离子比氯原子多一个电子。
      \task 氯离子呈黄绿色。
    \end{tasks}
    \item 新制备的氯水和长久搁置的氯水在成分上有什么不同?
    \item 写出氯气跟锌、铝、铁反应的化学方程式。
    \item 固态的磷跟氯气起反应,生成 \qty{1}{mol} 气态三氧化磷,放出 \qty{73.2}{kCal} 的热;气态三氯化磷再跟氧气起反应,生成  \qty{1}{mol} 气态五氧化磷,放出 \qty{22.1}{kCal} 的热。写出这两个热化学方程式。
    \item 取含 78\% \ce{MnO2} 的软锰矿 \qty{150}{g},跟足量浓盐酸起反应,可以制得氯气多少克?
  \end{question}
\end{Practice}
\section{氯化氢和盐酸}
\subsection{氯化氢}
\begin{Experiment}
  把少量食盐放在烧瓶里(\cref{fig:2-9})。通过分液漏斗注入浓硫酸,同时加热。把氯化氢收集在干燥的容器里。一部分氯化氢用水收集。
  \tcblower
  \begin{figurehere}
    \caption{实验室制取氯化氢}\label{fig:2-9}
  \end{figurehere}
\end{Experiment}

食盐跟浓硫酸起反应,不加热或稍微加热,就生成硫酸氢钠和氯化氢。
\[ \ce{NaCl + H2SO4 \text{(浓)} \xlongequal{\quad} NaHSO4 + HCl ^} \]
在 \qtyrange{500}{600}{\celsius} 的条件下,继续起反应而生成硫酸钠和氯化氢。
\[ \ce{NaHSO4 + NaCl \xlongequal{\triangle} Na2SO4 + HCl ^} \]

总的化学方程式可以表示如下:
\[ \ce{2NaCl + H2SO4 \xlongequal{\triangle} Na2SO4 +2HCl ^} \]

氯化氢是没有颜色而有刺激性的气体。
它易溶于水,在 \qty{0}{\celsius} 时,1 体积的水大约能溶解 500 体积的氯化氢。

\begin{Experiment}*[righthand ratio=0.6]
  在干燥的圆底烧瓶里装满氯化氢,用带有玻璃管和滴管(滴管里预先吸入水)的塞子塞紧瓶口。立即倒置烧瓶,使玻璃管放进盛着石蕊溶液的烧杯里。压缩滴管的胶头,出水几滴。烧杯里的溶液即由玻璃管喷入烧瓶,形成美丽的喷泉(\cref{fig:2-10})。
  \tcblower
  \begin{figurehere}
    \caption{氯化氢在水里溶解}\label{fig:2-10}
  \end{figurehere}
\end{Experiment}

\subsection{盐酸和金属氯化物}
氯化氢的水溶液呈酸性,叫氢氯酸,习惯上又叫盐酸。
盐酸的性质我们在初中化学里已经学过,它能够使酸碱指示剂变色,能够跟金属活动性顺序中氢以前的金属起置换反应,能够跟碱起中和反应,能够跟盐起复分解反应而生成不溶性 的或挥发性的物质。
它跟金属、碱或盐反应,生成金属氯化物。

金属氯化物在自然界里分布很广,也广泛地应用于日常生活中、工农业生产上等等。
重要的有氯化钠、氯化钾、氯化镁、氯化锌等。
在这里只介绍氯化钠。

氯化钠俗名食盐,它对于人和高等动物的正常生理活动是不可缺少的。
我们每天要吃一点食盐,来补充从尿、汗水里所排泄掉的氯化钠。
食盐在自然界里分布很广。
海水里含有丰富的食盐。
由于地壳的变化,食盐也蕴藏在盐湖、盐井和盐矿中。
我国有极为丰富的食盐资源,盛产海盐、井盐、池盐和岩盐。

海水和盐湖等都蕴藏着丰富的资源。
除生产食盐外,它们的综合利用能够制得钾肥和许多别的盐类,制得溴等产品。
这些产品是农业和许多工业不可缺少的原料。

从海水晒盐或用从盐井汲出的卤水煮盐,都是为了把水蒸发掉,使食盐溶液达到饱和,继续蒸发,食盐不断成晶体析出。
这样得到的食盐晶体还含有较多的杂质,常叫粗盐。
粗盐经过再结晶,就得到精盐。

纯净的氯化钠的晶体呈立方形,在 \qty{801}{\celsius} 熔化,在 \qty{1413}{\celsius} 沸腾。
纯净的氯化钠在空气里不潮解,粗盐因含有氯化镁、氯化钙等杂质,易潮解。

食盐的用途很广。
日常生活里用于调味和腌渍蔬菜、鱼肉、蛋类等等。
医疗上用的生理盐水是 0.9\% 的食盐水。
食盐是重要的化工原料,用于制取金属钠、氯气、氢氧化钠、纯碱等等化工产品。

\begin{Practice}[习题]
  \begin{question}
    \item \qty{11.2}{L} 氧气和 \qty{11.2}{L} 氢气起反应,生成多少升的氯化氢气体(气体体积都按标准状况计),把生成的氧化氢都溶解在 \qty{328.5}{g} 水里形成密度为 \qty{1.047}{g/cm^3} 的盐酸,计算这种盐酸的摩尔浓度。
    \item 写出盐酸跟下列物质起反应的化学方程式。
    \begin{tasks}(3)
      \task \ce{Mg}
      \task \ce{MgO}
      \task \ce{Mg(OH)2}
      \task \ce{Mg(HCO3)2}
      \task \ce{MgCO3}
    \end{tasks} 
    \item \qty{11.7}{g} 氯化钠跟 \qty{10}{g} 浓度为 98\% 的硫酸反应,微热时生成多少克氟化氢?继续加热到 \qty{600}{\celsius} 时,又生成多少克的氯化氢?
    \item \qty{1}{g} 的锌和 \qty{1}{g} 的铁分别跟足量的稀盐酸起反应,各生成多少升的氢气(在标准状况下)?
    \item \qty{1.5}{mL} 密度为 \qty{1.028}{g/cm^3} 的盐酸(6\% \ce{HCl})跟足量的硝酸银溶液起反应,计算生成的氯化银的质量。
    \item 为什么在晒盐的时候,日晒风吹都有利于食盐晶体的析出?
    \item 为什么制食盐的时候,不宜采用降低溶液温度的方法?
  \end{question}
\end{Practice}

\section{氧化—还原反应}
我们在初中化学里,已经学习过氧化—还原反应,知道氧化、还原并不限于得氧或失氧的反应,而是可以用正负化合价升降的观点来分析氧化-还原反应。
在反应过程里,物质所含元素化合价升高的反应是氧化反应;物质所含元素化合价降低的反应是还原反应。
所含元素化合价升高的物质是还原剂,所含元素化合价降低的物质是氧化剂。
我们现在就用这个观点来分析在这章里学过的几个氧化—还原反应。

我们在初中化学里已经知道,氯化钠是由氯离子和钠离子构成的。
钠原子失去 1 个电子,成为钠离子,氯原子得到 1 个电子,成为氯离子。
在这里,可以从电子得失的观点进一步分析几个氧化-还原反应。


我们知道,氯气跟钠或铜起反应,分别生成了氯化钠和氯化铜。
在反应过程里,铜原子也象钠原子一样失去了电子成为铜离子。
在下面的化学方程式里,“$e$” 表示电子,并用箭头表明同一元素的原子得到或失去电子的情况。


\ce{NaCl} 里,钠是 $+1$ 价,氯是 $-1$ 价。
在反应过程里,钠原子失去 1 个电子,钠从 0 价变到 $+1$ 价,化合价升高;氯原子得到 1 个电子,氟从 0 价变到 $-1$ 价,化合价降低。
同样地,铜原子失去 2 个电子,铜从 0 价升高到 $+2$ 价,化合价升高;氯原子得到 1 个电子,氯从 0 价降低到 $-1$ 价,化合价降低。
元素化合价升高是由于失去电子,升高的价数也就是失去的电子数。
元素化合价降低是由于得到电子,降低的价数也就是得到的电子数。
元素的化合价升降的原因就是它们的原子失去或得到电子的缘故。
所以,我们可以给氧化—还原反应下一个更确切的定义。

\emph{物质失去电子的反应就是}\Concept{氧化反应},\emph{物质得到电子的反应就是}\Concept{还原反应}。

在氧化—还原反应里,一种物质失去电子,必然同时有别的物质得到电子;一种物质失去电子的数目总是跟别的物质得到电子的数目相等。
\emph{得到电子的物质是}\Concept{氧化剂},\emph{失去电子的物质是}\Concept{还原剂}。

在下面的化学方程式里,用箭头表示不同元素间电子得失的情况。


但是,也有一些反应,如上面讲到的氢气跟氢气生成氯化氢的反应,生成的氯化氢是共价化合物。
我们在初中化学里已经知道,氯化氢分子里的共用电子对是偏向于氯原子,偏离于氢原子的。


在上述电子转移过程里,就没有那么显著,也就是并没有完全失去或完全得到电子,而是共用电子对的偏移。
这样的反应也属于氧化—还原反应。在这个反应里,氯气是氧化剂,氢气是还原剂。

氧化—还原反应里,电子转移(得失或偏移)、正负化合价升降的关系,可以用\cref{fig:2-11} 表示。
\begin{figure}
  \caption{氧化—还原反应中电子得失、化合价变化的关系简图}\label{fig:2-11}
\end{figure}

\begin{example}
  分析镁跟稀盐酸的反应里镁元素和氢元素在反应前后的电子得失、化合价升降和氧化、还原的关系。哪一种物质是氧化剂,哪一种物质是还原剂?
\end{example}
\begin{solution}
  写出反应的化学方程式,并用箭头表示镁元素和氢元素在反应前后的电子得失等。
  \[ \]
  答: \ce{Mg} 是还原剂,\ce{HCl} 是氧化剂。
\end{solution}

根据对许多例子的分析,可以说凡是没有电子转移、也就是没有正负化合价升降的反应,就不属于氧化—还原反应。

在工农业生产、科学技术和日常生活里,我们经常会碰到许多氧化—还原反应。
所以氧化—还原反应是很重要的一类化学反应。

\begin{Practice}[习题]
  \begin{question}
    \item 说明在下列反应里的氧化、还原、化合价升降的关系。
    \begin{tasks}
      \task \ce{ 2P + 3Cl2 $\xlongequal{\quad}$ 2PCl3}
      \task \ce{ C + O2 $\xlongequal{\quad}$ CO2}
      \task \ce{ 2Sb + 5Cl2 $\xlongequal{\quad}$ 2SbCl5}
      \task \ce{ 2KClO3 $\xlongequal{\quad}$ 2KCl + 3O2 }
    \end{tasks}
    \item 你怎样理解氧化—还原反应跟电子得失的关系?举例说明。
    \item 分析下列化学反应中化合价变化的关系,由此说明反应中的电子得失。
    \begin{tasks}
      \task \ce{ 2Mg + O2 $\xlongequal{\quad}$ 2MgO }
      \task \ce{ Zn + 2HCl $\xlongequal{\quad}$ ZnCl2 + H2 ^}
    \end{tasks}
    \item 分析下列氧化-还原反应里电子的转移情况,哪种物质是氧化剂,哪种物质是还原剂?
    \begin{tasks}
      \task \ce{ Zn + H2SO4 $\xlongequal{\quad}$ ZnSO4 + H2 ^}
      \task \ce{ Fe + 2HCl $\xlongequal{\quad}$ FeCl2 + H2 ^}
    \end{tasks}
    \item 写出氯气分别跟钾、钙起反应的化学方程式。在各个反应里分别指出不同元素间电子转移的关系。在各个反应里, 哪种物质是氧化剂,哪种物质是还原剂?
  \end{question}
\end{Practice}

\section{卤族元素}
我们已经学过氯元素的单质和一些重要的化合物。
现在学习氟、溴、碘等元素,并把它们跟氯元素相比较,它们在性质上、原子结构上有哪些相似和不同的地方。

\subsection{卤素的原子结构和它们的单质的物理性质}
卤素在自然界里都以化合态存在,它们的单质可由人工制得。
卤素的单质都是双原子的分子。
\cref{tab:2-1} 列出各元素的原子结构和单质的物理性质。

\begin{table}
  \caption{卤素的原子结构和单质的物理性质}\label{tab:2-1}
\end{table}

从\cref{tab:2-1} 可以看出,氟、氯、溴、碘的原子的最外电子层的电子数是相同的,都是 7 个电子,但电子层数不同。
因此,它们的原子半径或离子半径\footnote{离子半径是根据阴阳离子的核间距推算出来的。阴离子的半径比它们的原子半径大,阳离子的半径比它们的原子半径小。}都随着电子层数的增多而增大(\cref{fig:2-12}),它们的离子都因得到了一个电子,离子半径比相应的原子半径增大了。

\begin{figure}
  \caption{卤素的原子和离子大小示意图(数据单位是 \qty{e-10}{m})}\label{fig:2-12}
\end{figure}

从\cref{tab:2-1} 还可以看出,卤素的物理性质有较大的差别。
如在常温,氟、氯是气体,溴是液体,碘是固体,它们的沸点、熔点都逐渐升高,颜色由淡黄绿色到紫黑色,逐渐转深。

\begin{Experiment}
  打开盛溴的瓶的盖子,有什么现象发生?观察液态和气态的溴。
\end{Experiment}

可以观察到,液态溴容易挥发成溴蒸气。

\begin{Experiment}*[righthand ratio=0.5]
  观察碘的颜色、状态和光泽。把少量的碘晶体放在烧杯里,烧杯上放盛冷水的烧瓶,稍稍加热(\cref{fig:2-13})。观察发生的现象。
  % \tcblower
  % \begin{figurehere}
  %   \caption{碘的升华}\label{fig:2-13}
  % \end{figurehere}
\end{Experiment}

可以观察到,碘在常压下加热,不经过熔化就直接变成紫色蒸气,蒸气遇冷,重新凝成固体。
这种固态物质不经过转变成液态而直接变成气态的现象叫做\Concept{升华}。
\begin{figure}
  \begin{minipage}[b]{0.48\linewidth}\centering
    \caption{碘的升华}\label{fig:2-13}
  \end{minipage}
  \begin{minipage}[b]{0.48\linewidth}\centering
    \caption{溴在不同溶剂里的溶解}\label{fig:2-14}
  \end{minipage}
\end{figure}

\begin{Experiment}
  把水注入盛着少量溴的试管,振荡,水溶液的颜色显橙色(图 2-14,I)。把上部橙色溶液倒在另一个试管里,再注入少量无色的汽油(或苯或四氯化碳)(图 2-14,II),用力振荡,静置一会儿。观察油层和水溶液的颜色。
\end{Experiment}

\begin{Experiment}
  把水、酒精分别注入两个试管,各约占小半试管,并各投入少量的碘的晶体,振荡。比较碘在两种液体里的溶解性。把碘的水溶液注入另一空试管,再注入少量无色汽油(或苯或四氯化碳),振荡,静置一会儿,观察油层和溶液的颜色。
\end{Experiment}

可以观察到溴和碘都比较容易溶解于汽油、苯、四氯化碳、酒精等有机溶剂中。医疗上用的碘酒,即是碘的酒精溶液。

\subsection{卤素的单质的化学性质}
我们知道,氯的化学性质很活泼,它的原子的最外电子层是 7 个电子,在化学反应中容易得到一个电子而成为 8 个电子的稳定结构。
氟、溴、碘的原子的最外电子层也都是 7 个电子,因而它们的化学性质跟氯有很大的相似性。
\subsubsection{卤素都能跟金属起反应生成卤化物}


氟、溴、碘都能象氯一样跟钠等金属起反应。自然界里,也存在着许多种的金属跟卤素的化合物,如氟化钙、氯化钠、 氯化镁、溴化钾、碘化钾等等卤化物。

\subsubsection{卤素都能跟氢气起反应,生成卤化氢}
\begin{Experiment}
  在一个铁坩埚里放锌粉 \qty{0.5}{g},加入碘粉 \qty{0.5}{g},混和后,加水 1~2 滴作为催化剂,观察发生的现象。
\end{Experiment}
氟的性质比氯更活泼,氟气跟氢气的反应不需光照,在暗处就能剧烈化合,并发生爆炸。
\[ \ce{H2}\,\text{(气)} + \ce{F2}\,\text{(气)} \xlongequal{\quad} \ce{2HF}\,\text{(气)} + \qty{128.4}{kCal}\]

溴的性质不如氯活泼,溴跟氢气的反应在达到 \qty{500}{\celsius} 时即较慢地进行。
\[ \ce{H2}\,\text{(气)} + \ce{Br2}\,\text{(气)} \xlongequal{\quad} \ce{2HBr}\,\text{(气)} + \qty{17.3}{kCal}\]

碘的性质比溴更不活泼,碘跟氢气的反应在不断加热条件下缓慢地进行,生成的碘化氢很不稳定,同时发生分解。

\[ \ce{H2}\,\text{(气)} + \ce{I2}\,\text{(固)} \xlongequal{\triangle} \ce{2HI}\,\text{(气)} - \qty{12.4}{kCal}\]

卤素也能跟磷等非金属起反应。

\subsubsection{卤素跟水的反应}
氟遇水发生剧烈的反应,生成氟化氢和氧气。
\[ \ce{2H2O + 2F2 \xlongequal{\quad} 4HF + O2 ^}\]

溴跟水的反应比氯气跟水的反应更弱一些,碘跟水只有很微弱的反应。
\subsubsection{卤素各单质的活动性比较}
\begin{Experiment}
  把少量氯水分别注入盛着溴化钠溶液和碘化钾溶液的两个试管里,用力振荡后,再注入少量无色汽油 (或四氯化碳)。振荡。观察油层和溶液颜色的变化。
\end{Experiment}
\begin{Experiment}
  把少量溴水注入盛着碘化钾溶液的试管里,用力振荡。观察溶液颜色的变化。
\end{Experiment}

溶液颜色的变化,说明氯可以把溴或碘从它们的化合物里置换出来,溴可以把碘从它的化合物中置换出来。
\begin{gather*}
  \ce{ 2NaBr +Cl2 \xlongequal{\quad} 2NaCl +Br2 } \\ 
  \ce{ 2KI +Cl2 \xlongequal{\quad} 2KCl +I2 } \\ 
  \ce{ 2KI +Br2 \xlongequal{\quad} 2KCBr +I2 } 
\end{gather*}

由此可以证明,在氯、溴、碘这三种元素里,氯比溴活泼,溴又比碘活泼。
实验证明,氟的性质比氯、溴、碘更活泼,能把氯等从它们的卤化物中置换出来。

氟的性质特别活泼,它甚至还能够跟惰性气体中的氙、氪等起反应,生成氙和氪的氟化物\footnote{氙和氪的氟化物的结构比较特殊,不宜应用通常的正负化合价来解释。}:\ce{XeF2}、\ce{XeF4}、\ce{XeF6}、\ce{KrF2}等。
它们在常温下都是白色固体。

\subsubsection{碘跟淀粉的反应}
碘遇淀粉变蓝色。利用碘的这个特性,可以鉴定碘的存在。
\begin{Experiment}
  在试管里注入少量淀粉溶液,滴入几滴碘水,溶液显示出特殊的蓝色。
\end{Experiment}

从卤素的化学性质可以看出,它们有很多相似的地方,但也有差别(\cref{tab:2-2})。
卤素的原子,最外电子层都有 7 个电子,结合外来电子的能力很强,所以卤素是活泼的非金属元素。
卤素容易得到电子而被还原,它们本身是强氧化剂。
但是,氟、氯、溴、碘各原子的核电荷数不同、核外电子层数不同,原子和离子的大小也都不同,各原子核对外层电子的引力也有所不同。
原子的大小对非金属的活动性有很密切的关系。
氟的原子较小,外层电子受到核的引力最强,它得到电子的能力很强,非金属性最活泼,所以合成的氟化氢最稳定,合成的时候反应放热,并最剧烈。
碘的原子较大,最外层电子受到核的引力较弱,它得到电子的能力也较弱,非金属性也较弱,生成的碘化氢不稳定,合成的时候要吸热。
氯和溴的非金属性是介乎其间的,氯比溴又活泼一些。
总的看来,卤素是活泼的非金属元素,它们的活动性又随着核电荷数和电子层数的增加、原子半径的增大而减弱。

\begin{table}
  \caption{卤素的单质的化学性质比较}\label{tab:2-2}
\end{table}

\begin{Theorem}{讨论}
  从哪些方面可以比较氟、氯、溴、碘的性质的相似和不同之处?
\end{Theorem}



\subsection{卤素的几种化合物}
\subsubsection{氟化氢和氟化钙}
氟化钙俗名萤石,是自然界里存在相当广泛的氟的化合物。

使浓硫酸跟萤石在铅皿中起反应,就制得氟化氢。
\[ \ce{ CaF2 + H2SO4 \xlongequal{\quad} CaSO4 + 2HF ^ }\]

氟化氢也象氯化氢那样,在空气里呈现白雾。
氟化氢有剧毒。
氟化氢溶解在水里就是氢氟酸。

氟化氢用于雕刻玻璃和制造塑料、橡胶、药品等,还用于制备单质氟和提炼铀。
氟化氢还用于制造氟化钠等氟化物。
氟化钠是一种用来杀灭地下害虫的农药。

\subsubsection{溴化银和碘化银}
\begin{Experiment}
把少量硝酸银溶液分别滴入盛着溴化钠溶液和碘化钾溶液的两个试管。
在盛溴化钠溶液的试管里有浅黄色的溴化银沉淀生成,在盛碘化钾溶液的试管里有黄色的碘化银沉淀生成。
在两个试管里各加入少量稀硝酸,生成的溴化银、碘化银沉淀都不溶解。
\end{Experiment}
\begin{gather*}
  \ce{ NaBr + AgNO3 \xlongequal{\quad} AgBr v + NaNO3}\\
  \ce{ KI + AgNO3 \xlongequal{\quad} AgI v + KNO3 }
\end{gather*}

溴化银和碘化银都有感光性,在光的照射下会起分解反应。例如:
\[ \ce{ 2AgBr \xlongequal{\text{光照}} 2Ag + Br2} \]

照相用的感光片,就是在暗室里用溴化银的明胶凝胶,均匀地涂在胶卷或玻璃片上制成的。
照相时利用溴化银有感光性的原理,使底片感光后,再用还原剂(显影剂)和定影剂处理,得到明暗程度跟实物相反的底片。使感光片通过底片曝光,再经显影和定影,就得到明暗程度跟实物一致的照片。

碘化银可用于人工降雨,用小火箭、高射炮等工具把磨成很细粉末的碘化银发射到几千米的高空,使空气里的水蒸气凝聚成雨。

\begin{Practice}[习题]
  \begin{question}
    \item 写出氟、溴、碘跟金属钠的反应的化学方程式。
    \item 在三个试管里,分别盛着氯化钠、溴化钠、碘化钾的溶液。各加入一些氯水,发生什么反应?再加入一些溶剂汽油,振荡。出现什么现象?
    \item 从哪些化学性质可以证明氟是卤素中最活泼的元素?
    \item 日光照射在下列物质上,各有什么现象发生?为什么?写出化学方程式。
    \begin{tasks}(3)
      \task 氯水
      \task 氯气和氢气的混和物
      \task 溴化银
    \end{tasks} 
    \item 怎样鉴别单质碘和碘离子?
    \item 现有二氧化锰、氯化钾、溴化钾、浓硫酸和水五种物质,怎样从这些物质来制取盐酸、氯气和溴。写出反应的化学方程式。
    \item 氟化钙跟浓硫酸起反应,制得氟化氢,写出反应的化学方程式。为什么这个反应的装置不能用玻璃仪器?
  \end{question}
\end{Practice}

\section*{内容提要}
卤素是一族非金属元素。
它们的原子结构的共同之点是最外电子层都有 7 个电子,在化学反应中容易得到电子;差别之处是核电荷数不同、电子层数不同,原子半径也不同,由此形成了卤族元素既相似又有着差别的性质。
它们的化学性质主要是强的非金属性,它们的单质都是强氧化剂。
卤素里,氟原子很小,非金属性很强,氯、溴、碘随着原子的增大而非金属性减弱。

卤素的化学性质包括:
\begin{itemize}
  \item 跟金属的反应——生成金属卤化物。
  \item 跟氢气的反应——生成卤化氢。
  \item 跟水的反应——生成氢卤酸、次卤酸。
  \item 跟氢氧化物的反应——生成金属卤化物等。
  \item 跟卤化物的反应——把较不活泼的卤素从它们的卤化物溶液里置换出来。
\end{itemize}

物质失去电子的反应是氧化反应,物质得到电子的反应是还原反应。氧化和还原反应必然同时发生。
\begin{Review}
  \begin{question}
    \item 下列说法是否正确,如不正确,加以改正。
    \begin{tasks}
      \task 氯酸钾里有氧气,所以加热时有氧气放出。
      \task 卤素各单质都可以成为氧化剂。
      \task \qty{1}{mol} 液态 \ce{HCl} 在标准状况时约占 \qty{22.4}{L}。
    \end{tasks}
    \item 能不能在下列条件下收集氯气?说明原因,并写出可能发生的反应的化学方程式。
    \begin{tasks}(3)
      \task 排水收集
      \task 排氢氧化钾溶液收集
      \task 排碘化钾溶液收集
    \end{tasks}
    \item 在用氯酸钾制氧气和浓盐酸制氯气时都要用到二氧化锰,二氧化锰的作用是否相同,分别是起了什么作用?
    \item 实验室里制备氢气、氯气、二氧化碳都要用到盐酸。盐酸在制备这三种气体的反应里,各起什么作用?
    \item 从电子得失的观点来看,在已经学过的化学反应类型里,置换反应都属于氧化—还原反应,一部分化合反应和分解反应属于氧化—还原反应,复分解反应都不属于氧化—还原反应。你认为这个结论合理吗? 举例说明。
    \item 一包白色固体,可能是 \ce{CaCl2}、\ce{Na2CO3}、\ce{NaI} 三种物质之一,也可能是两者或三者的混和物。当把白色固体溶解于水时,发现有白色沉淀。过滤后,滤液是无色的。
    \begin{tasks}
      \task 把滤纸上的沉淀,移到试管里,加入盐酸,有气体产生。把气体通入澄清石灰水,石灰水显浑浊。
      \task 把滤液分成两部分。向一部分滤液加入几滴硝酸银溶液,出现白色沉淀。再加稀硝酸,白色沉淀不溶解。这白色沉淀光照后,逐渐呈黑色。向另一部分滤液里加几滴氯水,再注入少许四氯化碳,振荡,四氯化碳层不显紫色。
    \end{tasks}
    试分析并判断这包白色固体里含有哪些物质。写出反应的化学方程式。
    \item 使足量浓硫酸跟 \qty{11.7}{g} 氯化钠混和微热,反应生成的氯化氢通入 \qty{45}{g} 10\% 氢氧化钠溶液,向最后的溶液滴入几滴石蕊试液,将显什么颜色。
    \item 浓盐酸跟二氧化锰起反应,生成的氯气能从碘化钠溶液里置换出 \qty{1.27}{g} 碘。计算至少需用多少摩尔的 \ce{HCl} 和 \ce{MnO2} 。
    \item 把滤纸用淀粉和碘化钾的溶液浸泡,晾干后就是实验室常用的淀粉碘化钾试纸。这种试纸润湿后遇到氯气会发生什么变化?为什么?
  \end{question}
\end{Review}