\chapter{硫\texorpdfstring{\quad}{ }硫酸}
硫是一种重要的非金属元素。
硫的原子结构和性质跟我们已经学过的氧很相似,它们的原子的最外电子层都有 6 个电子,氧(\ce{O})、硫(\ce{S})和另外三种原子结构和性质相似的元素硒(\ce{Se})、碲(\ce{Te})、钋(\ce{Po})\footnote{硒音\pinyin{xi1},碲音\pinyin{di4},钋音\pinyin{po1}。},统称为氧族元素。
本章主要介绍硫及其化合物的知识。

\section{硫}
在自然界,游离态的天然硫, 存在于火山喷口附近或地壳的岩层里。
由于天然硫的存在,人类从远古时代起就知道硫了。
以化合态存在的硫分布很广,主要是硫化物和硫酸盐,如硫铁矿(\ce{FeS2}),黄铜矿(\ce{CuFeS2}),石膏(\ce{CaSO4.2H2O}),芒硝(\ce{Na2SO4.10H2O}),等等。
硫的化合物也常存在于火山喷 出的气体中和矿泉水里。
煤和石油里都含有少量硫。
硫还是某些蛋白质的组成元素,是生物生长所需要的一种元素。

\subsection{硫的物理性质}
硫通常是一种淡黄色的晶体,俗称硫黄。
它的密度大约是水的两倍。
硫很脆,容易研成粉末,不溶于水,微溶于酒精,容易溶于二硫化碳。
硫的熔点是 \qty{112.8}{\celsius},沸点是 \qty{444.6}{\celsius}。

\subsection{硫的化学性质}
硫的化学性质比较活泼,跟氧相似,容易跟金属、氢气和其它非金属发生反应。

\subsubsection{硫跟金属的反应}
\begin{Experiment}*[righthand ratio=0.7]
给盛着硫粉的大试管加热。加热到硫沸腾产生蒸气时,用坩埚钳夹住一束擦亮的细铜丝伸入管口(\cref{fig:3-1}),观察发生的现象。
  \tcblower 
  \begin{figurehere}
    \caption{铜在硫蒸气里燃烧}\label{fig:3-1}
  \end{figurehere}
\end{Experiment}
铜丝在硫蒸气里燃烧,生成黑色的硫化亚铜。
\[ \ce{ 2Cu + S \xlongequal{\triangle} Cu2S}\]

\begin{Experiment}
把少量硫粉和铁粉的混和物装在试管里,加热到红热,立即把酒精灯移开。
反应里放出的热,就能使反应继续进行。观察发生的现象。
\end{Experiment}

硫跟铁起反应,生成黑色的硫化亚铁。
\[ \ce{ Fe + S \xlongequal{\triangle} FeS} \]

硫还能跟其它金属起反应。硫跟金属的化合物,叫做金属硫化物。

\subsubsection{硫跟非金属的反应}
在初中化学里,我们已经知道硫能跟氧气发生反应,生成二氧化硫,并放出大量的热。
\[ \ce{S}\,\text{(固)} + \ce{O2}\,\text{(气)}\ce{ \xlongequal{\quad} SO2}\,\text{(气)} + \qty{71}{kCal}\]

此外,硫还能跟其它非金属发生反应。例如,硫的蒸气能跟氢气直接化合而生成硫化氢气体:
\[ \ce{S + H2 \xlongequal{\triangle} H2S }\]

\subsection{硫的用途}
硫的用途很广。
硫主要用来制造硫酸。
硫也是生产橡胶制品的重要原料。
硫还可用于制造黑色火药、焰火、火柴等。
硫又是制造某些农药(如石灰硫黄合剂)的原料。
医疗上,硫还可用来制硫黄软膏医治某些皮肤病,等等。
\begin{Practice}[习题]
  \begin{question}
    \item 写出硫跟氢气、硫跟氧气反应的化学方程式。
    \item 试计算 \qty{0.32}{kg} 的硫燃烧生成二氧化硫,放出多少热量。
    \item \qty{21}{g} 铁粉跟 \qty{8}{g} 硫粉混和加热可生成硫化亚铁多少克,哪一种物质过剩,剩余多少?
  \end{question}
\end{Practice}
\section{硫的氢化物和氧化物}
\subsection{硫的氢化物——硫化氢(\texorpdfstring{\ce{H2S}}{H2S})}
\subsubsection{硫化氢的实验室制法}
在实验室里,硫化氢通常是由硫化亚铁跟稀盐酸或稀硫酸反应而制得的。
\[ \ce {FeS +2HCl \xlongequal{\quad} FeCl2 + H2S ^}\]
\[ \ce {FeS +H2SO4 \xlongequal{\quad} FeSO4 + H2S ^}\]

这个反应可以在启普发生器里进行。

\subsubsection{硫化氢的性质}
硫化氢是一种没有颜色而有臭鸡蛋气味的气体。
它的密度比空气略大。
硫化氢有剧毒,是一种大气污染物。
空气里如果含有微量的硫化氢,就会使人感到头痛、头晕和恶心。
吸入较多的硫化氢,会使人昏迷甚至死亡。
因此,制取或使用硫化氢时,必须在密闭系统或通风橱中进行。

硫化氢能溶于水。
在常温、常压下,1 体积的水能溶解 2.6 体积的硫化氢。

在较高温度时,硫化氢分解成氢气和硫。
\[ \ce{H2S \xlongequal{\triangle} H2 + S}\]

硫化氢是一种可燃性气体。
\begin{Experiment}
在导管口用火点燃硫化氢气体,观察硫化氢完全燃烧时火焰的颜色。
然后把干燥的烧杯罩在火焰的上方,观察烧杯内壁附有什么物质。小心地闻一下气味。
\end{Experiment}

在空气充足条件下,硫化氢能完全燃烧而发生淡蓝色的火焰,并生成水和二氧化硫。
\[ \ce{ 2H2S + 3O2 \xlongequal{\text{点燃}} 2H2O +2SO2} \]

\begin{Experiment}
在导管口用火点燃硫化氢气体,用一个蒸发皿(或玻璃片),使蒸发皿底靠近硫化氢的火焰,观察蒸发皿底发生的现象。
\end{Experiment}
我们可以看到,蒸发皿底部附有黄色的粉末。这是硫化氢不完全燃烧时析出的单质硫。
\[ \ce{ 2H2S + O2 \xlongequal{\quad} 2H2O + 2S}  \]

如果在一个集气瓶里,使硫化氢跟二氧化硫两种气体充分混和。不久,在瓶壁上就有黄色的粉末一一硫的生成。
\[ \ce{ SO2 +2H2S \xlongequal{\quad} 2H2O + 3S} \]

由此可见,硫化氢具有还原性。
硫化氢里的硫是 $-2$ 价,它能够失去电子而变成游离态的单质硫或高价硫的化合物。

硫化氢的水溶液能够使石蕊试液变为浅红色,它是一种酸,叫做氢硫酸,当这种酸受热时,硫化氢又从水里逸出。
氢硫酸是一种弱酸,它具有酸的通性。

\subsection{硫的氧化物}
硫的氧化物中最重要的是二氧化硫和三氧化硫。

\subsubsection{二氧化硫(\ce{SO2})}
二氧化硫是没有颜色而有刺激性气味的有毒气体。
它的密度比空气大,容易液化(沸点是 \qty{-10}{\celsius}),易溶于水,在常温、 常压下,1 体积水大约能溶解 40 体积二氧化硫。

二氧化硫是酸性氧化物,它跟水化合而生成亚硫酸(\ce{H2SO3})。
因此,二氧化硫又叫做亚硫酐。
\[ \ce{ SO2 + H2O \xlongequal{\quad} H2SO3 }\]

亚硫酸很不稳定,容易分解成二氧化硫和水。
\[ \ce{ H2SO3 \xlongequal{\quad} H2O + SO2 ^} \]

通常把向生成物方向进行的反应叫做\Concept{正反应},向反应物方向进行的反应叫做\Concept{逆反应}。
象这种\emph{在同一条件下,既能向正反应方向进行,同时又能向逆反应方向进行的反应,叫做}\Concept{可逆反应}。
在化学方程式里,用两个方向相反的箭头代替等号来表示可逆反应。
\[ \ce{ SO2 + H2O <=> H2SO3 }\]

二氧化硫在适当的温度并有催化剂存在的条件下,可以被氧气氧化而生成三氧化硫。
\[ \ce{ 2SO2 +O2} \xlongequal[\triangle]{\text{催化剂}} \ce{2SO3} \]

\begin{Experiment}
把二氧化硫气体通入盛有品红溶液的试管里,观察品红溶液颜色的变化。把试管加热,再观察溶液发生的变化。
\end{Experiment}

\begin{Reading}[]{补充阅读}
亚硫酸钠是亚硫酸(\ce{H2SO3})的钠盐。亚硫酸钠溶液跟硫反应,生成硫代硫酸钠(\ce{Na2S2O3})。
\[ \ce{ Na2SO3 +S \xlongequal{\triangle} Na2S2O3} \]

硫代硫酸钠是硫代硫酸(\ce{H2S2O3})的钠盐。
硫代硫酸可以看作是硫酸分子中的一个氧原子被硫原子代换后所生成的酸。
带有五个结晶水的硫代硫酸钠(\ce{Na2S2O3.5H2O}),俗称大苏打或海波。
它是无色晶体,溶于水,在照相业中常用作定影剂,用于溶解照相底片或感光纸上尚未感光的溴化银。
\end{Reading}

\subsubsection{三氧化硫(\ce{SO3})}

\section{硫酸的工业制法——接触法}
\subsection{接触法制造硫酸的反应原理和生产过程}
\subsection{尾气中二氧化硫的回收和环境保护}
\begin{Practice}[习题]
  \begin{question}
    \item 
    \item 
    \item 
    \begin{tasks}
      \task
      \task
      \task
    \end{tasks}
    \item 
  \end{question}
\end{Practice}
\section{硫酸\texorpdfstring{\quad}{ }硫酸盐}
\subsection{硫酸}
\subsection{硫酸盐}
\subsection{硫酸根离子的检验}
\begin{Practice}[习题]
  \begin{question}
    \item 
    \begin{tasks}
      \task
      \task
      \task
      \task
      \task
    \end{tasks}
    \item 
    \item 
    \item 
    \item 
  \end{question}
\end{Practice}
\section{离子反应\texorpdfstring{\quad}{ }离子方程式}
\subsection{离子反应\texorpdfstring{\quad}{ }离子方程式}
\subsection{离子反应发生的条件}
\begin{Practice}[习题]
  \begin{question}
    \item 
    \begin{tasks}
      \task
      \task
      \task
      \task
      \task
      \task
    \end{tasks}
    \item 
    \item 
    \begin{tasks}
      \task
      \task
      \task
      \task
      \task
    \end{tasks} 
    \item 
    \item 
  \end{question}
\end{Practice}
\section{氧族元素}
\begin{Practice}[习题]
  \begin{question}
    \item 
    \item 
  \end{question}
\end{Practice}
\section*{内容提要}
\subsection{氧族元素}
\subsection{硫的化学性质}
\subsection{硫的重要氧化物}
\subsection{硫酸}
\subsection{离子反应和离子方程式}
\begin{Review}
  \begin{question}
    \item
    \begin{tasks}(5)
      \task 氧气
      \task 氯气
      \task 氯化氢
      \task 硫化氢
      \task 二氧化硫
    \end{tasks}
    \item 浓硫酸和稀硫酸的性质有哪些不同的地方?
    \item 回答下列问题:
    \begin{tasks}
      \task 
      \task 
      \task 
    \end{tasks}
    \item
    \item
    \item
    \item
    \item 在下列反应里,二氧化硫是氧化剂还是还原剂,为什么?
    \begin{tasks}
      \task \ce{2H2S + SO2 $\xlongequal{\quad}$ 3S +2H2O}
      \task \ce{Br2 + SO2 +2H2O $\xlongequal{\quad}$ H2SO4 + 2HBr}
    \end{tasks}
    \item 写出下列物质变化的化学方程式(其中的离子反应还要写出相应的离子方程式):
    \[ \ce{FeS2} \to \ce{SO2} \to \ce{SO3} \to \ce{H2SO4} \to \ce{CuSO4} \to \ce{Cu} \]
  \end{question}
\end{Review}