\chapter{原子结构\texorpdfstring{\quad}{ }元素周期律}
到目前为止,我们已经学习了氧、惰性气体、氢、碳、卤族、氧族、碱金属等元素和它们的一些化合物,知道元素、化合物的性质跟它们的结构有密切的关系。
只有了解了它们的结构,才能深刻地认识它们的性质和变化规律。
所以,我们要在初中学习的物质结构初步知识的基础上,进一步学习物质结构的知识。
本章我们先来学习有关原子结构和反映元素内在联系的元素周期律的知识。
至于物质结构的其它知识,以后将要逐步学习。

\section{原子核}
\subsection{原子核}
原子是由居于原子中心的带正电的原子核和核外带负电的电子构成的。
由于原子核带的电量跟该外电子的电量相等而电性相反,因此,原子作为一个整体不显电性。
原子很小,而原子核更小,它的半径约是原子的万分之一,它的体积只占原子体积的几千亿分之一。
原子核由质子和中子构成。
每个质子带一个单位正电荷,中子呈电中性,因此,核电荷数由质子数决定。
核电荷数的符号为 $Z$。
\[ \text{核电荷数}(Z) = \text{核内质子数} = \text{核外电子数}\]

质子的质量为 \qty{1.6726e-27}{kg},中子的质量稍大些,为 \qty{1.6748e-27}{kg},电子的质量很小,仅约为质子质量的 1/1836,所以,原子的质量主要集中在原子核上。
由于质子、 中子的质量很小,计算不方便,因此,通常用它们的相对质量。

通过科学实验测得,作为原子量标准的那种碳原子的质量是 \qty{1.9927e-26}{kg},它的 1/12 为 \qty{1.6606e-27}{kg}。
质子和中子对它的相对质量分别为 1.007 和 1.008,取近似整数值为 1。
显然,如果忽略电子的质量,将原子核内所有的质子和中子的相对质量取近似整数值加起来,所得的数值,叫做质量数,用符号 $A$ 表示。中子数用符号 $N$ 表示。则
\[ \text{质量数}(A)= \text{质子数}(Z)+\text{中子数}(N)\]

因此,只要知道上述三个数值中的任意两个,就可以推算出另一个数值来。
例如,知道硫原子的核电荷数为 16,质量数为 32,则
\[\text{硫原子的中子数}=A-Z= 32-16=16\]

归纳起来,如以 \ce{^$A$_$Z$ $X$} 代表一个质量数为 $A$、质子数为 $Z$ 的原子,那么,组成原子的粒子间的关系可以表示如下:
\[ \text{原子}\; \ce{^$A$_$Z$ $X$} 
     \left\{ \begin{array}{l}
       \text{原子核} \left\{
        \begin{array}{ll}
          \text{质子} & Z \text{个}\\
          \text{中子} & (A-Z) \text{个}\\
        \end{array}
       \right.\\
       \text{核外电子}\quad Z \text{个}    
    \end{array}
     \right.
\]

\subsection{同位素}
我们已经知道,具有相同核电荷数 (即质子数) 的同一类原子叫做元素。
也就是说,同种元素的原子的质子数相同,那么,它们的中子数是否相同呢? 
科学研究证明,不一定相同。
例如,氢元素的原子都含 1 个质子,但有的氢原子不含中子,有的氢原子含 1 个中子,还有的氢原子含 2 个中子:
\begin{itemize}[label={},leftmargin=2em]
  \item 不含中子的氢原子叫做氕;
  \item 含 1 个中子的氢原子叫做氘,就是重氢;
  \item 含 2 个中子的氢原子叫做氚\footnote{氕音\pinyin{pie1},氘音\pinyin{dao1},氚音\pinyin{chuan1}。},就是超重氢。
\end{itemize}

为了便于区别,将氕记为 \ce{^1_1H},氘记为 \ce{^2_1H}(或 \ce{D}),氚记为  \ce{^3_1H}(或 \ce{T})。
元素符号的左下角记核电荷数,左上角记质量数。

人们把原子里具有相同的质子数和不同的中子数的同一元素的原子互称同位素。
许多元素都有同位素。上述 \ce{^1_1H}、\ce{^2_1H}、\ce{^3_1H} 是氢的三种同位素,\ce{^2_1H}、\ce{^3_1H} 是制造氢弹的材料。
铀元素有 \ce{^234_92U}、\ce{^235_92U}、\ce{^238_92U} 等多种同位素,\ce{^235_92U} 是制造原子弹的材料和核反应堆的燃料。
碳元素有 \ce{^12_6C}、\ce{^13_6C} 和 \ce{^14_6C} 等几种同位素,而 \ce{^12_6C} 就是我们将它的质量的 1/12 当做原子量标准的那种碳原子。
同一元素的各种同位素虽然质量数不同,但它们的化学性质几乎完全相同。
在天然存在的某种元素里,不论是游离态还是化合态,各种同位素所占的原子百分比一般是不变的。
我们平常所说的某种元素的原子量,是按各种天然同位素原子所占的一定百分比算出来的平均值。
例如,元素氯是 \ce{^35_17Cl} 和  \ce{^37_17Cl} 两种同位素的混和物,从下列数据即可计算出氯元素的原子量:

\begin{tabular}{ccc}
  符号 & 同位素的原子量 & 在自然界各同位素原子的百分组成 \\
  \ce{^35_17Cl} & 34.969 & 75.77 \\
  \ce{^37_17Cl} & 36.966 & 24.23 \\
\end{tabular}
\[ 36.969\times 0.7577 + 36.966 \times 0.2423 = 35.453\]
即氯的原子量为 35.453。

同理,根据同位素的质量数,也可以算出近似原子量。
\begin{Practice}[习题]
  \begin{question}
    \item 下列说法是否正确? 如有错误,加以改正。
    \begin{tasks}
      \task 石墨和金刚石是由碳元素组成的两种同位素。
      \task 人们已经知道了 107 种元素,就是说人们已经知道了 107 种原子。
    \end{tasks}
    \item 指出下列各原子中质子、中子、电子的数目各是多少:
    \begin{gather*}
      \ce{^12_6C},\quad\ce{^13_6C},\quad\ce{^16_8O},\quad\ce{^17_8O},\quad\ce{^18_8O},\quad\ce{^19_9F},\quad\ce{^24_12Mg}\\
      \ce{^39_19K},\quad\ce{^40_19K},\quad\ce{^41_19K},\quad\ce{^40_20Ca},\quad\ce{^42_20Ca}
    \end{gather*}
    \item 氧有三种天然同位素,它们的同位素原子量和各同位素原子的百分组成数据分列如下:\par
    \begin{tabular}{ccc}
      \ce{^16_8O} & 15.994915 & 99.759\% \\
      \ce{^17_8O} & 16.999133 &  0.037\% \\
      \ce{^18_8O} & 17.99916  &  0.204\% \\
    \end{tabular}

    \noindent 计算氧元素的原子量。
    \item 镁有三种天然同位素: \ce{^24_12Mg}(占 78.7\%),\ce{^25_12Mg}(占 10.13\%),\ce{^26_12Mg}(占 11.17\%),计算镁元素的近似原子量。
  \end{question}
\end{Practice}

\section{核外电子的运动状态}
电子带负电荷,质量很小,仅 \qty{9.1095e-31}{kg}。
它在原子这样大小的空间(直径约 \qty{e-10}{m})内运动,速度很快,接近光速(\qty{3e8}{m/s})。
电子的运动情形跟质量大、速度小的普通物体是否相同?
有没有特殊的规律?
现在我们就来进行研究。

\subsection{电子云}
我们在生活中见到汽车在公路上奔驰,用仪器观察到人造卫星按一定轨道围绕地球旋转,都可以测定或根据一定的数据计算出它们在某一时刻所在的位置,并描画出它们的运动轨迹。
但是,核外电子的运动规律就跟上述普通物体不同。
核外电子的运动没有上述那样确定的轨道,我们不能测定或计算出它在某一时刻所在的位置,也不能描画它的运动轨迹。
我们在描述核外电子运动时,只能指出它在原子核外空间某处出现机会的多少。
电子在核外空间一定范围内出现,好象带负电荷的云雾笼罩在原子核周围,所以我们形象地称它为“电子云”。
为了便于理解,我们用给氢原子照像的比喻来加以说明。
我们知道,氢原子核外有一个电子。
为了在一瞬间找到电子在氢原子核外的确定位置,我们假想有一架特殊的照相机,可以用它来给氢原子照相。
先给某个氢原子拍五张照片,得到如\cref{fig:5-1} 所示的不同的图象。
\begin{figure}
  \begin{minipage}{0.19\linewidth}\centering
    \subcaption{}\label{fig:5-1a}
  \end{minipage}
  \begin{minipage}{0.19\linewidth}\centering
    \subcaption{}\label{fig:5-1b}
  \end{minipage}
  \begin{minipage}{0.19\linewidth}\centering
    \subcaption{}\label{fig:5-1c}
  \end{minipage}
  \begin{minipage}{0.19\linewidth}\centering
    \subcaption{}\label{fig:5-1d}
  \end{minipage}
  \begin{minipage}{0.19\linewidth}\centering
    \subcaption{}\label{fig:5-1e}
  \end{minipage}
  \caption{氢原子的五次瞬间照相}\label{fig:5-1}
\end{figure}
\cref{fig:5-1} 里 $\oplus$ 表示原子核,一个小黑点表示电子在这里出现过一次。
然后继续给氢原子拍上成千上万张照片,并把这些照片一一对比研究,这样,我们就获得一个印象: 电子好象是在氢原子核外作毫无规律的运动,一会儿在这里出现,一会儿在那里出现。
如果我们将这些照片叠印,就会看到如\cref{fig:5-2} 所示的图象。
图象说明,对氢原子的照片叠印张数越多,就越能使人形成一团电子云雾笼罩原子核的印象,而这团 “电子云雾” 呈球形对称,在离核越近处密度越大,离核越远处密度越小。
也就是说,在离核越近处单位体积的空间中电子出现的机会越多,离核越远处单位体积的空间中电子出现的机会越少。
实际上,\cref{fig:5-2d} 就是在通常状况下氢原子的电子云示意图。
\begin{figure}
  \begin{minipage}{0.24\linewidth}\centering
    \subcaption{5 张照片}\label{fig:5-2a}
  \end{minipage}
  \begin{minipage}{0.24\linewidth}\centering
    \subcaption{20 张照片}\label{fig:5-2b}
  \end{minipage}
  \begin{minipage}{0.24\linewidth}\centering
    \subcaption{100 张照片}\label{fig:5-2c}
  \end{minipage}
  \begin{minipage}{0.24\linewidth}\centering
    \subcaption{10000 张照片}\label{fig:5-2d}
  \end{minipage}
  \caption{将若干张氢原子瞬间照相叠印的结果}\label{fig:5-2}
\end{figure}

\subsection{核外电子的运动状态}
\subsubsection{电子层}
我们已经知道,在含有多个电子的原子里,电子的能量并不相同。
能量低的,通常在离核近的区域运动;能量高的,通常在离核远的区域运动。
根据电子的能量差异和通常运动的区域离核的远近不同,可以将核外电子分成不同的电子层。

我们怎么知道含有多个电子的原子里核外电子的能量并不相同呢?
根据对元素电离能数据的分析,可以初步得到这个结论。

什么是电离能?从气态原子(或气态阳离子)中去掉电子,把它变成气态阳离子(或更高价的气态阳离子),需要克服核电荷的引力而消耗能量,这个能量叫做\Concept{电离能},符号为 $I$,单位常用电子伏特\footnote{电子伏特(\unit{eV})是一个电子在真空中通过 \qty{1}{V} 电位差所获得的动能,它是一种描述微观粒子运动的能量单位。$\qty{1}{eV}=\qty{1.6022e-19}{J}$}。

从元素的气态原子去掉一个电子成为 $+1$ 价气态阳离子所需消耗的能量,称为第一电离能($I_1$),从 $+1$ 价气态阳离子再去掉一个电子成为 $+2$ 价气态阳离子所需消耗的能量,叫做第二电离能($I_2$);依次类推。

\cref{tab:5-1} 列出了几种元素电离能的数据。
\begin{table}
  \caption{几种元素的电离能(\unit{eV})}\label{tab:5-1}
  \begin{tblr}{
    colspec={cc*{9}{X[r]}},hline{2}=0.8pt,row{1}={m,c},
    hline{3}={4-4}{1.2pt},vline{4}={2-2}{1.2pt},
    hline{4}={5-5}{1.2pt},vline{5}={3-3}{1.2pt},
    hline{5}={6-6}{1.2pt},vline{6}={4-4}{1.2pt},
    hline{6}={7-7}{1.2pt},vline{7}={5-5}{1.2pt},
    hline{7}={8-8}{1.2pt},vline{8}={6-6}{1.2pt},
    hline{8}={9-9}{1.2pt},vline{9}={7-7}{1.2pt},
    vline{10}={8-8}{1.2pt}
    }
    {核电\\荷数} & {元素\\符号} & $I_1$ & $I_2$ & $I_3$ & $I_4$ & $I_5$ & $I_6$ & $I_7$ & $I_8$ & $I_9$ \\
    3 & \ce{Li} &  5.4 & 75.6 & 122.4 & &&&&&\\ 
    4 & \ce{Be} &  9.3 & 18.2 & 153.9 & 217.7 &&&&&\\ 
    5 & \ce{B}  &  8.3 & 25.1 &  37.9 & 259.3 & 340.1 &&&&\\ 
    6 & \ce{C}  & 11.3 & 24.4 &  47.9 &  64.5 & 392.0 & 489.8 &&&\\ 
    7 & \ce{N}  & 14.5 & 29.6 &  47.4 &  77.5 &  97.9 & 551.9 & 666.8 &&\\ 
    8 & \ce{O}  & 13.6 & 35.1 &  54.9 &  77.4 & 113.9 & 138.1 & 739.1 & 871.1 &\\ 
    9 & \ce{F}  & 17.4 & 35.0 &  62.6 &  87.1 & 114.2 & 157.1 & 185.1 & 953.6 & 1102 \\ 
  \end{tblr}
\end{table}

从表上数据可见,元素的第二电离能大于第一电离能,第三电离能大于第二电离能,依次类推,即 $I_1<I_2<I_3<\cdots$。 
这是容易理解的,因为从 $+1$ 价气态阳离子中去掉一个电子需克服的电性引力比从中性原子去掉一个电子要大,消耗的能量要多。
同理,从 $+2$ 价气态阳离子中去掉一个电子,需克服的电性引力,比从 $+1$ 价气态阳离子中去掉一个电子更大,消耗的能量更多。
因此,一个原子的电离能是依次增大,甚至是成倍增长的,但增大的倍数并不相同。
有的增大得不多,有的增大得很多。我们在\cref{tab:5-1} 上将增大倍数很多的电离能数据前面和下面标上粗线,以示区别。
下面就来分析这些数据。

\ce{Li},原子核外有 3 个电子。$I_3$ 比 $I_2$ 增大不到一倍,但 $I_2$ 比 $I_1$ 却增大了十几倍。这说明什么问题? 
说明这 3 个电子可分为两组,两组能量有差异。
$I_1$ 比 $I_2$、$I_3$ 小得多,说明有一个电子能量较高,通常在离核较远的区域运动,容易被去掉。
另外两个电子能量较低,通常在离核较近的区域运动。

\ce{Be},原子核外有 4 个电子。按照如上的分析,$I_2$ 比 $I_1$,$I_4$ 比 $I_3$ 均增大不到一倍,但 $I_3$ 比 $I_2$ 却增大了好几倍。
因此可以认为有两个电子能量较低,通常在离核较近的区域运动;另外两个电子能量较高,通常在离核较远的区域运动。

分析 \ce{B}、\ce{C}、\ce{N}、\ce{O}、\ce{F} 等元素的电离能的数据,将会发现它们的核外电子都分两组,第一组是两个电子,能量较低,通常在离核较近的区域运动;第二组分别是 3、4、5、6、7 个电子,能量较高,通常在离核较远的区域运动。

如果分析其它元素的电离能数据,也会得出相似的结论。 
可见,在含多个电子的原子中,电子是分层排布的。

\subsubsection{电子亚层和电子云形状}
科学研究发现,在同一电子层中,电子的能量还稍有差别,电子云的形状也不相同。
根据这个差别,又可以把一个电子层分成一个或几个亚层,分别用 $s$、$p$、$d$、$f$ 等符号\footnote{$s$、$p$、$d$、$f$ 是光谱学上的符号}表示。
K 层只包含一个亚层,即 $s$ 亚层;L 层包含两个亚层,即 $s$ 亚层和 $p$ 亚层;M 层包括三个亚层,即 $s$、$p$、$d$ 亚层;N 层包括四个亚层,即 $s$、$p$、$d$、$f$ 亚层。
不同亚层的电子云形状不同。
$s$ 亚层的电子云是以原子核为中心的球形,$p$ 亚层的电子云是 纺锤形,$d$ 亚层、$f$ 亚层的电子云形状比较复杂,这里就不介绍了。

在同一个电子层里,亚层电子的能量是按 $s$、$p$、$d$、$f$ 的次序递增的。
为了清楚地表示某个电子处于核外哪个电子层和亚层(自然同时也表示它的能量高低和电子云的形状),可将电子层的序数 $n$ 标在亚层符号的前面。
如处于 K 层的 $s$ 亚层的电子标为 $1s$;处于 L 层的 $s$ 亚层和 $p$ 亚层的电子标为 $2s$ 和 $2p$ ;处于 M 层的 $d$ 亚层的电子标为 $3d$;处于 N 层的 $f$ 亚层的电子标为 $4f$。\cref{fig:5-3a} 就是氢的 $1s$ 电子云。
\begin{figure}
  \begin{minipage}{0.32\linewidth}\centering
    \subcaption{}\label{fig:5-3a}
  \end{minipage}
  \begin{minipage}{0.32\linewidth}\centering
    \subcaption{}\label{fig:5-3b}
  \end{minipage}
  \begin{minipage}{0.32\linewidth}\centering
    \subcaption{}\label{fig:5-3c}
  \end{minipage}
  \caption{氢原子的 $1s$ 电子云}\label{fig:5-3}
\end{figure}

\cref{fig:5-3b} 虚线表示的球壳称为电子云的界面。
在界面内电子出现的机会最多,界面外电子出现的机会很少。
通常也用电子云界面图来表示电子云。
\cref{fig:5-3c} 是氢原子 \({1s}\) 电子云的界面图,它把表示电子出现机会的小黑点略去了。

\subsubsection{电子云的伸展方向}
电子云不仅有确定的形状,而且有一定的伸展方向。$s$ 电子云是球形对称的,在空间各个方向上伸展的程度相同。
$2p$ 电子云如\cref{fig:5-4} 所示,在空间可以有三种互相垂直的伸展方向。
$d$ 电子云可以有五种伸展方向,$f$ 电子云可以有七种伸展方向。
\begin{figure}
  \begin{minipage}{0.32\linewidth}\centering
    \subcaption{}\label{fig:5-4a}
  \end{minipage}
  \begin{minipage}{0.32\linewidth}\centering
    \subcaption{}\label{fig:5-4b}
  \end{minipage}
  \begin{minipage}{0.32\linewidth}\centering
    \subcaption{}\label{fig:5-4c}
  \end{minipage}
  \caption{$2p$ 电子云的三种伸展方向}\label{fig:5-4}
\end{figure}

如果把在一定的电子层上,具有一定的形状和伸展方向的电子云所占据的空间称为一个轨道,那么 $s$、$p$、$d$、$f$ 四个亚层就分别有 1、3、5、7 个轨道。这样,各电子层可能有的最多轨道数如下:

\begin{tabular}{rlr}
  电子层($n$)& 亚层 & 轨道数 \\
  $n=1$  & $s$                & $1=1^2$ \\
  $n=2$  & $s$、$p$           & $1+3=4=2^2$ \\
  $n=3$  & $s$、$p$、$d$      & $1+3+5=9=3^2$ \\
  $n=4$  & $s$、$p$、$d$、$f$ & $1+3+5+7=16=4^2$ \\
    $n$  &                    & $n^2$ \\
\end{tabular}

\noindent 即每个电子层可能有的最多轨道数应为 $n^2$。

\subsubsection{电子的自旋}
电子不仅在核外空间不停地运动,而且还作自旋运动。
电子自旋有两种状态,相当于顺时针和逆时针两种方向。
平常我们用向上箭头 $\uparrow$ 和向下箭头 $\downarrow$ 来表示不同的自旋状态。

通过以上的叙述我们可以看出,电子在原子核外的运动状态是相当复杂的,必须由它所处的电子层、电子亚层、电子云的空间伸展方向和自旋状态四个方面来决定。前三个方面跟电子在核外空间的位置有关,体现了电子在核外空间的运动状态,确定了电子的轨道。
因此,当我们要说明一个电子的运动状态时,必须同时指明它处于什么轨道和哪一种自旋状态。

\begin{Practice}[习题]
  \begin{question}
    \item 你如何理解电子层、电子亚层和电子云这三个概念?氢原子核外只有一个电子,为什么要用电子云来描述它的运动?
    \item 什么叫电离能?为什么根据元素电离能的变化可以判断核外电子是分层排布的?
    \item 原子核外电子的运动状态,必须从哪几个方面来进行描述?
    \item $1s$、$2p$、$3d$、$4f$ 各表示什么意思?
    \item $d$ 亚层和 $f$ 亚层各有多少轨道?在同一轨道上运动的电子可以有几种不同的运动状态?
    \item $2p_x$、$2p_y$、$2p_z$ 各表示什么意思?
  \end{question}
\end{Practice}

\section{原子核外电子的排布}
上一节我们学习了原子核外电子的运动状态,了解电子是分层排布的,而电子层又可分为几个电子亚层。现在,我们就来进一步讨论原子核外电子的排布规律。

\subsection[泡利不相容原理]{泡利\protect\footnote{泡利(Pauli,1900--1958),奥地利物理学家。}不相容原理}
我们先来讨论锂的核外电子排布。
锂原子有 3 个电子。 这 3 个电子是都在一个轨道上,还是分别在几个轨道上呢?
实验证明,有 2 个在 $1s$ 轨道上,1 个在 $2s$ 轨道上。
在 $1s$ 轨道上的 2 个电子自旋方向是平行的,还是相反的?
实验证明,是相反的。
在 $2s$ 轨道上的那个电子虽然自旋方向跟 $1s$ 轨道上的一个电子相同,但它们分别处于两个不同的轨道。
其它元素核外电子排布有类似的情况。

从以上建立在实验基础上的讨论中我们看到,在原子核外电子的排布中,排在同一轨道上的两个电子,自旋方向就相反; 而自旋方向相同的电子,必然处于不同的轨道上。
我们知道,一个轨道是由电子层、电子亚层和电子云的伸展方向三方面确定的,因此,可以得出一个结论: 在同一个原子里; 没有运动状态四个方面完全相同的电子存在。
这个结论是泡利提出来的,叫做\Concept{泡利不相容原理}。

根据这个原理,我们可以推算出各电子层可以容纳的最多电子数。
我们知道,每个电子层可能有的最多轨道数为 $n^2$,而每个轨道又只能容纳 2 个电子,因此,各电子层可能容纳的电子总数就是 $2n^2$。
现将 1~4 电子层可容纳电子的最大数目列于\cref{tab:5-2} 中。
\begin{table}
  \caption{1~4 电子层可容纳电子的最大数目}\label{tab:5-2}
  \begin{tblr}{colspec={X[6,c]*{10}{X[c]}}}
    电子层($n$)& \SetCell[c=1]{m,c}{K\\(1)}& \SetCell[c=2]{m,c}{L\\(2) }& & \SetCell[c=3]{m,c}{M\\(3)} & & & \SetCell[c=4]{m,c}{N\\(4)} & & & \\ 
    电子亚层     & $s$ & $s$ & $p$  & $s$ & $p$ & $d$ & $s$ & $p$ & $d$ & $f$ \\ 
    亚层中的轨道数     & 1 & 1 & 3  & 1 & 3 & 5 & 1 & 3 & 5 & 7 \\ 
    亚层中的电子数     & 2 & 2 & 6  & 2 & 6 & 10 & 2 & 6 & 10 & 14 \\ 
    每个电子层中可容纳电子的最大数目     & 2 & \SetCell[c=2]{m,c}8 &  & \SetCell[c=3]{m,c}18 &  &  & \SetCell[c=4]{m,c}32 &  &  &  \\ 
  \end{tblr}
\end{table}

\subsection{能量最低原理}
生活常识告诉我们,水总是由高处向低处流,山上的石头可以自动地向山下滚。
这是由于物体处于高势能状态时不如低势能状态稳定。
同理,在核外电子的排布中,通常状况下电子也总是尽先占有能量最低的轨道,只有当这些轨道占满后,电子才依次进入能量较高的轨道。
这个规律叫做\Concept{能量最低原理}。

那么,哪些轨道的能量高,哪些轨道的能量低呢?

我们知道,不同电子层具有不同的能量,而每个电子层中不同亚层的能量也不相同。
为了表示原子中各电子层和亚层电子能量的差异,人们把原子中不同电子层和亚层的电子按能量高低排成顺序,象台阶一样,叫做能级。
例如,$1s$ 能级,$2s$ 能级,$2p$ 能级,等等。
在一个原子中,离核越近、 $n$ 越小的电子层能量越低。
在同一电子层中,各亚层的能量是按 $s$、$p$、$d$、$f$ 的次序增高的。
因此,我们可以认为 $2s$ 能级高于 $1s$ 能级,$2p$ 能级高于 $2s$ 能级,等等。
可是对于那些核外电子数较多的元素来说,情况就比较复杂了。
为什么呢? 因为多电子原子的各个电子之间存在着排斥力,在研究某个外层电子的运动状态时,必须同时考虑到核对它的吸引力及其它电子对它的排斥力。
由于其它电子的存在,往往减弱了原子核对外层电子的吸引力,从而使多电子原子的电子所处的能级产生了交错现象。
\cref{fig:5-5} 是多电子原子电子的近似能级图,图上一个方框代表一个轨道。
\begin{figure}
  \caption{多电子原子电子的近似能级图}\label{fig:5-5}
\end{figure}

从\cref{fig:5-5} 可以看到,从第三电子层起就有能级交错现象,例如,$3d$ 电子的能量似乎应该低于 $4s$,而实际上 $E_{3d}>E_{4s}$。
按照能量最低原理,电子在进入核外电子层时,不是排完了 $3p$ 就排 $3d$,而是先排 $4s$。排完了 $4s$,才排 $3d$。

应用多电子原子电子的近似能级图,并根据能量最低原理,就可以确定电子排入各轨道的次序,如\cref{fig:5-6} 所示。
\begin{figure}
  \caption{电子填入轨道的顺序}\label{fig:5-6}
\end{figure}

\subsection[洪特规则]{洪特\protect\footnote{洪特(Hund,1896--1997),德国物理学家。}规则}
我们运用泡利不相容原理和能量最低原理,再来讨论碳、氮、氧三种元素原子的核外电子的排布情况。

碳元素的核电荷数为 6,即核外有 6 个电子。
根据上述两个原理,核外电子首先在 $1s$ 轨道排入两个自旋方向相反的电子,然后另 2 个自旋方向相反的电子排入 $2s$ 轨道,还剩 2 个电子,应排入 $2p$ 轨道。
$2p$ 轨道有 3 个,它们是以自旋方向相反的方式排入一个 $2p$ 轨道,还是以自旋方向相同的方式排入两个 $2p$ 轨道呢?
人们从科学实验中总结出的叫做\Concept{洪特规则}的一条规律回答了这个问题,这个规则指出,在同一亚层中的各个轨道(如 3 个 $p$ 轨道,或 5 个 $d$ 轨道,或 7 个 $f$ 轨道)上,电子的排布尽可能分占不同的轨道,而且自旋方向相同,这样排布整个原子的能量最低。
因此,碳、氮、氧三元素原子的电子层排布应该如\cref{fig:5-7} 所示。图中 \(\left| \frac{1s}{ \uparrow \downarrow }\right| \frac{2s}{\left( { \uparrow \downarrow }\right) \left| \uparrow \right| \uparrow }\frac{2p}{ \uparrow \uparrow }\) 叫做轨道表示式,一个方框表示一个轨道; 式子 $1s^22s^22p^2$ 叫做电子排布式,式中右上角的数字表示该轨道中电子的数目,如 $1s^2$ 表示在 $1s$ 轨道上有两个电子。
\begin{figure}
  \caption{碳、氮、氧原子的电子层排布}\label{fig:5-7}
\end{figure}

根据上述三个原理和多电子原子电子的近似能级图,我们将核电荷数为 1~36 的元素原子的核外电子的排布情况列入\cref{tab:5-3} 中。

\begin{table}
  \caption{核电荷数为 1~36 的元素的电子层排布}\label{tab:5-3}
  % \begin{tblr}{colspec={},}
  % \end{tblr}
\end{table}

从表上可以看出,核电荷数为 24 的元素 \ce{Cr},核电荷数为 29 的元素 \ce{Cu},它们的电子层结构并没有完全按照前述规律排布,\ce{Cr} 和 \ce{Cu} 在排了 $3p^6$ 后似应排成 $3d^44s^2$ 和 $3d^94s^2$,但实验数据表明应排成 $3d^54s^1$ 和 $3d^{10}4s^1$; 其它元素的电子层排布也有类似的情况。
根据这种情况,人们又归纳出一条规律,就是对于同一电子亚层,当电子排布为全充满、半充满或全空时,是比较稳定的。
即
\begin{description}
  \item[全充满] $p^6$ 或 $d^{10}$ 或 $f^{14}$ 
  \item[半充满] $p^6$ 或 $d^5$ 或 $f^7$ 
  \item[全 空] $p^6$ 或 $d^0$ 或 $f^0$ 
\end{description}
这是洪特规则的一种特例。上述 \ce{Cr}、\ce{Cu} 的电子层排布,就是属于 $d$ 轨道半充满、全充满时比较稳定的例子。

这里需要指出,核外电子的排布情况是通过实验测定的。
上面讲的泡利不相容原理、能量最低原理和洪特规则三条原理,是从大量事实中概括出来的,它们能帮助我们了解元素原子核外电子排布的规律,但不能用它们来解释有关电子排布的所有问题。
因此,这些原理只具有相对近似的意义。

\begin{Practice}[习题]
  \begin{question}
    \item 解释下列符号各代表什么意义:
    \item 用 $E$ 代表能量,把下列轨道按能量由低到高的顺序排列起来:
    \item N 电子层有哪几种轨道?轨道数共是多少?分别写出这些轨道的符号。
    \item 填空
    \item 解释下列事实:
    \begin{tasks}
      \task 核电荷数为 19 的元素 \ce{K} 的电子层排布为什么是 $1s^22s^22p^63s^23p^64s^1$,而不是 $1s^22s^22p^63s^23p^63d^1$?
      \task 核电荷数为 24 的元素 \ce{Cr} 的电子层排布为什么是 $1s^22s^22p^63s^23p^63d^54s^1$,而不是 $1s^22s^22p^63s^23p^63d^44s^2$?
    \end{tasks}
    \item 某元素 $2p$ 亚层上有 3 个电子; 这 3 个电子应该是按方式排布,还是按方式排布? 为什么?
    \item 某种元素的电子排布式是 $1s^22s^22p^63s^23p^63d^{10}4s^24p^6$ 说明它的原子Σ外有多少个电子层,各电子层有多少个电子,该元素的原子总共有多少个电子,核电荷数是几。
    \item 用电子排布式表示铝(核电荷数为 13)、氯(核电荷数为 17)、铁(核电荷数为 26)、铜(核电荷数为 29)和氪(核电荷数为 36)的电子层排布。
  \end{question}
\end{Practice}

\section{元素周期律}
从学初中化学到现在,我们已经学习了惰性气体、卤族、 氧族、碱金属几个元素族的知识,了解到一个自然族内的元素性质相似,而族跟族之间元素的性质不同。
这说明元素之间的关系存在着一定的规律。

为了认识元素间的这种规律性,我们将核电荷数为 1~18 的元素的核外电子排布、原子半径、第一电离能和主要化合价列成表(\cref{tab:5-4})来加以讨论。
为了方便,人们按核电荷数由小到大的顺序给元素编号,这种序号,叫做该元素的原子序数。
显然,原子序数在数值上与这种原子的核电荷数相等。
\cref{tab:5-4} 就是按原子序数的顺序编排的。
\begin{sidewaystable}
  \centering\small
  \caption{元素性质随着核外电子周期性排布而呈周期性的变化}\label{tab:5-4}
  \begin{tblr}{colspec={X[c]*{18}{c}},colsep=2pt,rowsep=10pt,row{4,6}={font=\scriptsize},cell{4,6}{1}={}{font=\small}}
    原子序数 & 1  & 2  & 3 & 4 & 5 & 6 & 7 & 8 & 9 & 10 & 11 & 12 & 13 & 14 & 15 & 16 & 17 & 18 \\
    元素名称 & 氢 & 氦 &锂 &铍 &硼 &碳 &氮 &氧 &氟 & 氖 & 钠 & 镁 & 铝 & 硅 & 磷 & 硫 & 氯 & 氩\\
    元素符号 & \ce{H}  & \ce{He}  & \ce{Li} & \ce{Be} & \ce{B} & \ce{C} & \ce{N} & \ce{O} & \ce{F} & \ce{Ne} & \ce{Na} & \ce{Mg} & \ce{Al} & \ce{Si} & \ce{P} & \ce{S} & \ce{Cl} & \ce{Ar} \\
    {最外层电\\子的排布} & $1s^1$  & $1s^2$  & $2s^1$ & $2s^2$ & $2s^22p^1$ & $2s^22p^2$ & $2s^22p^3$ & $2s^22p^4$ & $2s^22p^5$ & $2s^22p^6$ & $3s^1$ & $3s^2$ & $3s^23p^1$ & $3s^23p^2$ & $3s^23p^3$ & $3s^23p^4$ & $3s^23p^5$ & $3s^23p^6$ \\
    {原子半径\\(\qty{e-10}{m})} & 0.37  & 1.22  & 1.52 & 0.89 & 0.82 & 0.77 & 0.75 & 0.74 & 0.71 & 1.60 & 1.86 & 1.60 & 1.43 & 1.17 & 1.10 & 1.02 & 0.99 & 1.01 \\
    {第一\\电离能\\(\unit{eV})} & 13.595  & 24.481  & 5.39 & 9.32 & 8.296 & 11.256 & 14.53 & 13.614 & 17.418 & 21.559 & 5.138 & 7.644 & 5.984 & 8.149 & 10.484 & 10.357 & 13.01 & 15.755 \\
    化合价 & $+1$  & $0$  & $+1$ & $+2$ & $+3$ & {$+4$ \\ $-4$} & {$+5$ \\ $-3$} & $-2$ & $-1$ & $0$ & $+1$ & $+2$ & $+3$ & {$+4$ \\ $-4$} & {$+5$ \\ $-3$} & {$+6$ \\ $-2$} & {$+7$ \\ $-1$} & $0$ \\
  \end{tblr}
\end{sidewaystable}

\subsection{核外电子排布的周期性}
我们来看表 5-4 中原子序数 1~18 的元素原子电子层排布的情况。原子序数从 1~2 的元素,即从氢到氦,有一个电子层,电子层排布由 $1s^1$ 到 $1s^2$,电子由 1 个增到 2 个,达到稳定结构。
原子序数从 3~10 的元素,即从锂到氖,有两个电子层,最外电子层排布由 $2s^1$ 到 $2s^22p^6$,最外层电子从 1 个递增到 8 个,达到稳定结构。
原子序数从 11~18 的元素,即从钠到氩,有三个电子层,最外电子层排布从 $3s^1$ 到 $3s^23p^6$,最外层电子也从 1 个递增到 8 个,达到稳定结构。
如果我们对 18 号以后的元素继续研究下去,同样可以发现,每隔一定数目的元素,也会重复出现原子最外层电子数从 1 个递增到 8 个的情况。也就是说,随着原子序数的递增,元素原子的最外层电子排布呈周期性的变化。

\subsection{原子半径的周期性变化}
从\cref{tab:5-4} 可以看出,由碱金属元素锂到卤素氟,随着原子序数的递增,原子半径由 \qty{1.52e-10}{m} 递减到 \qty{0.71e-10}{m},即原子半径由大逐渐变小。
再由碱金属元素钠到卤素氯,随着原子序数的递增,原子半径由 \qty{1.86e-10}{m} 递减到 \qty{0.99e-10}{m},原子半径也是由大逐渐变小。
如果把所有的元素按原子序数递增的顺序排列起来,将会发现随着原子序数的递增,元素的原子半径发生周期性的变化\footnote{惰性气体元素原子半径跟邻近的非金属元素相比显得特别大,这是由于它们测定的根据跟其它元素不同。},\cref{fig:5-8} 表示碱金属等 7 个族和惰性气体元素的原子半径的周期性变化。
\begin{figure}
  \caption{元素原子半径的周期性变化}\label{fig:5-8}
\end{figure}


\subsection{第一电离能的周期性变化}
元素电离能的数值反映了元素原子失去电子的难易程度,元素的电离能越小,它的原子越容易失去电子。
因此,元素的第一电离能就是该元素的金属活动性的一种衡量尺度。

研究原子序数 1~18 的元素的第一电离能,我们将会发现,由氢到氦,由锂到氖,由钠到氩,第一电离能的变化趋势都是由小到大。
如果继续研究 18 号以后的元素,也会得出相同的结论。将 1~18 号元素的第一电离能数据绘成曲线图(\cref{fig:5-9}),从图上可以形象地看到,元素的第一电离能随着原子序数的递增,呈现周期性的变化。
\begin{figure}
  \caption{元素第一电离能的周期性变化}\label{fig:5-9}
\end{figure}
\begin{Theorem}{讨论}
  如何解释氨、铍、氖、镁几种元素的第一电离能比它们的相邻元素为高?
\end{Theorem}

\subsection{元素主要化合价的周期性变化}
从\cref{fig:5-4} 可以看到,第 11 号元素到第 18 号元素,在极大程度上重复着第 3 号元素到第 10 号元素所表现的化合价的变化——正价从 $+1$(\ce{Na})逐渐递变到 $+7$(\ce{Cl}),从中部的元素开始有负价,负价是从 $-4$(\ce{Si})递变到 $-1$(\ce{Cl})。
如果研究第 18 号元素以后的元素的化合价,同样可以看到与前面 18 种元素相似的变化。也就是说,元素的化合价随着原子序数的递增而起着周期性的变化。

原子半径、第一电离能和元素主要化合价,都是元素的重要性质。
通过上述的研究,我们可以引出这样一条规律,就是\emph{元素的性质随着元素原子序数的递增而呈周期性的变化}。 
这个规律叫做\Concept{元素周期律}。

元素性质的周期性变化是元素原子的核外电子排布的周期性变化的必然结果。

\begin{Practice}[习题]
  \begin{question}
    \item 随着原子序数的递增,原子半径有什么变化?
    \item 随着原子序数的递增,元素的第一 电离能和化合价各有什么变化?
    \item 用原子结构的观点说明为什么元素性质随原子序数的递增呈周期性的变化?
  \end{question}
\end{Practice}
\section{元素周期表}
根据元素周期律,把现在已知的 107 种元素\footnote{截至 2019 年,已知的元素种类已达 118 种。}中电子层数目相同的各种元素,按原子序数递增的顺序从左到右排成横行,再把不同横行中最外电子层的电子数相同的元素按电子层数递增的顺序由上而下排成纵行\footnote{严格说来,是把外围电子相似的元素按电子层数递增的顺序由上到下排成纵行。}。
这样得到一个表,叫做元素周期表(见附录: 元素周期表)。
元素周期表是元素周期律的具体表现形式,它反映了元素之间相互联系的规律。
下面我们就来学习元素周期表的有关知识。

\subsection{元素周期表的结构}
\subsubsection{周期}
元素周期表有 7 个横行,也就是 7 个周期。
具有相同的电子层数而又按照原子序数递增的顺序排列的一系列元素,称为一个周期。
周期的序数就是该周期元素原子具有的电子层数。

各周期里元素的数目不一定相同,第一周期只有 2 种元素; 第二、三周期各有 8 种元素; 第四、五周期各有 18 种元素; 第六周期有 32 种元素。
我们把含有元素较少的第一、二、 三周期叫短周期,把含有元素较多的四、五、六周期叫长周期。
第七周期到现在为止只发现了 21 种元素,还没有填满,叫不完全周期。

除第一周期外,同一周期中,从左到右,各元素原子最外电子层的电子数都是从 1 个逐步增加到 8 个。
除第一周期从气态元素氢开始,第七周期尚未填满外,每一周期的元素都是从活泼的金属元素——碱金属开始,逐渐过渡到活泼的非金属元素——卤素,最后以惰性气体结束。

第六周期中 57 号元素镧 \ce{La} 到 71 号元素镥 \ce{Lu},共 15 种元素,它们的电子层结构和性质非常相似,总称镧系元素。
为了使表的结构紧凑,将镧系元素放在周期表的同一格里,并按原子序数递增的顺序,把它们另列在表的下方,实际上还是各占一格。

第七周期 89 号元素锕 \ce{Ac} 至 103 号元素铹 \ce{Lr},共 15 种元素,它们彼此的电子层结构和性质也十分相似,总称锕系元素,同样把它们放在周期表的同一格里,并按原子序数递增的顺序另列在表下方镧系元素的下面。
锕系元素中铀后面的元素多数是人工进行核反应制得的元素,叫做超铀元素。

\subsubsection{族}
周期表有 18 个纵行。除第 8、9、10 三个纵行叫做第 \MyRoman{8} 族元素外,其余 15 个纵行,每个纵行标作一族。
族可分主族和副族。
由短周期元素和长周期元素共同构成的族,叫做主族;
完全由长周期元素构成的族,叫做副族。
主族元素在族的序数(习惯用罗马数字表示)后面标一 A 字,如 \MyRoman{1}A、 \MyRoman{2}A……,副族元素标一 B 字,如 \MyRoman{1}B、\MyRoman{2}B……。
惰性气体元素化学性质非常不活泼,在通常状况下难以发生化学反应,把它们的化合价看作为 0 ,因而叫做 0 族。
因此. 在整个周期表里,有 7 个主族,7 个副族,1 个第 \MyRoman{8} 族,1 个 0 族,共 16 个族。

\begin{Reading}{元素周期表根据原子的电子层结构分区}
根据原子的电子层结构的特征,元素周期表可划分为四个区(\cref{fig:5-10})。
\begin{figurehere}
  \begin{minipage}{\linewidth}\centering
    \caption{元素周期表根据原子的电子层结构分区}\label{fig:5-10}
  \end{minipage}
\end{figurehere}
\begin{enumerate}
  \item $s$ 区包括 \MyRoman{1}A 和 \MyRoman{2}A 两个主族,最外层只有 1~2 个 $s$ 电子。
  \item $p$ 区包括 \MyRoman{3}A~\MyRoman{7}A 五个主族和 0 族,最外层除了 2 个 $s$ 电子之外,有 1~6 个 $p$ 电子(\ce{He} 例外,无 $p$ 电子)。
  \item $d$ 区包括 \MyRoman{1}B~\MyRoman{7}B 七个副族和第 \MyRoman{8} 族,均属过渡元素。
  最外层有 2 个 $s$ 电子(个别为 1 个,\ce{Pd} 例外,无 $5s$ 电子),次外层有 1~10 个 $d$ 电子。 $d$ 区元素原子电子层的这种结构特征可用通式 $(n-1)d^xns^2$ 表示,$n$ 是周期数,$x$ 是 1~10 的正整数。
  这种表示原子电子层结构特征的式子,又叫原子的特征电子构型(也称外围电子)。
  $s$ 区的特征电子构型是 $ns^x$,$x$ 是 1~2 的正整数;$p$ 区的特征电子构型是 $ns^2np^x$,$x$ 是 1~6 的正整数。
  \item $f$ 区包括镧系和钠系,最外层有 2 个 8 电子,次外层有 2 个 8 电子和 6 个 $p$ 电子(个别有 $d$ 电子),例数第三层有 1~14 个 $f$ 电子。特征电子构型一般是 $(n-2)f^xns^2$,$x$ 是 1~14 的正整数。$f$ 区元素也是过渡元素。
\end{enumerate}

$s$ 区元素都是活泼的金属(氢除外),它们起化学反应时总是失去最外层的 $s$ 电子而成为 $+1$ 价或 $+2$ 价的阳离子。

$p$ 区元素除惰性气体外,有金属,也有非金属。
它们在起化学反应时只有最外层的 $s$ 亚层或 $p$ 亚层的电子发生得失或偏移,不牵涉内层电子。
$ns^2np^8$ 是惰性气体的特征电子构型(\ce{He} 的特征电子构型为 $1s^2$ ),是一种稳定结构,在通常状况下难以发生电子得失或偏移。

$d$ 区元素都是金属,它们在发生化学反应时,不仅有最外层的 $s$ 电子,而且可以有部分或全部次外层的 $d$ 电子失去或偏移。

$f$ 区元素也都是金属。
它们在发生化学反应时,不仅有最外层的 $s$ 电子,次外层的 $d$ 电子,而且可以有倒数第三层的部分或全部 $f$ 电子失去或偏移。
\end{Reading}

\subsection{元素的性质跟原子结构的关系}
\subsubsection{原子结构跟元素的金属性和非金属性的关系}
在同一周期中,各元素的原子核外电子层数虽然相同,但从左到右,核电荷依次增多,原子半径逐渐减小,电离能趋于增大,失电子能力逐渐减弱,得电子能力逐渐增强,因此,金属性逐渐减弱,非金属性逐渐增强。
从同周期元素化学性质变化情况的研究可以证实这个结论是正确的。

一般说来,我们可以从元素的单质跟水或酸反应置换出氢的难易,元素氧化物的水化物(氧化物间接或直接跟水生成的化合物)——氢氧化物的碱性强弱,来判断元素金属性的强弱; 可以从元素氧化物的水化物的酸性强弱,或从跟氢气生成气态氢化物的难易,来判断元素非金属性的强弱。
下面以第三周期元素为例,来研究同周期元素金属性和非金属性的递变。

我们知道,第 11 号元素钠的单质能跟冷水剧烈反应,放出氢气,生成的氢氧化钠是一种强碱。

第 12 号元素镁,它的单质跟水起反应的情况怎样呢?
\begin{Experiment}
  取两段镁带,用砂纸擦去氧化膜,放于试管中,加 \qty{3}{mL} 水,往水中滴 2 滴无色酚酞试液,观察现象。然后加热试管至水沸腾,观察现象。
\end{Experiment}

实验表明,镁不易跟冷水作用,但加热时能跟沸水起反应,产生大量气泡,反应后的溶液使无色酚酞试液变红。这个反应的化学方程式如下:
\[ \ce{ Mg + 2H2O \xlongequal{\quad} Mg(OH)2 +H2 ^} \]

镁能从水中置换出氢,说明它是一种活泼金属。
但它只容易跟沸水起反应,所生成的氢氧化镁的碱性也比氢氧化钠弱,说明它的金属活动性不如钠强。

现在我们来研究第 13 号元素铝的一些性质。
\begin{Experiment}
取一小片铝和一小段镁带,用砂纸擦去氧化膜,分别放入两个试管中,再各加入 \qty{2}{mL} $1M$ 盐酸,观察现象。
\end{Experiment}

实验表明,镁、铝都能跟盐酸起反应,置换出氢气,反应的化学方程式如下:
\begin{gather*}
  \ce{ Mg + 2HCl \xlongequal{\quad} MgCl2 + H2 ^}\\
  \ce{ 2Al + 6HCl \xlongequal{\quad} 2AlCl3 + 3H2 ^}
\end{gather*}
但铝跟酸的反应不如镁跟酸的反应剧烈。也就是说,铝的金属活动性不如镁强。

我们在初中已经知道,铝的氧化物 \ce{Al2O3} 既能跟酸反应,又能跟碱反应,是一种两性氧化物。那么,它的对应水化物氢氧化铝的酸碱性又怎样呢?
\begin{Experiment}
取少量 $1M$ 的三氯化铝溶液注入试管中,加入 $3M$ 的氢氧化钠溶液到产生大量的氢氧化铝白色絮状沉淀为止。
将氢氧化铝沉淀分盛在两个试管中,然后在两个试管中分别加入 $3M$ 的硫酸和 $6M$ 的氢氧化钠溶液,观察现象。
\end{Experiment}
我们看到,两个试管中的白色沉淀都消失了。这说明,氢氧化铝既能跟酸反应,又能跟碱反应。

上述反应的化学方程式如下:
\begin{gather*}
  \ce{ AlCl3 + 3NaOH \xlongequal{\quad} 3NaCl + Al(OH)3 v  } \\ 
  \ce{ 2Al(OH)3 + 3H2SO4 \xlongequal{\quad} Al2(SO4)3 + 6H2O}\\ 
  \ce{ H3AlO3 + NaOH \xlongequal{\quad} NaAlO2 + 2H2O}
\end{gather*}
当 \ce{Al(OH)3} 跟碱起反应时,它的分子式还可以写成 \ce{H3AlO3} 的形式。

象氢氧化铝这样既能跟酸起反应,又能跟碱起反应的氢氧化物,叫做两性氢氧化物。氢氧化铝既然呈两性,就说明铝已表现出一定的非金属性。

第 14 号元素硅是非金属。
硅的氧化物 \ce{SiO2} 是酸性氧化物,它的对应水化物是硅酸(\ce{H4SiO4})。
硅酸是一种很弱的酸。 
硅只有在高温下才能跟氢气起反应生成气态氢化物 \ce{SiH4}。

第 15 号元素磷是非金属,它的最高价氧化物是 \ce{P2O5},\ce{P2O5} 的对应水化物是磷酸(\ce{H3PO4}),属于中强酸。
磷的蒸气和氢气能起反应生成气态氢化物 \ce{PH3},但相当困难。

第 16 号元素硫是比较活泼的非金属,它的最高价氧化物是 \ce{SO3},\ce{SO3} 的对应水化物是硫酸。
硫酸是一种强酸。
在加热时硫能跟氢气化合生成气态氢化物硫化氢。

第 17 号元素氯是很活泼的非金属,它的最高价氧化物是 \ce{Cl2O7},\ce{Cl2O7} 的对应水化物是高氯酸(\ce{HClO4}),它是已知酸中最强的一种酸。
氯气跟氢气在光照或点燃时就能发生爆炸而化合,生成气态氢化物氯化氢。

第 18 号元素氩是一种惰性气体。

综上所述,可以得出如下结论:
\begin{center}
  \begin{tikzpicture}
    \foreach \x[count=\i] in {Na,Mg,Al,Si,P,S,Cl}
      { \node at (\i,0.6) [below] {\ce{\x}}; }
    \draw[->](0,0)--(8,0)node[midway,below]{金属性逐渐减弱,非金属性逐渐增强};
  \end{tikzpicture}
\end{center}

对其它周期元素的化学性质进行逐一的探讨,也会得到类似的结论。

在同一主族的元素中,由于从上到下电子层数增多,原子半径增大,电离能一般趋于减小,失电子能力逐渐增强,得电子能力逐渐减弱,所以元素的金属性逐渐增强,非金属性逐渐减弱。
这可以从碱金属元素和卤素的化学性质的递变中得到证明。
我们知道,碱金属元素的金属性是从上到下逐渐增强,卤素的非金属性是从上到下逐渐减弱的。

副族元素化学性质的变化规律比较复杂,这里就不讨论了。

我们还可以在周期表上对金属元素和非金属元素进行分区(\cref{tab:5-5})。
如果沿着周期表中硼、硅、砷、碲、破跟铝、锗、锑、 钋之间划一条虚线,虚线的左面是金属元素,右面是非金属元素。
左下方是金属性最强的元素,右上方是非金属性最强的元素。
由于金属性、非金属性没有严格的界线,位于分界线附近的元素,既表现某些金属性质,又表现某些非金属性质。

\begin{table}
  \caption{主族元素金属性和非金属性的递变}\label{tab:5-5}
  \begin{tblr}{
      colspec={c*{8}{X[c]}},
      hlines={0pt},vlines={0pt},rowsep=2pt,
      hline{1,Z}={1.5pt},vline{1,Z}={1.5pt},
      hline{2}={0.8pt},vline{2}={0.8pt},
      vline{4}={3-3}{dashed,1.2pt},hline{4}={4-4}{dashed,1.2pt},
      vline{5}={4-4}{dashed,1.2pt},hline{5}={5-5}{dashed,1.2pt},
      vline{6}={5-5}{dashed,1.2pt},hline{6}={6-6}{dashed,1.2pt},
      vline{7}={6-6}{dashed,1.2pt},hline{7}={7-7}{dashed,1.2pt},
      vline{8}={7-7}{dashed,1.2pt},hline{8}={8-8}{dashed,1.2pt}
    }
    \diagbox{周期}{族}& \MyRoman{1}A & \MyRoman{2}A & \MyRoman{3}A & \MyRoman{4}A & \MyRoman{5}A & \MyRoman{6}A & \MyRoman{7}A & \\
    1 &\SetCell[r=7]{m,r}{\tikz \draw[->](0,4)--(0,0)node[midway,left]{\parbox{1em}{金\\属\\性\\逐\\渐\\增\\强}};} 
    & \SetCell[c=6]{h,c}{\tikz \draw[-stealth](0,0)--(9,0) node[midway,above]{非金属性逐渐增强};} &&&&&
    &\SetCell[r=7]{m,l}{\tikz \draw[<-](0,4)--(0,0)node[midway,right]{\parbox{1em}{非\\金\\属\\性\\逐\\渐\\增\\强}};}\\
    2 & & &\ce{B}  & & & & &  \\
    3 & & &\ce{Al} & \ce{Si} &  & & &  \\
    4 & & &        & \ce{Ge} & \ce{As} & & & \\
    5 & & &        &         & \ce{Sb} & \ce{Te} & & \\
    6 & & &        &         &         & \ce{Po} & \ce{At} & \\
    7 & & \SetCell[c=6]{f,c}{\tikz \draw[stealth-](0,0)--(9,0) node[midway,below]{金属性逐渐增强};}&&&&&&\\
  \end{tblr}
\end{table}

\subsubsection{原子结构跟化合价的关系}
元素的化合价跟原子的电子层结构有密切关系,特别是跟最外层电子的数目有关,因此,元素原子的最外层电子,称为价电子。
有些元素的化合价跟它们原子的次外层或倒数第三层的部分电子有关,这部分电子也叫价电子。

在周期表中,主族元素的最高正化合价等于它所在族的序数,因为它们的最外层电子数,即价电子数,跟族的序数相当。
非金属元素的最高正化合价和它的负化合价绝对值的和等于 8。
因为非金属元素的最高正化合价,等于原子所失去或偏移的最外层上的电子数; 而它的负化合价,则等于原子最外层达到 8 个电子稳定结构所需得到的电子数。

副族和第 \MyRoman{8} 族元素的化合价比较复杂,它们原子次外层 $d$ 亚层或倒数第三层 $f$ 亚层上的电子不很稳定,在适当的条件下,和最外层电子一样,也可失去。
它们失去电子的最大数目一般说来跟它们的族的序数相当。

从以上的学习中我们可以知道,元素的性质是由原子结构决定的; 元素在周期表中的位置反映了那个元素的原子结构和一定的性质。
所以,元素性质、原子结构和该元素在周期表中的位置三者有着密切的关系。
我们可以根据元素在周期表中的位置,推论它的原子结构和一定的性质; 反过来,根据元素的原子结构,也可以推论它在周期表中的位置。

\begin{example}
  已知某元素在第四周期 \MyRoman{6}A 族,试写出它的电子排布式,指出它是金属元素还是非金属元素,最高正化合价是多少,最高价氧化物的水化物是酸还是碱。
\end{example}
\begin{solution}
设该元素为 $X$。已知 $X$ 在第四周期,因此,它的原子核外有 4 个电子层。

又知 $X$ 属 \MyRoman{6}A 族,即它的原子的最外层电子数是 6 个,因此它的电子排布式是 $1s^22s^22p^63s^23p^63d^{10}4s^24p^4$。

根据电子排布式判断,它在化学反应中易得到 2 个电子,形成 8 电子稳定结构,这时表现为 $-2$ 价,因此它是非金属元素。它的最高正化合价为 $+6$,最高价氧化物 \ce{$X$O_{$i$}} 的水化物为 \ce{H2$X$O4},是一种酸。
\end{solution}

\begin{example}
已知某元素原子序数为 32,试指出它属于哪一周期,哪一族,是什么元素。
\end{example}
\begin{solution}
  该元素原子序数为 32,即核外有 32 个电子。已知前 4 个周期共有 $2+8+8+18=36$ 个元素,第 36 号元素是惰性气体 \ce{Kr},它的最外层电子是 $4s^24p^6$,而该元素比  \ce{Kr} 少 $36-32=4$ 个电子,它的电子排布式为:$1s^22s^22p^63s^23p^63d^{10}4s^24p^2$。
  
根据电子排布式判断,它属于第四周期,\MyRoman{4}A 族。查周期表,知道该元素是锗 \ce{Ge}。
\end{solution}

注意,解答这类问题时,必须先经过分析推理,找出所求元素在周期表中的位置,然后查周期表得出它的名称。
决不能根据题设的原子序数等数据,直接查周期表得出答案。

\begin{Practice}[习题]
  \begin{question}
    \item 用原子结构的知识,说明元素周期表里的周期和族是按什么划分的?什么叫主族?什么叫副族?
    \item 对于同周期和同主族元素来说,元素的金属性和非金属性是怎样逆变的?在元素周期表上金属元素和非金属元素是怎样分区的?
    \item 有某元素 $A$,它的最高氧化物的分子式是 \ce{$A$O3},气态氢化物里含氢 2.489\%,这是什么元素?
    \item 某元素 $B$ 的最高正化合价和负化合价的绝对值相等,该元素在气态氢化物中占 87.5\%,问该元素的原子量是多少,它是什么元素。
    \item 根据元素在周期表中的位置,判断下列各组化合物的水溶液,哪个酸性较强? 哪个碱性较强?
    \begin{tasks}(2)
      \task \ce{H2CO3} 和 \ce{H3BO3}(硼酸)
      \task \ce{H3PO4} 和 \ce{HNO3}
      \task \ce{Ca(OH)2} 和 \ce{Mg(OH)2}
      \task \ce{Al(OH)3} 和 \ce{Mg(OH)2}
    \end{tasks}
    \item 已知三种元素的原子序数是 11、33 和 35,不看元素周期表,确定它们各处在哪一周期,哪一族,并说明你是如何推断的?
    \item 填空:
    
    % \begin{tblr}{}
    % \end{tblr}
    \item 甲元素原子的核电荷数为 17,乙元素的正二价离子跟氢原子(原子序数为 18)的电子层结构相同。试回答下列问题:
    \begin{tasks}
      \task 甲元素在周期表里位于第\_\_\_周期,第\_\_\_主族,电子排布式是\_\_\_,元素符号是\_\_\_,它的最高价氧化物对应的水化物分子式是\_\_\_,属于无机物的\_\_\_类。
      \task 乙元素在周期表里位于第\_\_\_周期,第\_\_\_主族,电子排布式是\_\_\_,元素符号是\_\_\_,它的最高价氧化物对应的水化物分子式是\_\_\_,属于无机物的\_\_\_ 类。
      \task 比较碘跟甲元素的非金属性哪个强,乙元素跟钾的金属性哪个强。
    \end{tasks}
  \end{question}
\end{Practice}

\section{元素周期律的发现和意义}
从十八世纪中叶到十九世纪中叶这一百年间,随着生产和科学技术的发展,新的元素不断地被发现。
到 1869 年,人们已经知道了 63 种元素。
对于这些元素的物理、化学性质的研究,也已积累了不少的资料。
但是它们还是杂乱无章、无甚头绪的材料,不便于进一步研究和使用。
因此,人们产生了整理和概括这些感性材料,将元素进行分类,寻找它们内在联系的迫切要求,元素周期律就是在这个时代背景下,经过许多人的努力,最后由俄国化学家门捷列夫发现的。
\begin{Reading}[]{补充阅读}
1829 年,德国人德贝莱纳(D\"obereiner,1780--1849)根据元素性质的相似性提出了“三素组”学说。当时它归纳出五个“三素组”,即

\[ \ce{Li},\ce{Na},\ce{K}\quad\ce{Ca},\ce{Sr},\ce{Ba}\quad\ce{P},\ce{As}, \ce{Sb}\quad\ce{S},\ce{Se},\ce{Te}\quad \ce{Cl},\ce{Br},\ce{I}\]

当时已经知道了 54 种元素,而他却只能将 15 种元素归纳入三素组,不能揭示其它大部分元素间的关系,因此,三素组学说没能引起人们的重视。

此后,又有许多人对元素分类作过研究,比较突出的有迈尔的《六元素表》和纽兰兹的《八音律表》。

1864 年,德国人迈尔(Meyer,1830--1895)发表了《六元素表》,在表中对于性质相似的元素六个、六个地进行了分族,但他已归纳成族的元素尚不及当时已知元素的一半。

1865 年,英国人纽兰兹(Newlands,1837--1898)把当时已知的元素按原子量的大小顺序排列,发现从任意一个元素算起,每到第八个元素就和第一个元素的性质相近,犹如八度音阶一样。
他把这个规律叫做“八音律”。
可是他按八音律排的元素表很多地方却是混乱的。
原因是他没充分估计到当时的原子量测定值可能有错误,而是机械地按原子量大小往下排;同时他也没考虑到还有未被发现的元素,没有留下空位。
显然他这样作不能把元素内在联系的规律揭示出来。

对于这项将元素进行科学分类,寻找它们内在联系规律的重要工作,1869 年俄国化学家门捷列夫({\rcmu Менделеев},1834--1907)获得了成功。
他在批判继承前人工作的基础上,对大量实验事实进行了订正、分析和概括,总结出一条规律:元素(以及由它所形成的单质和化合物)的性质随着原子量的递增而呈周期性的变化。
这就是元素周期律。
他还根据元素周期律编制了第一个元素周期表,把已经发现的 63 种元素全部列入表里,从而初步完成了使元素系统化的任务。
他还在表中留下空位,预言了类似硼、铝、硅的未知元素(门捷列夫叫它类硼、类铝和类硅,即以后发现的钪、镓、锗)的性质,并指出当时测定的某些元素原子量的数值有错误,而他在周期表中也没有机械地完全按照原子量数值的顺序排列。
若干年后,他的预言都得到了证实。
门捷列夫工作的成功,引起了科学界的震动。
人们为了纪念他的功绩,就把元素周期律和周期表称为门捷列夫元素周期律和门捷列夫元素周期表。
但是由于时代的局限,门捷列夫仍然未能认识到造成元素性质周期性变化的根本原因。

二十世纪以来,随着科学技术的发展,人们对于原子的结构有了更深刻的认识。
人们发现,引起元素性质周期性变化的本质原因不是原子量的递增,而是核电荷数(原子序数)的递增,也就是核外电子排布的周期性变化。
这样才把元素周期律修正为现在的形式,同时对于元素周期表也作了许多改进。
\end{Reading}

元素周期律的发现,对于化学科学的发展,有很大的影响。

元素周期表是学习和研究化学的一种重要工具。
元素周期表是元素周期律的具体表现,它反映了元素之间的内在联系,是对元素的一种很好的自然分类。
我们可以利用元素的性质、它在周期表中的位置和它的原子结构三者之间的密切关系,来指导我们对化学的学习和研究。

过去,门捷列夫曾用它预言未知元素并得到了证实;此后,人们在周期律、周期表的指导下,对元素的性质进行系统地研究,对物质结构理论的发展起了一定的推动作用。
不仅如此,元素周期律和周期表对于新元素的合成、预测它的原子结构和性质提供了线索。

元素周期律对于工农业生产也有一定的指导作用。
由于在周期表中位置靠近的元素性质相近,这就启发了人们在周期表中一定的区域内寻找新的物质。
例如通常用来制造农药的元素,如氟、氯、硫、磷、砷等在周期表里占有一定区域。
对这个区域里的元素进行充分的研究,有助于制造出新品种的农药。
又例如要找半导体材料,可以在周期表里金属和非金属的接界处去找,如硅、锗、硒等就是。
我们还可以在过渡元素中去寻找催化剂和耐高温、耐腐蚀的合金材料等。

元素周期律的重要意义还在于它从自然科学上有力地论证了事物变化的量变引起质变的规律性。
\section*{内容提要}
\setcounter{subsection}{0}
\subsection{原子结构}
\begin{enumerate}
  \item 组成原子的粒子间的关系如下:
  \[ \text{原子}\; \ce{^$A$_$Z$ $X$} 
     \left\{ \begin{array}{l}
       \text{原子核} \left\{
        \begin{array}{ll}
          \text{质子} & Z \text{个}\\
          \text{中子} & (A-Z) \text{个}\\
        \end{array}
       \right.\\
       \text{核外电子}\quad Z \text{个}    
    \end{array}
     \right.
  \]
  \item 具有相同质子数和不同中子数的同一元素的原子互称同位素。
  \item 电子在核外空间作高速的运动,好象带负电荷的云雾笼罩在原子核的周围,我们形象地称它为“电子云”。
  \item 从气态原子(或气态阳离子)中去掉电子,把它变成气. 态阳离子(或更高价的气态阳离子),所需消耗的能量叫做电离能。根据元素电离能的变化,可以判断电子是分层排布的。
  \item 一个电子的运动状态由它所处的电子层、电子亚层、 电子云的空间伸展方向和自旋状态四个方面来决定。
  \begin{enumerate}
    \item 电子层\quad 根据电子的能量差别和通常运动的区域离核的远近不同,可以将核外电子分成不同的电子层。
    \item 电子亚层\quad 在同一电子层中,根据电子能量的差别和电子云形状的不同,可以分为 $s$、$p$、$d$、$f$ 等几个亚层。
    \item 电子云的伸展方向\quad $s$ 电子云是球形对称的,$p$ 电子云有 3 种伸展方向,$d$ 电子云有 5 种伸展方向,$f$ 电子云有 7 种伸展方向。
    \item 电子的自旋\quad 电子的自旋有两种状态,相当于顺时针和逆时针两种方向。
  \end{enumerate}
  \item 在一定电子层上、具有一定形状和伸展方向的电子云所占据的空间称为一个轨道。
  \item 核外电子排布遵循以下规律:
  \begin{description}[style=nextline,leftmargin=10em]
    \item[泡利不相容原理] 在同一个原子中,不可能有运动状态完全相同的两个电子存在。
    \item[能量最低原理] 核外电子总是尽先占有能量最低的轨道。
    \item[洪特规则] 在同一亚层的各个轨道上,电子的排布将尽可能分占不同的轨道,而且自旋方向相同。
  \end{description}
\end{enumerate}
\subsection{元素周期律和周期表}
\begin{enumerate}
  \item 元素的性质随着元素原子序数的递增而呈周期性的变化。这就是元素周期律。
  \item 周期表中具有相同电子层数而又按照原子序数递增的顺序排列的一系列元素,叫做一个周期。周期表中每个纵行叫做一个族(第 \MyRoman{8} 族包括三个纵行)。
  \item 在同一周期中,从左到右(惰性气体除外),元素的金属性减弱,非金属性增强。在同一主族中,从上到下,元素的金属性增强,非金属性减弱。
  \item 主族元素的最高正化合价等于它所在的族的序数;非金属元素的最高正化合价和它的负化合价绝对值的和等于 8。
  \item 元素周期律的发现,对于化学科学的发展有很大的影响。元素周期表是学习和研究化学的一种重要工具,对于工农业生产也有一定的指导作用。
\end{enumerate}
\begin{Review}
  \begin{question}
    \item 下列各种事实跟原子结构的哪一部分有关?
    \begin{tasks}
      \task 元素在周期表中的排列顺序;
      \task 原子量的大小;
      \task 元素具有同位素;
      \task 元素的化学性质;
      \task 元素的化合价;
      \task 元素在周期表里处于哪个周期;
      \task 主族元素在周期表里处于哪个族。
    \end{tasks}
    \item 自然界里 \ce{^14_7N} 占 99.635\% ,\ce{^16_7N} 占 0.365\%,求氮元素的近似原子量。
    \item 为什么各个电子层所能容纳的最多电子数为 $2n^2$?
    \item 某元素原子的电子排布式是 $1s^22s^22p^63s^23p^63d^{10}4s^2$,说明这个元素的原子核外有多少个电子层?每个电子层有多少个轨道,有多少个电子?
    \item 已知下列元素原子的最外电子层结构(内层已填满)为:
    \[ 3s^1, \quad 4s^24p^1, \quad 3s^23p^3\]
    它们各属于第几周期?第几族?最高正化合价是多少?
    \item 已知 \ce{K}、\ce{Ca} 分别属于第 4 周期 \MyRoman{1}A 族和 \MyRoman{2}A 族,\ce{Ar} 是第 3 周期 0 族。
    \begin{tasks}
      \task 写出 \ce{K}、\ce{Ca}、\ce{Ar} 的电子排布式;
      \task 比较三个元素的电子层结构特征;
      \task 已知 \ce{K}、\ce{Ca} 的电离能(单位:\unit{eV})为:

      \begin{tabular}{cccc}
                 & $I_1$ & $I_1$  & $I_1$  \\
         \ce{K}  & 4.341 & 31.63  & 45.72  \\
         \ce{Ca} & 6.113 & 11.87  & 50.91  \\
      \end{tabular}
      
      试从电子层结构的观点及电离能的数据,说明在化学反应中 \ce{K} 表现为 $+1$ 价、\ce{Ca} 表现为 $+2$ 价的原因。
    \end{tasks}
    \item 设计实验证明钠、镁、铝的金属性依次减弱,并写出实验步骤、现象和有关的化学方程式。
    \item 某元素 $A$ 的气态氢化物 \ce{$A$H3} 中含 \ce{H} 为 17.65\%,又知该元素的原子核中有 7 个中子,试求:
    \begin{tasks}
      \task 该元素的原子量;
      \task 它在周期表的位置及元素名称。
    \end{tasks}
    (提示:可以粗略地从原子量推知质量数,下同。)
    \item 某元素 $B$ \qty{0.9}{g} 和稀盐酸反应生成 \ce{$B$Cl3},置换出 \qty{1.12}{L} 氢气(标准状况),$B$ 的原子核里有 14 个中子,根据计算结果,写出 $B$ 的电子排布式,说明它是什么元素。
  \end{question}
\end{Review}

\begin{Exercise}*[总复习题]
  \begin{question}
    \item 填空
    \begin{tasks}
      \task \qty{0.5}{mol} 水含\CJKunderline[hidden]{\ \num{3.01e23}\ }个水分子,共含有\CJKunderline[hidden]{\ \num{9.03e23}\ }个原子,它的质量是\CJKunderline[hidden]{\ \num{9}\ }\unit{g}。
      \task 燃烧 \qty{8}{g} 硫粉可以放出 \qty{17.7}{kCal} 的热量,该反应的热化学方程式是\CJKunderline[hidden]{该反应的热化学方程式该反应的热化学方程式}。
      \task 把 4 体积二氧化硫跟 3 体积氧气在一定条件下起反应,反应后的混和气体是\CJKunderline[hidden]{\ \num{3.01e23}\ }体积,其中,三氧化硫与氧气的摩尔数之比是\CJKunderline[hidden]{\ \num{3.01e23}\ }。
      \task \qty{0.38}{g} 某卤素单质在标准状况下的体积是 \qty{120}{mL},这种卤素单质的分子量是\CJKunderline[hidden]{\ \num{3.01e23}\ },这是\CJKunderline[hidden]{\ \num{3.01e23}\ }。
      \task
    \end{tasks}
    \item 选择正确的答案填写在括号里。
    \begin{enumerate}[label=(\arabic*),leftmargin=1.7em]
      \item 下列物质(或指溶液中的溶质)含分子数最多的是\hfill(\qquad )
      \begin{tasks}(2)
        \task \qty{22.4}{L} 氢气(标准状况)
        \task \num{3.0e23} 个氧分子,
        \task \qty{9e-3}{kg} 水,
        \task $2M$ 盐酸 \qty{600}{mL},
        \task 98\% 浓硫酸(密度为 \qty{1.84}{g/cm^3})\qty{100}{mL}。
      \end{tasks}
      \item 下列物质(或指溶液中的溶质)含分子数最多的是\hfill(\qquad )
    \end{enumerate}
    \item 
    \item 
    \item 
    \item 
    \item 
    \item 
    \item 
    \item 
    \item 
    \item 
    \item 
    \item 
    \item 
    \item 
    \item 
    \item 
    \item 
    \item 
  \end{question}
\end{Exercise}