\chapter{碱金属}
我们已经学习了卤族和氧族元素的知识,对非金属元素有了一些认识。
在这一章里,我们将要学习一族叫做碱金属的金属元素。
我们已经知道锂和钠的原子结构在最外电子层都只有 1 个电子。
具有相似结构的还有钾等四种元素。
碱金属包括锂、钠、钾、铷、铯、钫六种元素,因为它们的氧化物的水化物是可溶于水的碱,所以统称为碱金属。
本章主要介绍钠及其化合物的知识。

\section{钠}
\subsection{钠的物理性质}
\begin{Experiment}
  取一块金属钠,用刀切去一端的外皮,观察钠的颜色。
\end{Experiment}
金属钠很软,可以用刀切割。切开外皮后,可以看到钠的 “真面目”呈银白色,具有美丽的光泽。

钠是热和电的良导体,密度 \qty{0.97}{g/cm^3},比水还轻,能浮在水面上,熔点 \qty{97.81}{\celsius},沸点 \qty{882.9}{\celsius}。
\subsection{钠的化学性质}
钠的化学性质非常活泼。

\subsubsection{钠跟氧气的反应}
\begin{Experiment}
  用刀切开一小块钠,观察在光亮的断面上所发生的变化。把小块钠放在燃烧匙里加热,观察发生的变化。
\end{Experiment}
钠很容易氧化,在常温下就能够跟空气里的氧气化合而生成氧化物。
切开的光亮的金属钠断面很快地发暗,主要是因为生成了一薄层氧化物的缘故。
钠受热以后能够在空气里着火燃烧,在纯净的氧气里燃烧得更为剧烈,燃烧时发出黄色的火焰。
在钠跟氧气化合的过程中,可以生成氧化钠,氧化钠不稳定,继续氧化,生成过氧化钠。
过氧化钠比较稳定,所以钠在空气里燃烧,生成的是过氧化钠。
\[ \ce{2Na + O2 \xlongequal{\quad} Na2O2} \]

\subsubsection{钠跟硫等非金属的反应}
钠除了能跟氯气直接化合外,还能跟很多其它非金属直接化合,如跟硫化合时甚至发生爆炸,生成硫化钠。
\[ \ce{2Na + S \xlongequal{\quad} Na2S} \]

\subsubsection{钠跟水的反应}
钠跟水能起剧烈的反应。
\begin{Experiment}*[righthand ratio=0.5]
向一个盛有水的烧杯里,滴入几滴酚酞溶液。然后把一小块钠(约等于 1/2 豌豆那么大小)投入烧杯里。注意观察钠跟水起反应的情形和溶液颜色的变化。再用铝箔包好一小块钠,并在铝箔上刺些小孔,用镊子夹住,放在试管口下面,用排水法收集气体(\cref{fig:4-1})。小心地取出试管,移近火焰,检验试管里是不是收集了氢气。
  \tcblower
  \begin{figurehere}
    \caption{钠跟水起反应}\label{fig:4-1}
  \end{figurehere}
\end{Experiment}
投入烧杯里的钠比水轻,浮在水面上。
钠跟水起反应放出的热,立刻使钠熔成一个闪亮的小球。
小球向各个方向迅速游动,并逐渐缩小,最后完全消失。
钠跟水起反应后,使滴有酚酞溶液的水由无色变为红色。
这个现象说明有别的物质生成,这种生成物就是氢氧化钠。
试管里收集到的气体是氢气。
\[ \ce{2Na + 2H2O \xlongequal{\quad} 2NaOH +H2 ^}\]

由于钠很容易跟空气里的氧气或水起反应,所以通常保存在煤油里,跟空气和水隔绝。

\subsection{钠的存在}
钠的性质很活泼,所以它在自然界里不能以游离态存在,只能以化合态存在。
钠的化合物在自然界里分布很广,主要以氯化钠的形式存在,也以硫酸钠、碳酸钠和硝酸钠等形式存在。

\subsection{钠的制备和用途}
工业上,可以把直流电通入熔融的氯化钠来制取钠。

钠可以用来制取过氧化钠等化合物。
钠和钾的合金(含 50\%~80\% 钾)在室温下呈液态,是原子反应堆的导热剂。
钠是一种很强的还原剂,可以把钛、锆、铌、钽等金属从它们的熔融卤化物里还原出来。
钠也应用在电光源上。
高压钠灯发出的黄光射程远,透雾能力强,对道路平面的照度比高压水银灯高几倍。

\begin{Practice}[习题]
  \begin{question}
    \item 金属的应该怎样保存?为什么?
    \item 使 \qty{0.2}{mol} 钠跟水起反应,能生成多少升的氢气(标准状况)?
    \item 钠着火时,应选用下列哪种物质和器材灭火?为什么?
    \begin{tasks}(3)
      \task 水
      \task 泡沫灭火器
      \task 干粉灭火器
    \end{tasks}
  \end{question}
\end{Practice}

\section{钠的化合物}
\subsection{钠的氧化物}
钠的氧化物有氧化钠和过氧化钠等。
氧化钠是白色的固体,跟水起剧烈的反应,生成氢氧化钠。
\[ \ce{ Na2O + H2O \xlongequal{\quad} 2NaOH }\]

过氧化钠是淡黄色的固体,也能跟水起反应,生成氢氧化钠和氧气。
\begin{Experiment}
把水滴入盛有过氧化钠固体的试管,用带火星的木条放在管口,检验有没有氧气放出。
\end{Experiment}
\[ \ce{ 2Na2O2 + 2H2O \xlongequal{\quad} 4NaOH + O2 ^ }\]

过氧化钠是强氧化剂,可以用来漂白织物、麦秆、羽毛等等。

过氧化钠跟二氧化碳起反应,生成碳酸钠和氧气。
\[ \ce{ 2Na2O2 + 2CO2 \xlongequal{\quad} 2Na2CO3 + O2 ^ }\]

因此,它用在呼吸面具上和潜水艇里作为氧气的来源。

\subsection{钠的其他重要化合物}
我们在初中学过一种重要的钠的化合物——氢氧化钠。
下面简单介绍几种重要的钠盐。
\subsubsection{硫酸钠}

硫酸钠晶体俗名芒硝(\ce{Na2SO4.10H2O})。
硫酸钠是制玻璃和造纸(制浆)的重要原料,也用在染色、纺织、制水玻璃等工业上,在医药上用作缓泻剂。
自然界里的硫酸钠主要分布在盐湖和海水里。
我国盛产芒硝。

\subsubsection{碳酸钠和碳酸氢钠}
碳酸钠(\ce{Na2CO3})俗名纯碱或苏打,是白色粉末。
碳酸钠通常情况含有结晶水 (\ce{Na2CO3.10H2O})。
在空气里碳酸钠晶体很容易失去结晶水,表面失去光泽而逐渐发暗,并渐渐碎裂成粉末。
失水以后的碳酸钠叫做无水碳酸钠。
碳酸氢钠(\ce{NaHCO3})俗名小苏打,是一种细小的白色晶体。
碳酸钠较碳酸氢钠容易溶解于水。

碳酸钠和碳酸氢钠遇到盐酸都能放出二氧化碳。
\begin{gather*}
\ce{Na2CO3 + 2HCl \xlongequal{\quad} 2NaCl + H2O + CO2 ^} \\
\ce{NaHCO3 + HCl \xlongequal{\quad} NaCl + H2O + CO2 ^} 
\end{gather*}
\begin{Experiment}
把少量盐酸分别加入盛着碳酸钠和碳酸氢钠的两个试管里。
比较它们放出二氧化碳的快慢程度。
\end{Experiment}
碳酸氢钠遇到盐酸放出二氧化碳的作用,要比碳酸钠剧烈得多。

碳酸钠很稳定,受热很难分解,碳酸氢钠却不很稳定,受热容易分解。
\begin{Experiment}*[righthand ratio=0.5]
用\cref{fig:4-2} 的装置,把碳酸钠放入试管里,约占试管容积的 1/6,并往烧杯里倒入石灰水。
加热,观察澄清的石灰水是否起变化。
把试管拿掉,换上一个放入同样容积碳酸氢钠的试管。
再加热,观察澄清的石灰水所起的变化。
\tcblower
\begin{figurehere}
  \caption{鉴别碳酸钠和碳酸氢钠}\label{fig:4-2}
\end{figurehere}
\end{Experiment}
碳酸钠受热没有变化,而碳酸氢钠受热分解,放出二氧化碳。
\[ \ce{ 2NaHCO3 \xlongequal{\triangle} Na2CO3 + H2O + CO2 ^} \]

这个反应可以用来鉴别碳酸钠和碳酸氢钠。

碳酸钠是化学工业的重要产品之一,有很多用途。
它广泛地用在玻璃、制皂、造纸、纺织等工业上,也可以用来制造其它钠的化合物。
日常生活里也常用它作洗涤剂。
碳酸氢钠是焙制糕点所用的发酵粉的主要成分之一。
在医疗上,它是治疗胃酸过多的一种药剂。

碳酸钠有天然产出的。碱性土壤里和某些盐湖里常含有碳酸钠。
我国内蒙古自治区一带的盐湖就出产大量的天然碱。
\begin{Practice}[习题]
  \begin{question}
    \item 在呼吸面具里有时用到过氧化钠,这利用了它的什么性质?
    \item 怎样断定某种碳酸钠粉末里是否含有碳酸氢钠?怎样把混在碳酸钠里的碳酸氢钠除去?
    \item 写出下列各物质间转化的化学方程式(其中离子反应还要写出相应的离子方程式)。
    \[ \ce{Na} \to \ce{NaOH} \to \ce{NaCl} \to \ce{Na2SO4}\]
    \item 空气里通常含有 0.05\% \ce{CO2}(质量百分比),计算 \qty{10}{g} 的过氧化钠能够吸收多少升空气(标准状况)里的二氧化碳。
    \item 加热 \qty{410}{g} 小苏打到再没有气体放出时,剩余的物质是什么?它的质量是多少克?
    \item 把碳酸钠和碳酸氢钠的混和物 \qty{146}{g} 加热到质量不再继续减少为止。剩下的残渣的质量是 \qty{137}{g}。计算这混和物里含有百分之几的碳酸钠?
  \end{question}
\end{Practice}

\section{碱金属元素}
\subsection{碱金属元素的原子结构和碱金属的物理性质}
碱金属元素在自然界里都以化合态存在,它们的单质由人工制得。
碱金属除铯略带金色光泽外,都呈银白色,碱金属都比较柔软,有展性,它们的密度较小,熔点较低,铯在气温稍高的时候,就呈液态。
它们的导热、导电的性能都很强。
碱金属,特别是锂、钠、钾,是金属中比较轻的。
\cref{tab:4-1} 列出各元素的原子结构和物理性质。

从\cref{tab:4-1} 可以看出,锂、钠、钾、铷、铯的原子的最外电子层的电子数是相同的,都是一个电子。
这个电子对于原子的大小是有影响的,一旦这个电子失去而变成离子,离子就显著地比原子小了。
这可以从\cref{fig:4-3} 清楚地看到。
\begin{table}
  \caption{碱金属的原子结构和物理性质}\label{tab:4-1}
\end{table}
\begin{figure}
  \caption{碱金属的原子和离子的大小示意图(数据单位是 \qty{e-10}{m})}\label{fig:4-3}
\end{figure}

碱金属原子的原子半径\footnote{锂、钠、钾等金属的原子半径是指固态金属里两个临近原子核间的距离之半。}或离子半径一般都随着电子层数的增多而增大,这是跟卤素和氧族元素的原子的变化相一致的。
碱金属的熔点、沸点一般随着原子的电子层数的增加而降低。

\subsection{焰色反应}
我们在炒菜的时候,偶有食盐或食盐水溅在煤气火焰或煤火上,火焰就呈现黄色。
火焰呈现颜色的现象应用在科学实验上,可以检验一些金属或金属化合物。
多种金属或它们的化合物在灼烧时使火焰呈特殊的颜色,这在化学上叫做\Concept{焰色反应}。
\begin{Experiment}*
把装在玻璃棒上的铂丝(也可用光洁无锈的铁丝或镍、铬、钨丝)放在酒精灯火焰(最好用煤气灯,火焰的本身颜色较微弱)里灼烧,等到跟原来的火焰颜色相同的时候,用铂丝蘸碳酸钠溶液,放在火焰上,就可以看到火焰呈黄色(\cref{fig:4-4})。
每次试完后都要用稀盐酸洗净铂丝,在火焰上灼烧到没有什么颜色,再分别蘸碳酸钾溶液、碳酸锂溶液作试验,观察火焰的颜色。
在观察钾的火焰颜色的时候,要透过蓝色的钴玻璃去观察,这样就可以滤去黄色的光,避免碳酸钾里钠的杂质所造成的干扰。
  \tcblower
  \begin{figurehere}
    \caption{焰色反应试验的操作}\label{fig:4-4}
  \end{figurehere}
\end{Experiment}
碱金属和它们的化合物都能使火焰呈现出不同的颜色,即呈现焰色反应。
此外,钙、锶、钡等金属也能呈现焰色反应。
根据焰色反应所呈现的特殊颜色,可以测定金属或金属离子的存在。
下面列出各金属或金属离子的焰色反应的颜色:
\begin{description}
  \item[锂] 紫红色
  \item[钠] 黄色
  \item[钾] 浅紫色(透过蓝色钴玻璃)
  \item[铷] 紫色
  \item[钙] 砖红色
  \item[锶] 洋红色
  \item[钡] 黄绿色
  \item[铜] 绿色
\end{description}

在节日晚上燃放的五彩缤纷的焰火,其中就有碱金属和锶、钡等金属的化合物所呈现的各种鲜艳色彩。

\subsection{碱金属的化学性质}
我们知道,钠的化学性质很活泼。
它的原子的最外电子层是一个电子,在化学反应中容易失去。
锂、钾、铷、铯等原子的最外电子层都是一个电子,都容易失去,因此它们的化学性质都很活泼。
失去电子是氧化反应,所以碱金属是强还原剂。

\subsubsection{跟非金属的反应}
碱金属跟卤素的反应,有的是很剧烈的,这我们已经知道了。

其它碱金属都象钠一样能跟氧气起反应。锂跟氧气起反应,生成氧化锂。
\[ \ce{ 4Li+ O2 \xlongequal{\quad} 2 Li2O} \]

钾、铷等跟氧气起反应,生成比过氧化物更复杂的氧化物。

碱金属能够跟大多数的非金属起反应,表现出很强的金属性。

\subsubsection{跟水的反应}
碱金属都跟水起反应,生成氢氧化物并放出氢气。这类氢氧化物都能使酚酞溶液变红色。钾跟水的反应比钠更剧烈,常使生成的氢气燃烧,并发生轻微爆炸。
\begin{Experiment}
从煤油里取出一块金属钾,放在干燥玻璃片上,用滤纸吸干煤油,切取象绿豆大小那样的一块钾,放在装冷水的烧杯里,迅速用玻璃片盖好,以免因轻微爆炸而飞溅出液体来。
反应完成后滴入几滴酚酞溶液,观察溶液颜色的变化。
\end{Experiment}

\[ \ce{2K + 2H2O \xlongequal{\quad} 2KOH +H2 ^} \]

这个反应就是钾原子失去一个电子,水里的氢离子获得一个电子成为氢原子,氢原子构成氢分子。

在这几种碱金属中,由于原子的电子层数不同,核对层数越多的电子的吸引力越小,电子就越容易失去。
随着原子的电子层数增加,原子半径的增大,碱金属的活动性增强。
以钠和钾为例,钾跟氧气、跟水的反应都比钠刷烈,这些事实都可说明原子结构跟性质的关系。

\subsection{锂、钾、铷、铯的用途}
碱金属在生产和现代科学技术上都有一定的用途。
锂用以制备有机化学工业上的催化剂、多种合金、高强度玻璃等。
锂还用于制热核反应的材料氚。
钾的化合物象 \ce{KCl}、\ce{K2SO4} 等是重要的肥料。
铷和铯因在普通光的照射下能够放出电子,用于制光电管等。

钾的许多重要化合物,如氯化钾、硫酸钾、碳酸钾等都是钾肥。
我们在初中化学里已学过钾肥的初步知识。
土壤里钾的含量并不少,但大部分以钾的矿物形式存在,例如,正长石、 白云母\footnote{正长石:\ce{KAlSi3O8};白云母:\ce{KH2Al3Si3O12}。}等等。
这些矿物难溶于水,作物不能利用,只有在长期风化 (在土壤里受到空气、水分、酸的作用) 过程中,才能逐步转化为作物可以吸收的水溶性的钾的化合物。
因此,土壤里的钾常常不能满足作物生长的需要,人们往往要施用钾肥加以补充。

通常施用的钾肥主要是各种钾盐,如氯化钾、硫酸钾、碳酸钾(草木灰的主要成分)等。
这些钾盐都易溶于水,在溶液里钾以离子形式存在,易被作物吸收,所以,这些钾肥都是速效的。
但必须注意的是,由于它们易溶于水,在施用时要防止雨水淋失。

在科学种田、夺取高产的过程中,施用钾肥时,要因地制宜,注意氮、磷、钾三种肥料的合理配合。

\begin{Practice}[习题]
  \begin{question}
    \item 试比较钠和钾的物理性质和化学性质。
    \item 在卤族元素和碱金属元素中,哪一族元素的原子比它们相应的离子小,哪一族元素的原子比它们相应的离子大?试举例说明。
    \item 写出下列反应的化学方程式。
    \begin{tasks}
      \task \ce{ K2O +  H2O -> }
      \task \ce{ K2O2 + H2O -> }
      \task \ce{ Li2O + H2O -> }
    \end{tasks}
    \item 用电子得失的观点来说明下列氧化—还原反应。
    \begin{tasks}
      \task \ce{ 2K +  Cl2 $\xlongequal{\quad}$ 2KCl }
      \task \ce{ 2K + 2H2O $\xlongequal{\quad}$ 2KOH + H2 ^ }
    \end{tasks}
    \item 把 \qty{4}{g} 氢氧化钠溶解在水里,制成溶液。这溶液能跟多少毫升的密度为 \qty{1.19}{g/cm^3} 的盐酸起反应?
  \end{question}
\end{Practice}
\section*{内容提要}
碱金属是一族金属元素,它们的原子结构的共同之点是次外层有 8 个电子(锂是 2 个)和最外电子层都只有一个电子,在化学反应中容易失去电子,因此,它们的化学性质基本相似;差别之处是核电荷数不同,电子层数不同,原子半径也不同,因而碱金属元素的性质既相似又有差别。

碱金属的化学性质主要是强的金属性,随着原子半径的增大而金属性增强。
它们的单质都是强还原剂。

碱金属的化学性质:
\begin{itemize}
  \item 跟卤素的反应——生成卤化物。
  \item 跟氧气的反应——生成氧化物、过氧化物等等。
  \item 跟水的反应——生成氢氧化物,放出氢气。
\end{itemize}

碱金属和它们的化合物能使火焰呈现不同的颜色,即呈现焰色反应。
根据焰色反应所呈现的特殊颜色,可以判断某些金属或金属离子的存在。

\begin{Review}
  \begin{question}
    \item 回答下列问题:
    \begin{tasks}
      \task 使用金属钠
      \task 使用金属钠
      \task 使用金属钠
      \task 使用金属钠
    \end{tasks}
    \item 
    \item 选择正确的答案填写在括号里。
    \begin{enumerate}[itemindent=1.7em]
      \item 下列哪些物质可用来制氧气:\hfill(\hspace{2em})
      \begin{tasks}(5)
        \task \ce{Na2O2}
        \task \ce{CaCO3}
        \task \ce{KClO3}
        \task \ce{H2SO4}
        \task \ce{H2O}
      \end{tasks}
      \item 钠离子\hfill(\hspace{2em})
      \begin{tasks}(2)
        \task 遇水放出氢气,
        \task 要保存在煤油里,
        \task 比钠原子多一个电子,
        \task 在无色火焰上灼烧显黄色。
      \end{tasks}
      \item 下列哪些物质放在水里后,溶液显碱性:\hfill(\hspace{2em})
      \begin{tasks}(4)
        \task \ce{Na2O2}
        \task \ce{NaCl}
        \task \ce{CuO}
        \task \ce{K}
      \end{tasks}
    \end{enumerate}
    \item 写出下列物质转化的化学方程式(其中的离子反应还要写出相应的离子方程式)。
    \[ \ce{Na} \to \ce{Na2O2} \to \ce{NaOH} \to \ce{Na2CO3} \to \ce{CaCO3} \to \ce{Ca(HCO3)2}\]
  \end{question}
\end{Review}