\chapter{摩尔}\label{chp:mol}
摩尔是国际单位制的一种基本单位,它表示物质的量。
摩尔广泛地应用于科学研究、工农业生产等等方面。
在中学化学里,摩尔应用于计算微粒的数量、物质的质量、气体的体积、 溶液的浓度、反应过程的热量变化等等。

我们要重视摩尔的学习,理解摩尔的意义,学会使用摩尔这个基本单位的方法,并在以后各章的学习里不断应用。

\section{摩尔}
\subsection{摩尔}
我们在初中化学里,学习过原子、分子、离子等构成物质的微粒,知道单个这样的微粒是肉眼看不见的,也是难于称量的。
但是,在实验室里取用的物质,不论是单质还是化合物,应是看得见的、可以称量的。
生产上,物质的用量当然更大,常以吨计。
物质之间的反应,既是按照一定个数、肉眼看不见的原子、分子或离子来进行,而实践上又是以可称量的物质进行反应。
所以,很需要把微粒跟可称量的物质联系起来。

怎样联系起来呢? 就是要建立一种物质的量的基本单位,这个单位是含有同数的原子、分子、离子等等的集体。
科学上,已经建立把微粒跟微粒集体联系起来的单位。
那么,采取多大的集体作为物质的量的单位呢?

近年来,科学上应用 \qty{12}{g} \ce{^12C}(或 \qty{0.012}{kg} \ce{^12C})来衡量碳原子集体。\ce{^12C} 就是原子核里有 6 个质子和 6 个中子的碳原子。根据实验测定,\qty{12}{g} \ce{^12C} 含有的原子数就是阿伏伽德罗\footnote{阿伏伽德罗(Avogadro 1776--1856)是意大利物理学家。}常数。
阿伏伽德罗常数经过实验已测得比较精确的数值。
在这里,采用 \num{6.02e23} 这个非常近似的数值。

\emph{摩尔是表示物质的量的单位,每摩尔物质含有阿伏伽德罗常数个微粒}。例如:
\begin{itemize}[label={},labelsep=0pt]
  \item \qty{1}{mol}\footnote{摩尔可以简称为摩,符号 \unit{mol}。}的碳原子含有 \num{6.02e23} 个碳原子,
  \item \qty{1}{mol} 的氢原子含有 \num{6.02e23} 个氢原子,
  \item \qty{1}{mol} 的氧分子含有 \num{6.02e23} 个氧分子,
  \item \qty{1}{mol} 的水分子含有 \num{6.02e23} 个水分子,
  \item \qty{1}{mol} 的二氧化碳分子含有 \num{6.02e23} 个二氧化碳分子,
  \item \qty{1}{mol} 的氢离子含有 \num{6.02e23} 个氢离子,
  \item \qty{1}{mol} 的氢氧根离子含有 \num{6.02e23} 个氢氧根离子。
\end{itemize}

阿伏伽德罗常数是很大的数值,但摩尔作为物质的量的单位应用极为方便。
因为实验测得 \qty{1}{mol} \ce{^12C} 的质量是 \qty{12}{g},即含有 \num{6.02e23} 个碳原子的质量。
由此我们可以推算 \qty{1}{mol} 任何原子的质量。

一种元素的原子量是以 \ce{^12C} 的质量的 $1/12$ 作为标准,其它元素原子的质量跟它相比较所得的数值,如氧的原子量是 16,氢的原子量是 1,铁的原子量是 55.85,等等。
1 个碳原子的质量跟 1 个氧原子的质量之比是 $12:16$。
\qty{1}{mol} 碳原子跟 \qty{1}{mol} 氧原子所含有的原子数相同,都是 \num{6.02e23}。
\qty{1}{mol} 碳原子是 \qty{12}{g},那么 \qty{1}{mol} 氧原子就是 \qty{16}{g}。
同理,\qty{1}{mol} 任何原子的质量就是以克为单位,数值上等于该种原子的原子量。
由此我们可以直接推知:
\begin{itemize}[label={},labelsep=0pt]
  \item 氢的原子量是 1,\qty{1}{mol} 氢原子的质量是 \qty{1}{g},
  \item 铁的原子量是 55.85,\qty{1}{mol} 铁原子的质量是 \qty{55.85}{g}。
\end{itemize}

其次,我们用摩尔来衡量双原子分子或多原子分子构成的各种物质的时候,那么同样地可以推知,\qty{1}{mol} 任何分子的质量,就是以克为单位,数值上等于该种分子的分子量。
\begin{itemize}[label={},labelsep=0pt]
  \item 氢气的分子量是 2,\qty{1}{mol} 氢气的质量是 \qty{2}{g},
  \item 氧气的分子量是 32,\qty{1}{mol} 氧气的质量是 \qty{32}{g},
  \item 二氧化碳的分子量是 44,\qty{1}{mol} 二氧化碳的质量是 \qty{44}{g},
  \item 水的分子量是 18,\qty{1}{mol} 水的质量是 \qty{18}{g}。
\end{itemize}

当摩尔应用于表示离子的时候,同样可以推知 \qty{1}{mol} 离子的质量。
由于电子的质量过于微小,失去或得到的电子的质量可以略去不计。
\begin{itemize}[label={},labelsep=0pt]
  \item \qty{1}{mol} \ce{H+} 的质量是 \qty{1}{g},
  \item \qty{1}{mol} \ce{OH-} 的质量是 \qty{17}{g},
  \item \qty{1}{mol} \ce{Cl-} 的质量是 \qty{35.5}{g}。
\end{itemize}

对于离子化合物也可以同样推知,如 \qty{1}{mol} \ce{NaCl} 的质量是 \qty{58.5}{g}。

总之,摩尔象一座桥梁把单个的、肉眼看不见的微粒跟很大数量的微粒集体、可称量的物质之间联系起来了。

应用摩尔来衡量物质的量,在科学技术上带来了方便。
如从化学反应中反应物和生成物之间的原子、分子等微粒的比值,可以直接知道它们之间摩尔的数目之比,
\begin{align*}
\ce{\underset{\qty{1}{mol}}{\ce{C_{\phantom{2}}}} + \underset{\qty{1}{mol}}{\ce{O2}} & \xlongequal{\quad} \underset{\qty{1}{mol}}{\ce{CO2}}}\\
\ce{\underset{\qty{1}{mol}}{\ce{Mg}} + \underset{\qty{2}{mol}}{\ce{2HCl_{\phantom{1}}}} & \xlongequal{\quad} \underset{\qty{1}{mol}}{\ce{MgCl2}} + \underset{\qty{1}{mol}}{\ce{H2}} ^} 
\end{align*}

\subsection{关于摩尔质量的计算}
\qty{1}{mol} 物质的质量通常也叫做该物质的\Concept{摩尔质量}。摩尔质量的单位是“克/摩尔(\unit{g/mol})”。
物质的量、物质的质量和摩尔质量之间的关系可以用下式表示:
\[ \frac{\text{物质的质量}(\unit{g})}{\text{摩尔质量}(\unit{g/mol})}=\text{物质的量}(\unit{mol})\]

\begin{example}
  \qty{90}{g} 水相当于多少摩尔水分子?
\end{example}
\begin{solution}
  水的分子量是 18,水的摩尔质量是 \qty{18}{g/mol}。
  \[ \frac{\qty{90}{g}}{\qty{18}{g/mol}}=\qty{5}{mol}\]
  答:\qty{90}{g} 水相当于 \qty{5}{mol} 水,也可以说 \qty{90}{g} 水所含的摩尔数是 5 。
\end{solution}

\begin{example}
  \qty{2.5}{mol} 铜原子的质量是多少克?
\end{example}
\begin{solution}
  铜的原子量是 63.5,铜的摩尔质量是 \qty{63.5}{g/mol}。
  \[ \qty{2.5}{mol}\ \text{铜的质量}= \qty{63.5}{g/mol} \times \qty{2.5}{mol} = \qty{158.8}{g} \]
  答:\qty{2.5}{mol} 铜原子(或简称 \qty{2.5}{mol} 铜)的质量等于 \qty{158.8}{g}。
\end{solution}

\begin{example}
  \qty{4.9}{g} 硫酸里含有多少硫酸分子?
\end{example}
\begin{solution}
  硫酸的分子量是 98,硫酸的摩尔质量是 \qty{98}{g/mol}。
  \[\begin{split} 
  \frac{\qty{4.9}{g}}{\qty{98}{g/mol}} & = \qty{0.05}{mol}\\
  \qty{4.9}{g}\ \text{硫酸的分子数}    & = \qty{6.02e23}{mol^{-1}} \times \qty{0.05}{mol}\\
   & = \num{3.01e22} \\
  \end{split}\]
  答:\qty{4.9}{g} 克硫酸里含有 \num{3.01e22} 个分子。
\end{solution}

\begin{Practice}[习题]
  \begin{question}
    \item 2 个氧分子、\qty{2}{g} 氧气、\qty{2}{mol} 氧分子有什么区别?
    \item 选择正确的答案填写在括号里。
    
    \noindent\qty{0.5}{mol} 氢气含有(\hspace{2em})
    \begin{tasks}(3)
      \task 0.5 个氢分子
      \task 1 个氢原子
      \task \num{6.02e23} 个氢原子
      \task \num{3.01e23} 个氢分子
      \task \num{3.01e12} 个氢分子
    \end{tasks}
    \item 计算 \qty{1}{mol} 下列物质的质量。
    \begin{tasks}(2)
      \task 氦、镁、氯原子、磷原子。
      \task 硝酸、硝酸铵、蔗糖(\ce{C12H22O11})
    \end{tasks}
    \item 下列物质的量各等于多少摩尔。
    \begin{tasks}
      \task \qty{1}{kg}硫原子,\qty{0.5}{kg}铝原子,\qty{0.25}{kg} 锌原子
      \task \qty{22}{g} 二氧化碳,\qty{500}{g} 氯化钠,\qty{1.5}{kg} 蔗糖
    \end{tasks}
    \item 分别列出铝、铁、铅的摩尔质量。根据 \qty{20}{\celsius} 时,铝、 铁、铅 的密度\footnote{按照国际单位制,密度的单位应是千克/米$^3$,在这里,暂按克/厘米$^3$(\unit{g/cm^3})或克/升(\unit{g/L})为单位。}分别是 \qtylist{2.70;7.86;11.3}{g/cm^3},计算 \qty{1}{mol} 铝、铁、铅的体积。
    \item 在 \qty{15}{\celsius} 时,蔗糖的密度是 \qty{1.588}{g/cm^3},计算 \qty{1}{mol} 蔗糖的体积。
    \item 分解氯酸钾制氧气的时候,制 \qty{0.6}{mol} 氧气需要多少摩尔的氯酸钾?
    \item 跟含 \qty{4}{g} 氢氧化钠的溶液起反应使生成正盐,需用下列酸各多少摩尔。
    \begin{tasks}(5)
      \task \ce{HCl}
      \task \ce{HNO3}
      \task \ce{H2SO4}
      \task \ce{H3PO4}
      \task \ce{HClO3}
    \end{tasks}
    \item 硫酸铵、硝酸铵、磷酸氢二铵(\ce{(NH4)2HPO4})、尿素都可以作为氮肥。试计算:
    \begin{tasks}
      \task \qty{1}{mol} 上述物质的质量各是多少克,
      \task \qty{1}{mol} 上述物质的质量各含多少摩尔氮原子。
    \end{tasks}
  \end{question}
\end{Practice}

\section{气体摩尔体积}
\subsection{气体摩尔体积}
对于固态或液态的物质来说,\qty{1}{mol} 各种物质的体积是不相同的。
例如,\qty{20}{\celsius} 时,\qty{1}{mol} 铁的体积是 \qty{7.1}{cm^3},\qty{1}{mol} 铝的体积是 \qty{10}{cm^3},\qty{1}{mol} 铅的体积是 \qty{18.3}{cm^3}(\cref{fig:1-1});\qty{1}{mol} 水的体积是 \qty{18.0}{cm^3}, \qty{1}{mol} 纯硫酸的体积是 \qty{54.1}{cm^3},\qty{1}{mol} 蔗糖的体积是 \qty{215.5}{cm^3}(图 1-2)。
\begin{figure}
  \includegraphics{1-1.pdf}
  \caption{\qty{1}{mol}的几种金属}\label{fig:1-1}
  \caption{\qty{1}{mol}的几种化合物}\label{fig:1-2}
\end{figure}

\qty{1}{mol} 固态或液态的物质的体积为什么不同呢? 
这因为对固态或液态的物质来说,构成它们的微粒间的距离是很小的,\qty{1}{mol} 物质的体积主要决定于原子、分子或离子的大小。
构成不同物质的原子、分子或离子的大小是不同的,所以它们 \qty{1}{mol} 的体积也就有所不同。

但是,对气体来说,情况就大不相同。

我们分别计算 \qty{1}{mol} 氢气、氧气和二氧化碳在标准状况\footnote{标准状况是指压强为 \qty{1}{atm} 和温度为 \qty{0}{\celsius}。根据国际单位制压强单位是帕(\unit{Pa})。在这里暂用标准大气压(\unit{atm})。$\qty{1}{atm}=\qty{101325}{Pa}$} 时的体积。
氢气的摩尔质量是 \qty{2}{g/mol},氧气的摩尔质量是 \qty{32}{g/mol},二氧化碳的摩尔质量是 \qty{44}{g/mol},同时它们的密度分别是 \qtylist{0.0899;1.429;1.977}{g/L}。
这样就可以算出上述气体在标准状况时所占的体积。
\begin{align*}
  \text{氢气的摩尔体积}&=\frac{\qty{2.016}{g/mol}}{\qty{0.0899}{g/L}}=\qty{22.4}{L/mol}\\
  \text{氧气的摩尔体积}&=\frac{\qty{32.0}{g/mol}}{\qty{1.429}{g/L}}=\qty{22.4}{L/mol}\\
  \text{二氧化碳的摩尔体积}&=\frac{\qty{44.0}{g/mol}}{\qty{1.977}{g/L}}=\qty{22.3}{L/mol}
\end{align*}

从上面几个例子可以看出,在标准状况时,\qty{1}{mol} 三种气体的体积都约是 \qty{22.4}{L}。
而且经过许多实验发现和证实,\qty{1}{mol} 的任何气体在标准状况下所占的体积都约是 \qty{22.4}{L}(\cref{fig:1-3})。
\begin{figure}
  \begin{minipage}{0.48\linewidth}\centering
    % \includegraphics{1-3.pdf}
    \caption{气体摩尔体积}\label{fig:1-3}
  \end{minipage}
  \begin{minipage}{0.48\linewidth}\centering
    % \includegraphics{1-4.pdf}
    \caption{气体分子的运动和距离}\label{fig:1-4}
  \end{minipage}
\end{figure}

\emph{在标准状况下,\qty{1}{mol} 的任何气体所占的体积都约是 \qty{22.4}{L},这个体积叫做}\Concept{气体摩尔体积}。

为什么 \qty{1}{mol} 的固体、液体的体积各不相同,而 \qty{1}{mol} 气体在标准状况时所占的体积都相同呢? 
这要从气态物质的结构去找原因。
气体的分子在较大的空间里迅速地运动着(\cref{fig:1-4})。
在通常情况下气态物质的体积要比它在液态或固态时大 1000 倍左右,这是因为气体分子间有着较大的距离。
通常情况下一般气体的分子直径约是 \qty{4e-10}{m},分子间的平均距离约是 \qty{4e-9}{m},即平均距离是分子直径的 10 倍左右(\cref{fig:1-5})。
这就可以推知,气体体积主要决定于分子间的平均距离,而不象液体或固体那样,体积主要决定于分子的大小。
在标准状况下,不同气体分子间的平均距离几乎是相等的,所以任何物质的气体摩尔体积都约是 \qty{22.4}{L/mol}。
\begin{figure}
  % \includegraphics{1-5.pdf}
  \caption{固体、液体跟气体的分子间距离比较示意图(以碘为例)}\label{fig:1-5}
\end{figure}

气体摩尔体积约是 \qty{22.4}{L/mol},为什么一定要加上标准状况这个条件呢: 这是因为气体的体积较大地受到温度和压强的影响。
温度升高时,气体分子间的平均距离增大,温度降低时平均距离减小;压强增大时,气体分子间的平均距离减小,压强减小时,平均距离增大。
各种气体在一定温度和压强下,分子间的平均距离是相等的。
在一定的温度和压强下,气体体积的大小只随分子数的多少而变化,相同的体积含有相同的分子数。
这是经过生产上和科学实验的许多事实所证明的。

在相同的温度和压强下,相同体积的任何气体都含有相同数目的分子,这就是阿伏伽德罗定律。

\begin{Theorem}{讨论}
如果已经知道水的分子式是 \ce{H2O},我们能够根据氢气跟氧气化合成水蒸气的体积比是 $2:1:2$,应用阿伏伽德罗定律来证明 1 个氧分子里含有 2 个氧原子吗?
\end{Theorem}

\subsection{关于气体摩尔体积的计算}
\begin{example}
  \qty{5.5}{g} 氨相当于多少摩尔氨,在标准状况时它的体积应是多少升?
\end{example}
\begin{solution}
  氨的分子量是 17,氨的摩尔质量是 \qty{17}{g/mol}。
  \[\begin{split} 
    \frac{\qty{5.5}{g}}{\qty{17}{g/mol}} &= \qty{0.32}{mol} \\
    \qty{5.5}{g}\ \text{氨的体积} & = \qty{22.4}{L/mol} \times \qty{0.32}{mol} =\qty{7.2}{L}
  \end{split}\]
  答:\qty{5.5}{g} 氨相当于 \qty{0.32}{mol} 的氨,在标准状况时,它的体积是 \qty{7.2}{L}。
\end{solution}

\begin{example}
  在实验室里使稀盐酸跟锌起反应,在标准状况时生成 \qty{3.36}{L} 氢气。计算需要多少摩尔的 \ce{HCL} 和锌。
\end{example}
\begin{solution}
  设 $x$ 为所需多少摩尔的锌,$y$ 为所需多少摩尔的 \ce{HCl}。
  \begin{gather*} 
    \ce{\underset{\qty{1}{mol}}{\ce{Zn}} + \underset{\qty{2}{mol}}{\ce{HCl}} \xlongequal{\quad} ZnCl2 + \underset{\qty{22.4}{L}}{\ce{H2}} ^} \\
   \scriptstyle x\quad\quad\ y \qquad\qquad\ \ \qquad \qty{3.36}{L}
  \end{gather*}
  \[\begin{split} 
    x & = \frac{\qty{1}{mol}\times\qty{3.36}{L}}{\qty{22.4}{L}} = \qty{0.15}{mol}\\
    y & = \frac{\qty{2}{mol}\times\qty{3.36}{L}}{\qty{22.4}{L}} = \qty{0.30}{mol}\\
  \end{split}\]
  答:需 \qty{0.15}{mol} 锌和 \qty{0.30}{mol} \ce{HCl}。
\end{solution}

\begin{example}
  在标准状况时,\qty{0.20}{L} 的容器里所含一氧化碳的质量为 \qty{0.25}{g},计算一氧化碳的分子量。
\end{example}

根据摩尔体积可以计算出一氧化碳的摩尔质量,而摩尔质量的数值就等于它的分子量。

\begin{solution}
  \[\begin{split}
    \text{一氧化碳的摩尔质量} & = \text{一氧化碳的密度} \times \text{一氧化碳的摩尔体积}\\
    & = \frac{\qty{0.25}{g}}{\qty{0.20}{L}} \times \qty{22.4}{L} = \qty{28}{g/mol}\\ 
    \text{一氧化碳的分子量} & = 28
  \end{split}\]
  答:一氧化碳的分子量是 28。
\end{solution}

\begin{Practice}[习题]
  \begin{question}
    \item 改正下列说法里可能有的错误,并说明理由。
    \begin{tasks}
      \task \qty{1}{mol} 任何气体的体积都是 \qty{22.4}{L}。
      \task \qty{1}{mol} 氢气的质量是 \qty{1}{g},它所占的体积是 \qty{22.4}{L}。
      \task \qty{1}{mol} 任何物质在标准状况时所占的体积都约是 \qty{22.4}{L}。
      \task \qty{1}{mol} 氢气和 \qty{1}{mol} 水所含的分子数相同,在标准状况时所占体积都约是 \qty{22.4}{L}。
    \end{tasks}
    \item 在标准状况时,\qty{1}{L} 氮气约含有多少个氮分子?
    \item 在标准状况时,\qty{15}{g} 氟气所占的体积比 \qty{1}{g} 氢气所占的体积是大还是小?
    \item 在标准状况时,\qty{4.4}{g} 二氧化碳的体积跟多少克氧气的体积相等?
    \item 在实验室制备氢气的时候,用 \qty{0.1}{mol} 的锌跟足量稀盐酸起反应,计算所产生的氢气的体积(在标准状况)。
    \item 氮气在标准状况时的密度是 \qty{1.25}{g/L},液态氮在 \qty{-195.8}{\celsius} 的密度是 \qty{0.808}{g/cm^3},固态氮在  \qty{-232.5}{\celsius} 的密度是 \qty{1.026}{g/cm^3},比较 \qty{1}{mol} 的氮在气态、液态、固态各占多少体积。
    \item 在标准状况时,\qty{235}{mL} 某种气体的质量是 \qty{0.406}{g},计算这种气体的分子量。
  \end{question}
\end{Practice}

\section{摩尔浓度}
\subsection{摩尔浓度}
我们在初中化学里学习过百分比浓度,应用这种表示溶液浓度的方法,可以了解和计算一定质量的溶液中所含溶质的质量。
但是,我们在许多场合取用溶液时,一般不是去称它的质量而是量它的体积。
同时,物质起反应时,反应物和生成物各是多少摩尔相互之间有一定的关系; 知道一定体积溶液里含多少摩尔溶质,运算起来很便利。
因此,摩尔浓度是生产上和科学实验上常用的一种表示溶液浓度的重要方法。

\emph{以 \qty{1}{L} 溶液里含有多少摩尔溶质来表示的溶液浓度叫}\Concept{摩尔浓度}。摩尔浓度\footnote{按照国际单位制的规定,物质的量浓度的单位是摩尔/米$^3$(\unit{mol/m^3})。在这里仍暂按习惯沿用摩尔浓度,单位也用摩尔/升(\unit{mol/L})。}通常用 $M$ 表示。
\[
\text{摩尔浓度}(M) = \frac{\text{溶质的量}(\unit{mol})}{\text{溶液的体积}(\unit{L})}
\]

\qty{1}{L} 溶液中含 \qty{1}{mol} 的溶质, 这种溶液就是 \qty{1}{mol} 浓度的溶液,通常用 $1M$ 表示。如蔗糖的摩尔质量是 \qty{342}{g/mol}。
把 \qty{342}{g} 蔗糖溶解在适量水里配成 \qty{1}{L} 溶液,它的摩尔浓度就是 \qty{1}{mol/L} 或 $1M$。
\qty{1}{L} 溶液中含 \qty{171}{g} 蔗糖,它的摩尔浓度就是 $0.5M$。
又如 \qty{1}{L} 的氯化钠的质量是 \qty{58.5}{g} 克,把 \qty{58.5}{g} 氯化钠溶解在适量水里制成 \qty{1}{L} 溶液时, 它的摩尔浓度就是 $1M$。
\qty{1}{L} 溶液中含 \qty{29.3}{g} 氯化钠的溶液浓度就是 $0.5M$。

溶液用摩尔浓度表示时,常简称为摩尔溶液。

\begin{Experiment}*%[righthand ratio=0.3]
  在天平上称出 \qty{29.3}{g} 氯化钠,把称好的氯化钠放在烧杯里,用适量蒸馏水使它完全溶解。把制得的溶液小心地注入 \qty{1000}{mL} 的容量瓶(\cref{fig:1-6})。用蒸馏水洗涤烧杯内壁两次,把每次洗下来的水都注入容量瓶。振荡容量瓶里的溶液使混和均匀。然后缓缓地把蒸馏水直接注入容量瓶直到液面接近刻度 \qtyrange{2}{3}{cm} 处。改用胶头滴管加水到瓶颈刻度的地方,使溶液的凹面正好跟刻度相平。把容量瓶塞好,反复摇匀。
  \tcblower
  \begin{figurehere}
    \caption{配置摩尔浓度的溶液}\label{fig:1-6}
  \end{figurehere}
\end{Experiment}

这样配制成的溶液就是 $0.5M$ 的氯化钠溶液。

\subsection{在摩尔溶液中溶质微粒的数目}

\qty{1}{mol} 任何物质的微粒数都是 \num{6.02e23}。
\qty{1}{L} $1M$ 的蔗糖溶液含有 \num{6.02e23} 个蔗糖分子。
对于非电解质来说,体积相同的同摩尔浓度的溶液都应含有相同的溶质分子数。

但是,对于溶质为离子化合物或在水里完全电离的共价化合物等电解质来说,情况就比较复杂。
例如,氯化钠溶解在水里电离为 \ce{Na+} 和 \ce{Cl-}。
所以,\qty{1}{L} $1M$ 的 \ce{NaCl} 溶液含有 \num{6.02e23} 个 \ce{Na+} 和 \num{6.02e23} 个 \ce{Cl-}。
同样地,\qty{1}{L} $1M$ 的 \ce{NaOH} 溶液含有 \num{6.02e23} 个 \ce{Na+} 和 \num{6.02e23} 个 \ce{OH-};\qty{1}{L} $1M$ 的 \ce{CaCl2} 溶液含有 \num{6.02e23} 个 \ce{Ca^{2+}} 和 \numproduct{2x6.02e23} 个 \ce{Cl-}。
又例如 \qty{1}{L} $1M$ 的 \ce{HCl} 溶液里含有 \num{6.02e23} 个 \ce{H+} 和 \num{6.02e23} 个 \ce{Cl-}。

\subsection{关于摩尔浓度的计算}
\subsubsection{已知溶质的质量和溶液的体积,计算溶液的摩尔浓度}
\begin{example}
在 \qty{200}{mL} 稀盐酸里溶有 \qty{0.73}{g} \ce{HCl},计算溶液的摩尔浓度。
\end{example}
摩尔浓度所表示的就是 \qty{1}{L} 溶液里含多少摩尔溶质,在这题里,就是要算出 \qty{1}{L} 溶液里含多少摩尔的 \ce{HCl}。
\begin{solution}
\ce{HCl} 的分子量是 36.5,它的摩尔质量是 \qty{36.5}{g/mol}。\qty{0.73}{g} \ce{HCl} 相当于:
\[ \frac{\qty{0.73}{g}}{\qty{36.5}{g/mol}} = \qty{0.02}{mol}\]

\qty{1000}{mL} 溶液中含 \ce{HCl}
\[ \qty{0.02}{mol} \times \frac{\qty{1000}{mL}}{\qty{200}{mL}} = \qty{0.1}{mol}\]
答:这种稀盐酸的浓度是 $0.1M$。
\end{solution}

\subsubsection{已知溶液的摩尔浓度,计算一定体积的溶液里所含溶质的质量}
\begin{example}
  计算配制 \qty{500}{mL} $0.1M$ 的 \ce{NaOH} 溶液所需 \ce{NaOH} 的质量。
\end{example}
\begin{solution}
  \ce{NaOH} 的分子量是 40,它的摩尔质量是 \qty{40}{g/mol}。

  \qty{0.1}{mol} 的 \ce{NaOH} 的质量 $= \qty{40}{g/mol} \times \qty{0.1}{mol}=\qty{4}{g}$

  \qty{500}{mL} $0.1M$ \ce{NaOH} 溶液所含 \ce{NaOH} 的质量是:
  \[ \frac{\qty{500}{mL}\times\qty{4}{g}}{\qty{1000}{mL}} =\qty{2}{g}\]
  答: 制备 \qty{500}{mL} $0.1M$ 的 \ce{NaOH} 溶液需 \qty{2}{g} \ce{NaOH}。
\end{solution}

\subsubsection{应用摩尔浓度作关于浓溶液稀释的计算}
\begin{example}
  20\% 的蔗糖溶液 \qty{200}{g},加适量的水稀释到 \qty{1}{L},计算稀释后蔗糖溶液的摩尔浓度。
\end{example}
\begin{solution}
  蔗糖 \ce{C12H22O11} 的分子量是 342,蔗糖的摩尔质量是 \qty{342}{g/mol}。

  溶液里所含蔗糖的质量是 $\qty{200}{g} \times 20\% = \qty{40}{g}$
  \[\frac{\qty{40}{g}}{\qty{342}{g/mol}}=\qty{0.117}{mol}\]

  \qty{1}{L} 蔗糖溶液里含有 \qty{0.117}{mol} 蔗糖。

\noindent 答: 蔗糖溶液的摩尔浓度是 $0.117M$。
\end{solution}
\begin{example}
  计算配制 \qty{500}{mL} $1M$ 的硫酸溶液需要密度为 \qty{1.836}{g/cm^3} 的浓硫酸 (98\% \ce{H2SO4})多少毫升?
\end{example}

浓硫酸稀释后,所含 \ce{H2SO4} 的质量是不变的,因而在硫酸溶液里多少摩尔  \ce{H2SO4} 等于浓硫酸里多少摩尔的 \ce{H2SO4}。 
先算出硫酸溶液里含多少摩尔 \ce{H2SO4},再算出浓硫酸的摩尔浓度,从所含多少摩尔 \ce{H2SO4} 可以算出需要的浓硫酸的体积。
\begin{solution}
硫酸溶液中含 \ce{H2SO4}
\[ \frac{\qty{500}{mL}}{\qty{1000}{mL}}\times \qty{1}{mol} =\qty{0.5}{mol} \]

硫酸的摩尔质量 $=\qty{98}{g/mol}$

\qty{1000}{mL} 浓硫酸中 \ce{H2SO4} 的质量
\[ = \qty{1000}{mL} \times \qty{1.836}{mL} \times \frac{98}{100} = \qty{1799}{g}\]

\qty{1000}{mL} 浓硫酸里含 \ce{H2SO4}
\[ \frac{\qty{1799}{g}}{\qty{98}{g/mol}}=\qty{18.4}{mol} \]

含 \qty{0.5}{mol} \ce{H2SO4} 的浓硫酸的体积
\[ = \frac{\qty{0.5}{mol}}{\qty{18.4}{mol/L}}=\qty{0.0272}{L}=\qty{27.2}{mL} \]
需要浓硫酸 (98\% \ce{H2SO4})\qty{27.2}{mL}。
\end{solution}

\subsubsection{已知起反应的两种溶液的摩尔浓度以及其中一种溶液的体积,计算另一种溶液的体积}
\begin{example}
  中和 \qty{1}{L} $0.5M$ \ce{NaOH} 溶液,需要多少升的 $1M$ 的 \ce{H2SO4} 溶液?
\end{example}
\begin{solution}
  \[ \ce{\underset{\qty{2}{mol}}{\ce{2NaOH_$\phantom{2}$}} + \underset{\qty{1}{mol}}{\ce{H2SO4}} \xlongequal{\quad} Na2SO4 + 2H2O} \]

  \qty{1}{L} $0.5M$ 溶液中含 \ce{NaOH}
  \[ \qty{1}{L} \times \qty{0.5}{mol/L} = \qty{0.5}{mol}\]

中和 \qty{0.5}{mol} \ce{NaOH} 需 \ce{H2SO4}
\[ \qty{0.5}{mol}\times\frac{1}{2}=\qty{0.25}{mol} \]

含 \qty{0.25}{mol} 的 $1M$ \ce{H2SO4} 溶液的体积
\[= \frac{\qty{1}{L} \times \qty{0.25}{mol}}{\qty{1}{mol}} = \qty{0.25}{L}\]

答: 中和 \qty{1}{L} $0.5M$ \ce{NaOH} 溶液需 $1M$ 的 \ce{H2SO4} 溶液 \qty{0.25}{L}。
\end{solution}

\begin{Practice}[习题]
  \begin{question}
    \item 制备下列各物质的 $0.2M$ 溶液各 \qty{50}{mL},需用下列物质各多少克?
    \begin{tasks}(4)
      \task \ce{HClO3}
      \task \ce{H2SO4}
      \task \ce{Na2SO4}
      \task \ce{FeSO4.7H2O}
    \end{tasks}
    \item 在下列各种溶液里取用溶质 \qty{1}{g},各需溶液多少毫升?
    \begin{tasks}(2)
      \task $0.1M$ \ce{H2SO4} 溶液
      \task $3M$   \ce{KOH} 溶液
      \task $0.2M$ \ce{BaCl2} 溶液
      \task $0.5M$ \ce{Na2SO4} 溶液
    \end{tasks}
    \item 下列说法是否正确,说明理由。
    \begin{tasks}
      \task 体积相同、摩尔浓度相同的任何物质的溶液含有相同的分子数。
      \task \qty{10}{mL} $1M$ 硫酸溶液比 \qty{100}{mL} $1M$ 硫酸溶液的浓度小。
      \task \qty{100}{mL} $0.1M$ 硫酸溶液和 \qty{50}{mL} $1M$ 硫酸溶液分别跟\qty{10}{mL} $1M$ \ce{BaCl2} 溶液起反应,前者生成的 \ce{BaSO4} 沉淀多。 
    \end{tasks}
    \item 中和 \qty{4}{g} 氢氧化钠,用去盐酸 \qty{25}{mL},计算这种盐酸的摩尔浓度。
    \item 某种待测浓度的 \ce{NaOH} 溶液 \qty{25}{mL},加入 \qty{20}{mL} $1M$ 的 \ce{H2SO4} 溶液后已显酸性,再滴入 $1M$ \ce{KOH} 溶液 \qty{1.5}{mL} 才达到中和。计算待测浓度的 \ce{NaOH} 溶液的摩尔浓度。
    \item 37\% 的盐酸(密度 \qty{1.19}{g/cm^3})相当于多少摩尔浓度的盐酸?
    \item 实验室常用的 65\% 浓硝酸,密度为 \qty{1.4}{g/cm^3},计算它的摩尔浓度。要配制 $3M$ 的硝酸 \qty{100}{mL},需用这种浓硝酸多少毫升?
  \end{question}
\end{Practice}

\section{反应热}
\subsection{热化学方程式}
化学反应都伴随着能量的变化,通常表现为热量的变化,即有放热或吸热的现象发生。
反应过程中放出或吸收的热都属于反应热。
远古时代,人类的祖先守着一堆篝火,烘烤食物,寒夜取暖,这就是利用燃烧放出的热。
到了近代; 利用化学反应的热能的规模日益扩大了。
煤炭、石油、天然气等能源不断开发出来,作为燃料和动力,用来开动火车、汽车、飞机、 拖拉机、联合收割机,开动工厂里的各种机器,并供日常生活中做饭、取暖之用。
现代,这些能源正在以更大的规模被利用着。
总而言之,化学反应放出的热能对我们是极为重要的。

化学反应里有原子和原子的重新结合,反应过程里放出或吸收的热量都和原子跟原子的分离和结合联系着。
例如,碳跟氧气的反应里,碳原子跟氧分子里的氧原子结合就会导致热量的产生。
人们通常应用摩尔这个物质的量的单位来计算可称量物质在反应过程里放出或吸收的热量。
这是把微粒跟可测量的热量联系起来的一个例子。
通过实验,测得 \qty{1}{mol} 碳(\num{6.02e23} 个碳原子)跟 \qty{1}{mol} 氧气(\num{6.02e23} 个氧分子)起反应,生成 \qty{1}{mol} 二氧化碳(\num{6.02e23} 个 \ce{CO2} 分子),放出 \qty{94}{kCal}\footnote{按照国际单位制,热量的单位是焦耳(\unit{J})。在这里暂时仍沿用卡(\unit{Cal})作为热量单位。$\qty{1}{Cal}=\qty{4.184}{J}$。所列反应放出或吸收的热量的数据一般是指在 \qty{1}{atm} 和 \qty{25}{\celsius} 的条件下测得的热量。以下同。}的热。\qty{2}{mol} 氢气跟 \qty{1}{mol} 氧气起反应,生成 \qty{2}{mol} 水蒸气,放出 \qty{115.6}{kCal} 的热。
\[ \ce{C}\,\text{(固)} + \ce{O2}\,\text{(气)} \xlongequal{\quad} \ce{CO2}\,\text{(气)} +\qty{94}{kCal}\]
\[ \ce{2H2}\,\text{(气)} + \ce{O2}\,\text{(气)} \xlongequal{\quad} \ce{2H2O}\,\text{(气)} +\qty{115.6}{kCal}\]

上面的反应是放热的,也有一些反应是吸热的。
当水蒸气跟灼热的碳接触时,发生的反应就要吸收热量。
\[ \ce{C}\,\text{(固)} + \ce{H2O}\,\text{(气)} \xlongequal{\triangle} \ce{CO}\,\text{(气)} + \ce{H2O}\,\text{(气)} -\qty{31.4}{kCal}\]

\qty{1}{mol} 的碳跟 \qty{1}{mol} 的水蒸气起反应,吸收 \qty{31.4}{kCal} 的热量。

在化学方程式里,为什么要在物质的右边注明固、液、气等状态呢?
我们知道,物质呈现哪一种聚集状态是跟它们含有的能量有关的。
为了精确起见,要注明反应物和生成物的状态才能确定放出或吸收的热量多少。例如:
\begin{gather*}
  \ce{2H2}\,\text{(气)} + \ce{O2}\,\text{(气)} \xlongequal{\quad} \ce{2H2O}\,\text{(气)} +\qty{115.6}{kCal} \\
  \ce{2H2}\,\text{(气)} + \ce{O2}\,\text{(气)} \xlongequal{\quad} \ce{2H2O}\,\text{(液)} +\qty{136.6}{kCal} 
\end{gather*}

放出的热量用“$+$”号表示,吸收的热量用 “$-$” 号表示。
\emph{这种表明反应所放出或吸收的热量的化学方程式叫做}\Concept{热化学方程式}。

应用热化学方程式可以计算生产上出现的热量的变化。
例如,可以计算甲烷燃烧所放出的热量。
在生产上,很注意反应所放出的热量的充分利用。
\begin{example}
  \qty{1}{mol} 甲烷燃烧时,生成液态水和二氧化碳,同时放出 \qty{212.8}{kCal} 的热。计算燃烧 \qty{1000}{L}(标准状况)甲烷所产生的热量。
\end{example}
\begin{solution}
  \qty{1000}{L} 甲烷燃烧所产生的热量
  \[ = \frac{\qty{1000}{L}}{\qty{22.4}{L}} \times \qty{212.8}{kCal} = \qty{9.50e3}{kCal}\]
  答:\qty{1000}{L}(标准状况)甲烷燃烧所放出的热是 \qty{9.50e3}{kCal}。
\end{solution}

\subsection{燃烧热}
由于反应的情况不同,反应热可以分为许多种,如燃烧热、中和热等。
在这里,只介绍燃烧热。

许多单质或化合物在燃烧时放出热量,生成稳定的物质,如二氧化碳,水、氯化氢等等。

\emph{\qty{1}{mol} 物质完全燃烧时所放出的热量,叫做该物质的}\Concept{燃烧热}。
燃烧热通常可由实验测得。
例如,测得 \qty{1}{mol} 碳完全燃烧放出的热量是 \qty{94}{kCal},这就是碳的燃烧热。
\[ \ce{C}\,\text{(固)} + \ce{O2}\,\text{(气)} \xlongequal{\quad} \ce{CO2}\,\text{(气)} +\qty{94}{kCal}\]

由实验测得,\qty{1}{mol} 氢气燃烧而生成液态水,放出的热量是 \qty{68.3}{kCal},\qty{68.3}{kCal} 就是氢气的燃烧热。

\[ \ce{H2}\,\text{(气)} + \frac{1}{2}\ce{O2}\,\text{(气)} \xlongequal{\quad} \ce{H2O}\,\text{(液)} +\qty{68.3}{kCal}\]

在计算燃烧热时,可燃物质是以 \qty{1}{mol} 作为标准来计算的,所以热化学方程式里的元素符号或分子式前面的系数可以用分数表示。

由实验测得,\qty{1}{mol} 气态一氧化碳燃烧而生成气态二氧化碳时,放出 \qty{67.6}{kCal} 的热。

\[ \ce{CO}\,\text{(气)} + \frac{1}{2}\ce{O2}\,\text{(气)} \xlongequal{\quad} \ce{CO2}\,\text{(气)} +\qty{67.6}{kCal}\]

一氧化碳的燃烧热就是 \qty{67.6}{kCal}。

学习反应热概念,能够帮助我们理解反应中的热量变化,也是开始通过能量变化来了解物质性质及其反应过程。

\begin{Practice}[习题]
  \begin{question}
    \item 在足量氧气里燃烧 \qty{2.5}{mol} 的碳,生成二氧化碳,能放出多少千卡的热量?
    \item 要燃烧多少摩尔的氢气,生成液态水. 才能得到 \qty{1000}{kCal} 的热量?
    \item 比较燃烧氢气(生成液态水)和碳(生成二氧化碳)各 \qty{1}{kg} 所放出的热量?
    \item 燃烧 \qty{1}{g} 甲烷(\ce{CH4},气体),生成液态水和二氧化碳能放出 \qty{13.3}{kCal} 的热量,计算燃烧 \qty{5}{mol} 甲烷能放出多少热量。
    \item 燃烧 \qty{1}{g} 乙炔(\ce{C2H2},气体),生成液态水和二氧化碳能放出 \qty{11.9}{kCal} 的热量,计算燃烧 \qty{3}{mol} 乙炔所放出的热量。燃烧摩尔数相同的甲烷和乙炔,哪种气体放出的热量多?
    \item 燃烧 \qty{0.11}{g} 酒精(\ce{C2H4OH},液体),生成液态水和二氧化碳,放出的热量能使 \qty{100}{g} 水升高温度 \qty{7.12}{\celsius},计算燃烧 \qty{1}{mol} 酒精时放出的热量。(水的比热容为 \qty{1}{Cal/(g.\celsius)})
    \item 根据下列数据分别算出镁、铝、硫的燃烧热,并写出热化学方程式。
    \begin{tasks}
      \task \qty{1}{g} 镁完全燃烧,放出 \qty{6.0 }{kCal} 的热。
      \task \qty{3}{g} 铝完全燃烧,放出 \qty{21.4}{kCal} 的热。
      \task \qty{5}{g} 硫完全燃烧,生成二氧化硫,放出 \qty{11.1}{kCal} 的热。
    \end{tasks}
  \end{question}
\end{Practice}

\section*{内容提要}
\setcounter{subsection}{0}
\subsection{摩尔}
摩尔是表示物质的量的单位,每摩尔物质含有阿伏伽德罗常数个微粒(分子、原子、离子等)。


\subsection{反应热}
\begin{enumerate}
  \item 热化学方程式: 表明反应所放出或吸收的热量的化学方程式。
  \item 燃烧热: \qty{1}{mol} 物质在完全燃烧时所放出的热量。燃烧热热量的计算以 \qty{1}{mol} 反应物(可燃物)为单位。
\end{enumerate}

\begin{Review}
  \begin{question}
    \item 下列说法是否正确?并说明理由。
    \begin{tasks}
      \task 同温同压下,相同质量的气体都占有相同的体积。
      \task 摩尔浓度是指 \qty{1}{L} 水里所含溶质的摩尔数。
      \task \qty{1}{mol} \ce{NaCl} 在水里电离后,可以得到 \qty{0.5}{mol} \ce{Na+} 和 \qty{0.5}{mol} \ce{Cl-}。
    \end{tasks}
    \item 在标准状况时,有 \qty{11}{g} 二氧化碳、\qty{0.5}{mol} 氢气、\qty{10}{L} 氮气。根据上述情况,回答下列问题。
    \begin{tasks}
      \task 哪一种物质的质量最大,哪一种最小?
      \task 哪一种物质所含分子数最多,哪一种最少?
      \task 哪一种物质所占体积最大,哪一种最小?
    \end{tasks}
    \item \qty{2}{g} 硫铵肥料跟浓碱液混和加热,收集到 \qty{600}{mL} 氨(标准状况)。计算肥料含氮元素的百分比。
    \item 浓度为 15\%、密度为 \qty{1.2}{g/cm^3} 的废硫酸 \qty{250}{mL}(不含铁化合物或其它酸)跟过量的铁屑充分反应, 计算:
    \begin{tasks}
      \task 这种废硫酸的摩尔浓度。
      \task 制得氢气(标准状况)的体积。
      \task 把生成的硫酸亚铁配制成 \qty{400}{mL} 溶液,这溶液的摩尔浓度是多少。
    \end{tasks}
    \item 在含有硫酸钠和碳酸钠的溶液里,加入足量的氯化钡溶液,生成沉淀 \qty{3.5}{g}。把沉淀另用足量的硝酸溶液处理,放出 \qty{150}{mL} 二氧化碳气体(标准状况)。计算溶液里含硫酸钠和碳酸钠各多少摩尔。
    \item 计算燃烧多少克氢气,生成液态水放出的热量跟燃烧 \qty{1}{kg} 碳,生成二氧化碳所放出的热量相等。
    \item 称取 \qty{1.721}{g} 某种硫酸钙的结晶水合物 $\ce{CaSO4 .x H2O}$, 加热,使它失去全部结晶水。这时候硫酸钙的质量是 \qty{1.721}{g},计算所含结晶水数($x$)。
    \item 已知 \ce{KCl} 在 \qty{24}{\celsius} 时的溶解度是 \qty{33.2}{g},计算 \qty{24}{\celsius} 时 \ce{KCl} 饱和溶液的百分比浓度。这样浓度的 \ce{KCl} 溶液密度为 \qty{1.16}{g/cm^3},计算它的摩尔浓度。
  \end{question}
\end{Review}