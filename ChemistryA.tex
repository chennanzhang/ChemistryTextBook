\documentclass[colortheme=magenta,txconfig=txchemistry.cfg]{textbook}
\stylesetup{ 
  % fullwidth-stop = catcode,
  boldemph = false,
}
\Booksetup{
  BookSeries  = 中学经典教材丛书, 
  BookTitle   = 高中化学(甲种本),
  BookTitle*  = {Textbook for Middle School Chemistry},
  SubTitle    = 第一册,
  SubTitle*   = Volume I,
  BriefIntro    = 
    { 
      本书供六年制中学高中一年级选用, 每周授课 3 课时。本书是在中小学通用教材化学编写组编的《全日制十年制学校初中课本(试用本)化学》卤素和碱金属一章和《全日制十年制学校高中课本(试用本)化学》第一册硫、硫酸、摩尔反应热和物质结构、元素周期律部分内容的基础上, 吸收了几年来各地在试用中的一些教学经验和意见编写成的。参加本书编写工作的有许国培、程名荣、张健如、胡美玲、王存志等。北京师范大学化学系的何少华也参加了编写工作。责任编辑是许国培, 审定者是武永兴、梁英豪。希望广大教师和研究中学化学教学的同志提出批评和修改意见。
    },
  DedicatedTo   = 奔赴高考的莘莘学子,
  % CoverGraph    = graphics/A.pdf,
  AuthorList    = {人民教育出版社化学室},
  ReleaseDate   = 2024-12-12,
  % Url           = https://www.tjad.cn,
  % ISBN          = 978-7-302-11622-6,
  % Publisher     = 同济极客出版社,
  % Logo          = graphics/logo.pdf,
  % Editor        = {张晨南},
  WrittenStyle  = 编,
}
\graphicspath{{figures/A/}}
\begin{document}
\frontmatter
\tableofcontents
\mainmatter
\chapter{摩尔}\label{chp:mol}
摩尔是国际单位制的一种基本单位,它表示物质的量。
摩尔广泛地应用于科学研究、工农业生产等等方面。
在中学化学里,摩尔应用于计算微粒的数量、物质的质量、气体的体积、 溶液的浓度、反应过程的热量变化等等。

我们要重视摩尔的学习,理解摩尔的意义,学会使用摩尔这个基本单位的方法,并在以后各章的学习里不断应用。

\section{摩尔}
\subsection{摩尔}
我们在初中化学里,学习过原子、分子、离子等构成物质的微粒,知道单个这样的微粒是肉眼看不见的,也是难于称量的。
但是,在实验室里取用的物质,不论是单质还是化合物,应是看得见的、可以称量的。
生产上,物质的用量当然更大,常以吨计。
物质之间的反应,既是按照一定个数、肉眼看不见的原子、分子或离子来进行,而实践上又是以可称量的物质进行反应。
所以,很需要把微粒跟可称量的物质联系起来。

怎样联系起来呢? 就是要建立一种物质的量的基本单位,这个单位是含有同数的原子、分子、离子等等的集体。
科学上,已经建立把微粒跟微粒集体联系起来的单位。
那么,采取多大的集体作为物质的量的单位呢?

近年来,科学上应用 \qty{12}{g} \ce{^12C}(或 \qty{0.012}{kg} \ce{^12C})来衡量碳原子集体。\ce{^12C} 就是原子核里有 6 个质子和 6 个中子的碳原子。根据实验测定,\qty{12}{g} \ce{^12C} 含有的原子数就是阿伏伽德罗\footnote{阿伏伽德罗(Avogadro 1776--1856)是意大利物理学家。}常数。
阿伏伽德罗常数经过实验已测得比较精确的数值。
在这里,采用 \num{6.02e23} 这个非常近似的数值。

\emph{摩尔是表示物质的量的单位,每摩尔物质含有阿伏伽德罗常数个微粒}。例如:
\begin{itemize}[label={},labelsep=0pt]
  \item \qty{1}{mol}\footnote{摩尔可以简称为摩,符号 \unit{mol}。}的碳原子含有 \num{6.02e23} 个碳原子,
  \item \qty{1}{mol} 的氢原子含有 \num{6.02e23} 个氢原子,
  \item \qty{1}{mol} 的氧分子含有 \num{6.02e23} 个氧分子,
  \item \qty{1}{mol} 的水分子含有 \num{6.02e23} 个水分子,
  \item \qty{1}{mol} 的二氧化碳分子含有 \num{6.02e23} 个二氧化碳分子,
  \item \qty{1}{mol} 的氢离子含有 \num{6.02e23} 个氢离子,
  \item \qty{1}{mol} 的氢氧根离子含有 \num{6.02e23} 个氢氧根离子。
\end{itemize}

阿伏伽德罗常数是很大的数值,但摩尔作为物质的量的单位应用极为方便。
因为实验测得 \qty{1}{mol} \ce{^12C} 的质量是 \qty{12}{g},即含有 \num{6.02e23} 个碳原子的质量。
由此我们可以推算 \qty{1}{mol} 任何原子的质量。

一种元素的原子量是以 \ce{^12C} 的质量的 $1/12$ 作为标准,其它元素原子的质量跟它相比较所得的数值,如氧的原子量是 16,氢的原子量是 1,铁的原子量是 55.85,等等。
1 个碳原子的质量跟 1 个氧原子的质量之比是 $12:16$。
\qty{1}{mol} 碳原子跟 \qty{1}{mol} 氧原子所含有的原子数相同,都是 \num{6.02e23}。
\qty{1}{mol} 碳原子是 \qty{12}{g},那么 \qty{1}{mol} 氧原子就是 \qty{16}{g}。
同理,\qty{1}{mol} 任何原子的质量就是以克为单位,数值上等于该种原子的原子量。
由此我们可以直接推知:
\begin{itemize}[label={},labelsep=0pt]
  \item 氢的原子量是 1,\qty{1}{mol} 氢原子的质量是 \qty{1}{g},
  \item 铁的原子量是 55.85,\qty{1}{mol} 铁原子的质量是 \qty{55.85}{g}。
\end{itemize}

其次,我们用摩尔来衡量双原子分子或多原子分子构成的各种物质的时候,那么同样地可以推知,\qty{1}{mol} 任何分子的质量,就是以克为单位,数值上等于该种分子的分子量。
\begin{itemize}[label={},labelsep=0pt]
  \item 氢气的分子量是 2,\qty{1}{mol} 氢气的质量是 \qty{2}{g},
  \item 氧气的分子量是 32,\qty{1}{mol} 氧气的质量是 \qty{32}{g},
  \item 二氧化碳的分子量是 44,\qty{1}{mol} 二氧化碳的质量是 \qty{44}{g},
  \item 水的分子量是 18,\qty{1}{mol} 水的质量是 \qty{18}{g}。
\end{itemize}

当摩尔应用于表示离子的时候,同样可以推知 \qty{1}{mol} 离子的质量。
由于电子的质量过于微小,失去或得到的电子的质量可以略去不计。
\begin{itemize}[label={},labelsep=0pt]
  \item \qty{1}{mol} \ce{H+} 的质量是 \qty{1}{g},
  \item \qty{1}{mol} \ce{OH-} 的质量是 \qty{17}{g},
  \item \qty{1}{mol} \ce{Cl-} 的质量是 \qty{35.5}{g}。
\end{itemize}

对于离子化合物也可以同样推知,如 \qty{1}{mol} \ce{NaCl} 的质量是 \qty{58.5}{g}。

总之,摩尔象一座桥梁把单个的、肉眼看不见的微粒跟很大数量的微粒集体、可称量的物质之间联系起来了。

应用摩尔来衡量物质的量,在科学技术上带来了方便。
如从化学反应中反应物和生成物之间的原子、分子等微粒的比值,可以直接知道它们之间摩尔的数目之比,
\begin{align*}
\ce{\underset{\qty{1}{mol}}{\ce{C_{\phantom{2}}}} + \underset{\qty{1}{mol}}{\ce{O2}} & \xlongequal{\quad} \underset{\qty{1}{mol}}{\ce{CO2}}}\\
\ce{\underset{\qty{1}{mol}}{\ce{Mg}} + \underset{\qty{2}{mol}}{\ce{2HCl_{\phantom{1}}}} & \xlongequal{\quad} \underset{\qty{1}{mol}}{\ce{MgCl2}} + \underset{\qty{1}{mol}}{\ce{H2}} ^} 
\end{align*}

\subsection{关于摩尔质量的计算}
\qty{1}{mol} 物质的质量通常也叫做该物质的\Concept{摩尔质量}。摩尔质量的单位是“克/摩尔(\unit{g/mol})”。
物质的量、物质的质量和摩尔质量之间的关系可以用下式表示:
\[ \frac{\text{物质的质量}(\unit{g})}{\text{摩尔质量}(\unit{g/mol})}=\text{物质的量}(\unit{mol})\]

\begin{example}
  \qty{90}{g} 水相当于多少摩尔水分子?
\end{example}
\begin{solution}
  水的分子量是 18,水的摩尔质量是 \qty{18}{g/mol}。
  \[ \frac{\qty{90}{g}}{\qty{18}{g/mol}}=\qty{5}{mol}\]
  答:\qty{90}{g} 水相当于 \qty{5}{mol} 水,也可以说 \qty{90}{g} 水所含的摩尔数是 5 。
\end{solution}

\begin{example}
  \qty{2.5}{mol} 铜原子的质量是多少克?
\end{example}
\begin{solution}
  铜的原子量是 63.5,铜的摩尔质量是 \qty{63.5}{g/mol}。
  \[ \qty{2.5}{mol}\ \text{铜的质量}= \qty{63.5}{g/mol} \times \qty{2.5}{mol} = \qty{158.8}{g} \]
  答:\qty{2.5}{mol} 铜原子(或简称 \qty{2.5}{mol} 铜)的质量等于 \qty{158.8}{g}。
\end{solution}

\begin{example}
  \qty{4.9}{g} 硫酸里含有多少硫酸分子?
\end{example}
\begin{solution}
  硫酸的分子量是 98,硫酸的摩尔质量是 \qty{98}{g/mol}。
  \[\begin{split} 
  \frac{\qty{4.9}{g}}{\qty{98}{g/mol}} & = \qty{0.05}{mol}\\
  \qty{4.9}{g}\ \text{硫酸的分子数}    & = \qty{6.02e23}{mol^{-1}} \times \qty{0.05}{mol}\\
   & = \num{3.01e22} \\
  \end{split}\]
  答:\qty{4.9}{g} 克硫酸里含有 \num{3.01e22} 个分子。
\end{solution}

\begin{Practice}[习题]
  \begin{question}
    \item 2 个氧分子、\qty{2}{g} 氧气、\qty{2}{mol} 氧分子有什么区别?
    \item 选择正确的答案填写在括号里。
    
    \noindent\qty{0.5}{mol} 氢气含有(\hspace{2em})
    \begin{tasks}(3)
      \task 0.5 个氢分子
      \task 1 个氢原子
      \task \num{6.02e23} 个氢原子
      \task \num{3.01e23} 个氢分子
      \task \num{3.01e12} 个氢分子
    \end{tasks}
    \item 计算 \qty{1}{mol} 下列物质的质量。
    \begin{tasks}(2)
      \task 氦、镁、氯原子、磷原子。
      \task 硝酸、硝酸铵、蔗糖(\ce{C12H22O11})
    \end{tasks}
    \item 下列物质的量各等于多少摩尔。
    \begin{tasks}
      \task \qty{1}{kg}硫原子,\qty{0.5}{kg}铝原子,\qty{0.25}{kg} 锌原子
      \task \qty{22}{g} 二氧化碳,\qty{500}{g} 氯化钠,\qty{1.5}{kg} 蔗糖
    \end{tasks}
    \item 分别列出铝、铁、铅的摩尔质量。根据 \qty{20}{\celsius} 时,铝、 铁、铅 的密度\footnote{按照国际单位制,密度的单位应是千克/米$^3$,在这里,暂按克/厘米$^3$(\unit{g/cm^3})或克/升(\unit{g/L})为单位。}分别是 \qtylist{2.70;7.86;11.3}{g/cm^3},计算 \qty{1}{mol} 铝、铁、铅的体积。
    \item 在 \qty{15}{\celsius} 时,蔗糖的密度是 \qty{1.588}{g/cm^3},计算 \qty{1}{mol} 蔗糖的体积。
    \item 分解氯酸钾制氧气的时候,制 \qty{0.6}{mol} 氧气需要多少摩尔的氯酸钾?
    \item 跟含 \qty{4}{g} 氢氧化钠的溶液起反应使生成正盐,需用下列酸各多少摩尔。
    \begin{tasks}(5)
      \task \ce{HCl}
      \task \ce{HNO3}
      \task \ce{H2SO4}
      \task \ce{H3PO4}
      \task \ce{HClO3}
    \end{tasks}
    \item 硫酸铵、硝酸铵、磷酸氢二铵(\ce{(NH4)2HPO4})、尿素都可以作为氮肥。试计算:
    \begin{tasks}
      \task \qty{1}{mol} 上述物质的质量各是多少克,
      \task \qty{1}{mol} 上述物质的质量各含多少摩尔氮原子。
    \end{tasks}
  \end{question}
\end{Practice}

\section{气体摩尔体积}
\subsection{气体摩尔体积}
对于固态或液态的物质来说,\qty{1}{mol} 各种物质的体积是不相同的。
例如,\qty{20}{\celsius} 时,\qty{1}{mol} 铁的体积是 \qty{7.1}{cm^3},\qty{1}{mol} 铝的体积是 \qty{10}{cm^3},\qty{1}{mol} 铅的体积是 \qty{18.3}{cm^3}(\cref{fig:1-1});\qty{1}{mol} 水的体积是 \qty{18.0}{cm^3}, \qty{1}{mol} 纯硫酸的体积是 \qty{54.1}{cm^3},\qty{1}{mol} 蔗糖的体积是 \qty{215.5}{cm^3}(图 1-2)。
\begin{figure}
  \includegraphics{1-1.pdf}
  \caption{\qty{1}{mol}的几种金属}\label{fig:1-1}
  \caption{\qty{1}{mol}的几种化合物}\label{fig:1-2}
\end{figure}

\qty{1}{mol} 固态或液态的物质的体积为什么不同呢? 
这因为对固态或液态的物质来说,构成它们的微粒间的距离是很小的,\qty{1}{mol} 物质的体积主要决定于原子、分子或离子的大小。
构成不同物质的原子、分子或离子的大小是不同的,所以它们 \qty{1}{mol} 的体积也就有所不同。

但是,对气体来说,情况就大不相同。

我们分别计算 \qty{1}{mol} 氢气、氧气和二氧化碳在标准状况\footnote{标准状况是指压强为 \qty{1}{atm} 和温度为 \qty{0}{\celsius}。根据国际单位制压强单位是帕(\unit{Pa})。在这里暂用标准大气压(\unit{atm})。$\qty{1}{atm}=\qty{101325}{Pa}$} 时的体积。
氢气的摩尔质量是 \qty{2}{g/mol},氧气的摩尔质量是 \qty{32}{g/mol},二氧化碳的摩尔质量是 \qty{44}{g/mol},同时它们的密度分别是 \qtylist{0.0899;1.429;1.977}{g/L}。
这样就可以算出上述气体在标准状况时所占的体积。
\begin{align*}
  \text{氢气的摩尔体积}&=\frac{\qty{2.016}{g/mol}}{\qty{0.0899}{g/L}}=\qty{22.4}{L/mol}\\
  \text{氧气的摩尔体积}&=\frac{\qty{32.0}{g/mol}}{\qty{1.429}{g/L}}=\qty{22.4}{L/mol}\\
  \text{二氧化碳的摩尔体积}&=\frac{\qty{44.0}{g/mol}}{\qty{1.977}{g/L}}=\qty{22.3}{L/mol}
\end{align*}

从上面几个例子可以看出,在标准状况时,\qty{1}{mol} 三种气体的体积都约是 \qty{22.4}{L}。
而且经过许多实验发现和证实,\qty{1}{mol} 的任何气体在标准状况下所占的体积都约是 \qty{22.4}{L}(\cref{fig:1-3})。
\begin{figure}
  \begin{minipage}{0.48\linewidth}\centering
    % \includegraphics{1-3.pdf}
    \caption{气体摩尔体积}\label{fig:1-3}
  \end{minipage}
  \begin{minipage}{0.48\linewidth}\centering
    % \includegraphics{1-4.pdf}
    \caption{气体分子的运动和距离}\label{fig:1-4}
  \end{minipage}
\end{figure}

\emph{在标准状况下,\qty{1}{mol} 的任何气体所占的体积都约是 \qty{22.4}{L},这个体积叫做}\Concept{气体摩尔体积}。

为什么 \qty{1}{mol} 的固体、液体的体积各不相同,而 \qty{1}{mol} 气体在标准状况时所占的体积都相同呢? 
这要从气态物质的结构去找原因。
气体的分子在较大的空间里迅速地运动着(\cref{fig:1-4})。
在通常情况下气态物质的体积要比它在液态或固态时大 1000 倍左右,这是因为气体分子间有着较大的距离。
通常情况下一般气体的分子直径约是 \qty{4e-10}{m},分子间的平均距离约是 \qty{4e-9}{m},即平均距离是分子直径的 10 倍左右(\cref{fig:1-5})。
这就可以推知,气体体积主要决定于分子间的平均距离,而不象液体或固体那样,体积主要决定于分子的大小。
在标准状况下,不同气体分子间的平均距离几乎是相等的,所以任何物质的气体摩尔体积都约是 \qty{22.4}{L/mol}。
\begin{figure}
  % \includegraphics{1-5.pdf}
  \caption{固体、液体跟气体的分子间距离比较示意图(以碘为例)}\label{fig:1-5}
\end{figure}

气体摩尔体积约是 \qty{22.4}{L/mol},为什么一定要加上标准状况这个条件呢: 这是因为气体的体积较大地受到温度和压强的影响。
温度升高时,气体分子间的平均距离增大,温度降低时平均距离减小;压强增大时,气体分子间的平均距离减小,压强减小时,平均距离增大。
各种气体在一定温度和压强下,分子间的平均距离是相等的。
在一定的温度和压强下,气体体积的大小只随分子数的多少而变化,相同的体积含有相同的分子数。
这是经过生产上和科学实验的许多事实所证明的。

在相同的温度和压强下,相同体积的任何气体都含有相同数目的分子,这就是阿伏伽德罗定律。

\begin{Theorem}{讨论}
如果已经知道水的分子式是 \ce{H2O},我们能够根据氢气跟氧气化合成水蒸气的体积比是 $2:1:2$,应用阿伏伽德罗定律来证明 1 个氧分子里含有 2 个氧原子吗?
\end{Theorem}

\subsection{关于气体摩尔体积的计算}
\begin{example}
  \qty{5.5}{g} 氨相当于多少摩尔氨,在标准状况时它的体积应是多少升?
\end{example}
\begin{solution}
  氨的分子量是 17,氨的摩尔质量是 \qty{17}{g/mol}。
  \[\begin{split} 
    \frac{\qty{5.5}{g}}{\qty{17}{g/mol}} &= \qty{0.32}{mol} \\
    \qty{5.5}{g}\ \text{氨的体积} & = \qty{22.4}{L/mol} \times \qty{0.32}{mol} =\qty{7.2}{L}
  \end{split}\]
  答:\qty{5.5}{g} 氨相当于 \qty{0.32}{mol} 的氨,在标准状况时,它的体积是 \qty{7.2}{L}。
\end{solution}

\begin{example}
  在实验室里使稀盐酸跟锌起反应,在标准状况时生成 \qty{3.36}{L} 氢气。计算需要多少摩尔的 \ce{HCL} 和锌。
\end{example}
\begin{solution}
  设 $x$ 为所需多少摩尔的锌,$y$ 为所需多少摩尔的 \ce{HCl}。
  \begin{gather*} 
    \ce{\underset{\qty{1}{mol}}{\ce{Zn}} + \underset{\qty{2}{mol}}{\ce{HCl}} \xlongequal{\quad} ZnCl2 + \underset{\qty{22.4}{L}}{\ce{H2}} ^} \\
   \scriptstyle x\quad\quad\ y \qquad\qquad\ \ \qquad \qty{3.36}{L}
  \end{gather*}
  \[\begin{split} 
    x & = \frac{\qty{1}{mol}\times\qty{3.36}{L}}{\qty{22.4}{L}} = \qty{0.15}{mol}\\
    y & = \frac{\qty{2}{mol}\times\qty{3.36}{L}}{\qty{22.4}{L}} = \qty{0.30}{mol}\\
  \end{split}\]
  答:需 \qty{0.15}{mol} 锌和 \qty{0.30}{mol} \ce{HCl}。
\end{solution}

\begin{example}
  在标准状况时,\qty{0.20}{L} 的容器里所含一氧化碳的质量为 \qty{0.25}{g},计算一氧化碳的分子量。
\end{example}

根据摩尔体积可以计算出一氧化碳的摩尔质量,而摩尔质量的数值就等于它的分子量。

\begin{solution}
  \[\begin{split}
    \text{一氧化碳的摩尔质量} & = \text{一氧化碳的密度} \times \text{一氧化碳的摩尔体积}\\
    & = \frac{\qty{0.25}{g}}{\qty{0.20}{L}} \times \qty{22.4}{L} = \qty{28}{g/mol}\\ 
    \text{一氧化碳的分子量} & = 28
  \end{split}\]
  答:一氧化碳的分子量是 28。
\end{solution}

\begin{Practice}[习题]
  \begin{question}
    \item 改正下列说法里可能有的错误,并说明理由。
    \begin{tasks}
      \task \qty{1}{mol} 任何气体的体积都是 \qty{22.4}{L}。
      \task \qty{1}{mol} 氢气的质量是 \qty{1}{g},它所占的体积是 \qty{22.4}{L}。
      \task \qty{1}{mol} 任何物质在标准状况时所占的体积都约是 \qty{22.4}{L}。
      \task \qty{1}{mol} 氢气和 \qty{1}{mol} 水所含的分子数相同,在标准状况时所占体积都约是 \qty{22.4}{L}。
    \end{tasks}
    \item 在标准状况时,\qty{1}{L} 氮气约含有多少个氮分子?
    \item 在标准状况时,\qty{15}{g} 氟气所占的体积比 \qty{1}{g} 氢气所占的体积是大还是小?
    \item 在标准状况时,\qty{4.4}{g} 二氧化碳的体积跟多少克氧气的体积相等?
    \item 在实验室制备氢气的时候,用 \qty{0.1}{mol} 的锌跟足量稀盐酸起反应,计算所产生的氢气的体积(在标准状况)。
    \item 氮气在标准状况时的密度是 \qty{1.25}{g/L},液态氮在 \qty{-195.8}{\celsius} 的密度是 \qty{0.808}{g/cm^3},固态氮在  \qty{-232.5}{\celsius} 的密度是 \qty{1.026}{g/cm^3},比较 \qty{1}{mol} 的氮在气态、液态、固态各占多少体积。
    \item 在标准状况时,\qty{235}{mL} 某种气体的质量是 \qty{0.406}{g},计算这种气体的分子量。
  \end{question}
\end{Practice}

\section{摩尔浓度}
\subsection{摩尔浓度}
我们在初中化学里学习过百分比浓度,应用这种表示溶液浓度的方法,可以了解和计算一定质量的溶液中所含溶质的质量。
但是,我们在许多场合取用溶液时,一般不是去称它的质量而是量它的体积。
同时,物质起反应时,反应物和生成物各是多少摩尔相互之间有一定的关系; 知道一定体积溶液里含多少摩尔溶质,运算起来很便利。
因此,摩尔浓度是生产上和科学实验上常用的一种表示溶液浓度的重要方法。

\emph{以 \qty{1}{L} 溶液里含有多少摩尔溶质来表示的溶液浓度叫}\Concept{摩尔浓度}。摩尔浓度\footnote{按照国际单位制的规定,物质的量浓度的单位是摩尔/米$^3$(\unit{mol/m^3})。在这里仍暂按习惯沿用摩尔浓度,单位也用摩尔/升(\unit{mol/L})。}通常用 $M$ 表示。
\[
\text{摩尔浓度}(M) = \frac{\text{溶质的量}(\unit{mol})}{\text{溶液的体积}(\unit{L})}
\]

\qty{1}{L} 溶液中含 \qty{1}{mol} 的溶质, 这种溶液就是 \qty{1}{mol} 浓度的溶液,通常用 $1M$ 表示。如蔗糖的摩尔质量是 \qty{342}{g/mol}。
把 \qty{342}{g} 蔗糖溶解在适量水里配成 \qty{1}{L} 溶液,它的摩尔浓度就是 \qty{1}{mol/L} 或 $1M$。
\qty{1}{L} 溶液中含 \qty{171}{g} 蔗糖,它的摩尔浓度就是 $0.5M$。
又如 \qty{1}{L} 的氯化钠的质量是 \qty{58.5}{g} 克,把 \qty{58.5}{g} 氯化钠溶解在适量水里制成 \qty{1}{L} 溶液时, 它的摩尔浓度就是 $1M$。
\qty{1}{L} 溶液中含 \qty{29.3}{g} 氯化钠的溶液浓度就是 $0.5M$。

溶液用摩尔浓度表示时,常简称为摩尔溶液。

\begin{Experiment}*%[righthand ratio=0.3]
  在天平上称出 \qty{29.3}{g} 氯化钠,把称好的氯化钠放在烧杯里,用适量蒸馏水使它完全溶解。把制得的溶液小心地注入 \qty{1000}{mL} 的容量瓶(\cref{fig:1-6})。用蒸馏水洗涤烧杯内壁两次,把每次洗下来的水都注入容量瓶。振荡容量瓶里的溶液使混和均匀。然后缓缓地把蒸馏水直接注入容量瓶直到液面接近刻度 \qtyrange{2}{3}{cm} 处。改用胶头滴管加水到瓶颈刻度的地方,使溶液的凹面正好跟刻度相平。把容量瓶塞好,反复摇匀。
  \tcblower
  \begin{figurehere}
    \caption{配置摩尔浓度的溶液}\label{fig:1-6}
  \end{figurehere}
\end{Experiment}

这样配制成的溶液就是 $0.5M$ 的氯化钠溶液。

\subsection{在摩尔溶液中溶质微粒的数目}

\qty{1}{mol} 任何物质的微粒数都是 \num{6.02e23}。
\qty{1}{L} $1M$ 的蔗糖溶液含有 \num{6.02e23} 个蔗糖分子。
对于非电解质来说,体积相同的同摩尔浓度的溶液都应含有相同的溶质分子数。

但是,对于溶质为离子化合物或在水里完全电离的共价化合物等电解质来说,情况就比较复杂。
例如,氯化钠溶解在水里电离为 \ce{Na+} 和 \ce{Cl-}。
所以,\qty{1}{L} $1M$ 的 \ce{NaCl} 溶液含有 \num{6.02e23} 个 \ce{Na+} 和 \num{6.02e23} 个 \ce{Cl-}。
同样地,\qty{1}{L} $1M$ 的 \ce{NaOH} 溶液含有 \num{6.02e23} 个 \ce{Na+} 和 \num{6.02e23} 个 \ce{OH-};\qty{1}{L} $1M$ 的 \ce{CaCl2} 溶液含有 \num{6.02e23} 个 \ce{Ca^{2+}} 和 \numproduct{2x6.02e23} 个 \ce{Cl-}。
又例如 \qty{1}{L} $1M$ 的 \ce{HCl} 溶液里含有 \num{6.02e23} 个 \ce{H+} 和 \num{6.02e23} 个 \ce{Cl-}。

\subsection{关于摩尔浓度的计算}
\subsubsection{已知溶质的质量和溶液的体积,计算溶液的摩尔浓度}
\begin{example}
在 \qty{200}{mL} 稀盐酸里溶有 \qty{0.73}{g} \ce{HCl},计算溶液的摩尔浓度。
\end{example}
摩尔浓度所表示的就是 \qty{1}{L} 溶液里含多少摩尔溶质,在这题里,就是要算出 \qty{1}{L} 溶液里含多少摩尔的 \ce{HCl}。
\begin{solution}
\ce{HCl} 的分子量是 36.5,它的摩尔质量是 \qty{36.5}{g/mol}。\qty{0.73}{g} \ce{HCl} 相当于:
\[ \frac{\qty{0.73}{g}}{\qty{36.5}{g/mol}} = \qty{0.02}{mol}\]

\qty{1000}{mL} 溶液中含 \ce{HCl}
\[ \qty{0.02}{mol} \times \frac{\qty{1000}{mL}}{\qty{200}{mL}} = \qty{0.1}{mol}\]
答:这种稀盐酸的浓度是 $0.1M$。
\end{solution}

\subsubsection{已知溶液的摩尔浓度,计算一定体积的溶液里所含溶质的质量}
\begin{example}
  计算配制 \qty{500}{mL} $0.1M$ 的 \ce{NaOH} 溶液所需 \ce{NaOH} 的质量。
\end{example}
\begin{solution}
  \ce{NaOH} 的分子量是 40,它的摩尔质量是 \qty{40}{g/mol}。

  \qty{0.1}{mol} 的 \ce{NaOH} 的质量 $= \qty{40}{g/mol} \times \qty{0.1}{mol}=\qty{4}{g}$

  \qty{500}{mL} $0.1M$ \ce{NaOH} 溶液所含 \ce{NaOH} 的质量是:
  \[ \frac{\qty{500}{mL}\times\qty{4}{g}}{\qty{1000}{mL}} =\qty{2}{g}\]
  答: 制备 \qty{500}{mL} $0.1M$ 的 \ce{NaOH} 溶液需 \qty{2}{g} \ce{NaOH}。
\end{solution}

\subsubsection{应用摩尔浓度作关于浓溶液稀释的计算}
\begin{example}
  20\% 的蔗糖溶液 \qty{200}{g},加适量的水稀释到 \qty{1}{L},计算稀释后蔗糖溶液的摩尔浓度。
\end{example}
\begin{solution}
  蔗糖 \ce{C12H22O11} 的分子量是 342,蔗糖的摩尔质量是 \qty{342}{g/mol}。

  溶液里所含蔗糖的质量是 $\qty{200}{g} \times 20\% = \qty{40}{g}$
  \[\frac{\qty{40}{g}}{\qty{342}{g/mol}}=\qty{0.117}{mol}\]

  \qty{1}{L} 蔗糖溶液里含有 \qty{0.117}{mol} 蔗糖。

\noindent 答: 蔗糖溶液的摩尔浓度是 $0.117M$。
\end{solution}
\begin{example}
  计算配制 \qty{500}{mL} $1M$ 的硫酸溶液需要密度为 \qty{1.836}{g/cm^3} 的浓硫酸 (98\% \ce{H2SO4})多少毫升?
\end{example}

浓硫酸稀释后,所含 \ce{H2SO4} 的质量是不变的,因而在硫酸溶液里多少摩尔  \ce{H2SO4} 等于浓硫酸里多少摩尔的 \ce{H2SO4}。 
先算出硫酸溶液里含多少摩尔 \ce{H2SO4},再算出浓硫酸的摩尔浓度,从所含多少摩尔 \ce{H2SO4} 可以算出需要的浓硫酸的体积。
\begin{solution}
硫酸溶液中含 \ce{H2SO4}
\[ \frac{\qty{500}{mL}}{\qty{1000}{mL}}\times \qty{1}{mol} =\qty{0.5}{mol} \]

硫酸的摩尔质量 $=\qty{98}{g/mol}$

\qty{1000}{mL} 浓硫酸中 \ce{H2SO4} 的质量
\[ = \qty{1000}{mL} \times \qty{1.836}{mL} \times \frac{98}{100} = \qty{1799}{g}\]

\qty{1000}{mL} 浓硫酸里含 \ce{H2SO4}
\[ \frac{\qty{1799}{g}}{\qty{98}{g/mol}}=\qty{18.4}{mol} \]

含 \qty{0.5}{mol} \ce{H2SO4} 的浓硫酸的体积
\[ = \frac{\qty{0.5}{mol}}{\qty{18.4}{mol/L}}=\qty{0.0272}{L}=\qty{27.2}{mL} \]
需要浓硫酸 (98\% \ce{H2SO4})\qty{27.2}{mL}。
\end{solution}

\subsubsection{已知起反应的两种溶液的摩尔浓度以及其中一种溶液的体积,计算另一种溶液的体积}
\begin{example}
  中和 \qty{1}{L} $0.5M$ \ce{NaOH} 溶液,需要多少升的 $1M$ 的 \ce{H2SO4} 溶液?
\end{example}
\begin{solution}
  \[ \ce{\underset{\qty{2}{mol}}{\ce{2NaOH_$\phantom{2}$}} + \underset{\qty{1}{mol}}{\ce{H2SO4}} \xlongequal{\quad} Na2SO4 + 2H2O} \]

  \qty{1}{L} $0.5M$ 溶液中含 \ce{NaOH}
  \[ \qty{1}{L} \times \qty{0.5}{mol/L} = \qty{0.5}{mol}\]

中和 \qty{0.5}{mol} \ce{NaOH} 需 \ce{H2SO4}
\[ \qty{0.5}{mol}\times\frac{1}{2}=\qty{0.25}{mol} \]

含 \qty{0.25}{mol} 的 $1M$ \ce{H2SO4} 溶液的体积
\[= \frac{\qty{1}{L} \times \qty{0.25}{mol}}{\qty{1}{mol}} = \qty{0.25}{L}\]

答: 中和 \qty{1}{L} $0.5M$ \ce{NaOH} 溶液需 $1M$ 的 \ce{H2SO4} 溶液 \qty{0.25}{L}。
\end{solution}

\begin{Practice}[习题]
  \begin{question}
    \item 制备下列各物质的 $0.2M$ 溶液各 \qty{50}{mL},需用下列物质各多少克?
    \begin{tasks}(4)
      \task \ce{HClO3}
      \task \ce{H2SO4}
      \task \ce{Na2SO4}
      \task \ce{FeSO4.7H2O}
    \end{tasks}
    \item 在下列各种溶液里取用溶质 \qty{1}{g},各需溶液多少毫升?
    \begin{tasks}(2)
      \task $0.1M$ \ce{H2SO4} 溶液
      \task $3M$   \ce{KOH} 溶液
      \task $0.2M$ \ce{BaCl2} 溶液
      \task $0.5M$ \ce{Na2SO4} 溶液
    \end{tasks}
    \item 下列说法是否正确,说明理由。
    \begin{tasks}
      \task 体积相同、摩尔浓度相同的任何物质的溶液含有相同的分子数。
      \task \qty{10}{mL} $1M$ 硫酸溶液比 \qty{100}{mL} $1M$ 硫酸溶液的浓度小。
      \task \qty{100}{mL} $0.1M$ 硫酸溶液和 \qty{50}{mL} $1M$ 硫酸溶液分别跟\qty{10}{mL} $1M$ \ce{BaCl2} 溶液起反应,前者生成的 \ce{BaSO4} 沉淀多。 
    \end{tasks}
    \item 中和 \qty{4}{g} 氢氧化钠,用去盐酸 \qty{25}{mL},计算这种盐酸的摩尔浓度。
    \item 某种待测浓度的 \ce{NaOH} 溶液 \qty{25}{mL},加入 \qty{20}{mL} $1M$ 的 \ce{H2SO4} 溶液后已显酸性,再滴入 $1M$ \ce{KOH} 溶液 \qty{1.5}{mL} 才达到中和。计算待测浓度的 \ce{NaOH} 溶液的摩尔浓度。
    \item 37\% 的盐酸(密度 \qty{1.19}{g/cm^3})相当于多少摩尔浓度的盐酸?
    \item 实验室常用的 65\% 浓硝酸,密度为 \qty{1.4}{g/cm^3},计算它的摩尔浓度。要配制 $3M$ 的硝酸 \qty{100}{mL},需用这种浓硝酸多少毫升?
  \end{question}
\end{Practice}

\section{反应热}
\subsection{热化学方程式}
化学反应都伴随着能量的变化,通常表现为热量的变化,即有放热或吸热的现象发生。
反应过程中放出或吸收的热都属于反应热。
远古时代,人类的祖先守着一堆篝火,烘烤食物,寒夜取暖,这就是利用燃烧放出的热。
到了近代; 利用化学反应的热能的规模日益扩大了。
煤炭、石油、天然气等能源不断开发出来,作为燃料和动力,用来开动火车、汽车、飞机、 拖拉机、联合收割机,开动工厂里的各种机器,并供日常生活中做饭、取暖之用。
现代,这些能源正在以更大的规模被利用着。
总而言之,化学反应放出的热能对我们是极为重要的。

化学反应里有原子和原子的重新结合,反应过程里放出或吸收的热量都和原子跟原子的分离和结合联系着。
例如,碳跟氧气的反应里,碳原子跟氧分子里的氧原子结合就会导致热量的产生。
人们通常应用摩尔这个物质的量的单位来计算可称量物质在反应过程里放出或吸收的热量。
这是把微粒跟可测量的热量联系起来的一个例子。
通过实验,测得 \qty{1}{mol} 碳(\num{6.02e23} 个碳原子)跟 \qty{1}{mol} 氧气(\num{6.02e23} 个氧分子)起反应,生成 \qty{1}{mol} 二氧化碳(\num{6.02e23} 个 \ce{CO2} 分子),放出 \qty{94}{kCal}\footnote{按照国际单位制,热量的单位是焦耳(\unit{J})。在这里暂时仍沿用卡(\unit{Cal})作为热量单位。$\qty{1}{Cal}=\qty{4.184}{J}$。所列反应放出或吸收的热量的数据一般是指在 \qty{1}{atm} 和 \qty{25}{\celsius} 的条件下测得的热量。以下同。}的热。\qty{2}{mol} 氢气跟 \qty{1}{mol} 氧气起反应,生成 \qty{2}{mol} 水蒸气,放出 \qty{115.6}{kCal} 的热。
\[ \ce{C}\,\text{(固)} + \ce{O2}\,\text{(气)} \xlongequal{\quad} \ce{CO2}\,\text{(气)} +\qty{94}{kCal}\]
\[ \ce{2H2}\,\text{(气)} + \ce{O2}\,\text{(气)} \xlongequal{\quad} \ce{2H2O}\,\text{(气)} +\qty{115.6}{kCal}\]

上面的反应是放热的,也有一些反应是吸热的。
当水蒸气跟灼热的碳接触时,发生的反应就要吸收热量。
\[ \ce{C}\,\text{(固)} + \ce{H2O}\,\text{(气)} \xlongequal{\triangle} \ce{CO}\,\text{(气)} + \ce{H2O}\,\text{(气)} -\qty{31.4}{kCal}\]

\qty{1}{mol} 的碳跟 \qty{1}{mol} 的水蒸气起反应,吸收 \qty{31.4}{kCal} 的热量。

在化学方程式里,为什么要在物质的右边注明固、液、气等状态呢?
我们知道,物质呈现哪一种聚集状态是跟它们含有的能量有关的。
为了精确起见,要注明反应物和生成物的状态才能确定放出或吸收的热量多少。例如:
\begin{gather*}
  \ce{2H2}\,\text{(气)} + \ce{O2}\,\text{(气)} \xlongequal{\quad} \ce{2H2O}\,\text{(气)} +\qty{115.6}{kCal} \\
  \ce{2H2}\,\text{(气)} + \ce{O2}\,\text{(气)} \xlongequal{\quad} \ce{2H2O}\,\text{(液)} +\qty{136.6}{kCal} 
\end{gather*}

放出的热量用“$+$”号表示,吸收的热量用 “$-$” 号表示。
\emph{这种表明反应所放出或吸收的热量的化学方程式叫做}\Concept{热化学方程式}。

应用热化学方程式可以计算生产上出现的热量的变化。
例如,可以计算甲烷燃烧所放出的热量。
在生产上,很注意反应所放出的热量的充分利用。
\begin{example}
  \qty{1}{mol} 甲烷燃烧时,生成液态水和二氧化碳,同时放出 \qty{212.8}{kCal} 的热。计算燃烧 \qty{1000}{L}(标准状况)甲烷所产生的热量。
\end{example}
\begin{solution}
  \qty{1000}{L} 甲烷燃烧所产生的热量
  \[ = \frac{\qty{1000}{L}}{\qty{22.4}{L}} \times \qty{212.8}{kCal} = \qty{9.50e3}{kCal}\]
  答:\qty{1000}{L}(标准状况)甲烷燃烧所放出的热是 \qty{9.50e3}{kCal}。
\end{solution}

\subsection{燃烧热}
由于反应的情况不同,反应热可以分为许多种,如燃烧热、中和热等。
在这里,只介绍燃烧热。

许多单质或化合物在燃烧时放出热量,生成稳定的物质,如二氧化碳,水、氯化氢等等。

\emph{\qty{1}{mol} 物质完全燃烧时所放出的热量,叫做该物质的}\Concept{燃烧热}。
燃烧热通常可由实验测得。
例如,测得 \qty{1}{mol} 碳完全燃烧放出的热量是 \qty{94}{kCal},这就是碳的燃烧热。
\[ \ce{C}\,\text{(固)} + \ce{O2}\,\text{(气)} \xlongequal{\quad} \ce{CO2}\,\text{(气)} +\qty{94}{kCal}\]

由实验测得,\qty{1}{mol} 氢气燃烧而生成液态水,放出的热量是 \qty{68.3}{kCal},\qty{68.3}{kCal} 就是氢气的燃烧热。

\[ \ce{H2}\,\text{(气)} + \frac{1}{2}\ce{O2}\,\text{(气)} \xlongequal{\quad} \ce{H2O}\,\text{(液)} +\qty{68.3}{kCal}\]

在计算燃烧热时,可燃物质是以 \qty{1}{mol} 作为标准来计算的,所以热化学方程式里的元素符号或分子式前面的系数可以用分数表示。

由实验测得,\qty{1}{mol} 气态一氧化碳燃烧而生成气态二氧化碳时,放出 \qty{67.6}{kCal} 的热。

\[ \ce{CO}\,\text{(气)} + \frac{1}{2}\ce{O2}\,\text{(气)} \xlongequal{\quad} \ce{CO2}\,\text{(气)} +\qty{67.6}{kCal}\]

一氧化碳的燃烧热就是 \qty{67.6}{kCal}。

学习反应热概念,能够帮助我们理解反应中的热量变化,也是开始通过能量变化来了解物质性质及其反应过程。

\begin{Practice}[习题]
  \begin{question}
    \item 在足量氧气里燃烧 \qty{2.5}{mol} 的碳,生成二氧化碳,能放出多少千卡的热量?
    \item 要燃烧多少摩尔的氢气,生成液态水. 才能得到 \qty{1000}{kCal} 的热量?
    \item 比较燃烧氢气(生成液态水)和碳(生成二氧化碳)各 \qty{1}{kg} 所放出的热量?
    \item 燃烧 \qty{1}{g} 甲烷(\ce{CH4},气体),生成液态水和二氧化碳能放出 \qty{13.3}{kCal} 的热量,计算燃烧 \qty{5}{mol} 甲烷能放出多少热量。
    \item 燃烧 \qty{1}{g} 乙炔(\ce{C2H2},气体),生成液态水和二氧化碳能放出 \qty{11.9}{kCal} 的热量,计算燃烧 \qty{3}{mol} 乙炔所放出的热量。燃烧摩尔数相同的甲烷和乙炔,哪种气体放出的热量多?
    \item 燃烧 \qty{0.11}{g} 酒精(\ce{C2H4OH},液体),生成液态水和二氧化碳,放出的热量能使 \qty{100}{g} 水升高温度 \qty{7.12}{\celsius},计算燃烧 \qty{1}{mol} 酒精时放出的热量。(水的比热容为 \qty{1}{Cal/(g.\celsius)})
    \item 根据下列数据分别算出镁、铝、硫的燃烧热,并写出热化学方程式。
    \begin{tasks}
      \task \qty{1}{g} 镁完全燃烧,放出 \qty{6.0 }{kCal} 的热。
      \task \qty{3}{g} 铝完全燃烧,放出 \qty{21.4}{kCal} 的热。
      \task \qty{5}{g} 硫完全燃烧,生成二氧化硫,放出 \qty{11.1}{kCal} 的热。
    \end{tasks}
  \end{question}
\end{Practice}

\section*{内容提要}
\setcounter{subsection}{0}
\subsection{摩尔}
摩尔是表示物质的量的单位,每摩尔物质含有阿伏伽德罗常数个微粒(分子、原子、离子等)。


\subsection{反应热}
\begin{enumerate}
  \item 热化学方程式: 表明反应所放出或吸收的热量的化学方程式。
  \item 燃烧热: \qty{1}{mol} 物质在完全燃烧时所放出的热量。燃烧热热量的计算以 \qty{1}{mol} 反应物(可燃物)为单位。
\end{enumerate}

\begin{Review}
  \begin{question}
    \item 下列说法是否正确?并说明理由。
    \begin{tasks}
      \task 同温同压下,相同质量的气体都占有相同的体积。
      \task 摩尔浓度是指 \qty{1}{L} 水里所含溶质的摩尔数。
      \task \qty{1}{mol} \ce{NaCl} 在水里电离后,可以得到 \qty{0.5}{mol} \ce{Na+} 和 \qty{0.5}{mol} \ce{Cl-}。
    \end{tasks}
    \item 在标准状况时,有 \qty{11}{g} 二氧化碳、\qty{0.5}{mol} 氢气、\qty{10}{L} 氮气。根据上述情况,回答下列问题。
    \begin{tasks}
      \task 哪一种物质的质量最大,哪一种最小?
      \task 哪一种物质所含分子数最多,哪一种最少?
      \task 哪一种物质所占体积最大,哪一种最小?
    \end{tasks}
    \item \qty{2}{g} 硫铵肥料跟浓碱液混和加热,收集到 \qty{600}{mL} 氨(标准状况)。计算肥料含氮元素的百分比。
    \item 浓度为 15\%、密度为 \qty{1.2}{g/cm^3} 的废硫酸 \qty{250}{mL}(不含铁化合物或其它酸)跟过量的铁屑充分反应, 计算:
    \begin{tasks}
      \task 这种废硫酸的摩尔浓度。
      \task 制得氢气(标准状况)的体积。
      \task 把生成的硫酸亚铁配制成 \qty{400}{mL} 溶液,这溶液的摩尔浓度是多少。
    \end{tasks}
    \item 在含有硫酸钠和碳酸钠的溶液里,加入足量的氯化钡溶液,生成沉淀 \qty{3.5}{g}。把沉淀另用足量的硝酸溶液处理,放出 \qty{150}{mL} 二氧化碳气体(标准状况)。计算溶液里含硫酸钠和碳酸钠各多少摩尔。
    \item 计算燃烧多少克氢气,生成液态水放出的热量跟燃烧 \qty{1}{kg} 碳,生成二氧化碳所放出的热量相等。
    \item 称取 \qty{1.721}{g} 某种硫酸钙的结晶水合物 $\ce{CaSO4 .x H2O}$, 加热,使它失去全部结晶水。这时候硫酸钙的质量是 \qty{1.721}{g},计算所含结晶水数($x$)。
    \item 已知 \ce{KCl} 在 \qty{24}{\celsius} 时的溶解度是 \qty{33.2}{g},计算 \qty{24}{\celsius} 时 \ce{KCl} 饱和溶液的百分比浓度。这样浓度的 \ce{KCl} 溶液密度为 \qty{1.16}{g/cm^3},计算它的摩尔浓度。
  \end{question}
\end{Review}
\chapter{卤素}\label{chp:halogen}
我们在初中化学里,已经知道氟原子和氯原子的电子层结构,它们的最外电子层都有 7 个电子。
在 107 种元素的原子里,还有溴、碘、砹的原子结构跟氟和氯相似,在最外层都有 7 个电子。
氟、氯、溴、碘、砹具有相似的化学性质,成为一族,称为卤族元素,简称卤素。
砹在自然界里含量很少。
在这章里,主要介绍氯,并在认识氯的基础上,学习氟、溴、碘。

\section{氯气}
\subsection{氯气的性质}
\medskip\noindent
\begin{minipage}{0.52\linewidth}\parindent2em
氯气(\ce{Cl2})的分子是由两个氯原子\footnotemark[1]组成的双原子分子(\cref{fig:2-1})。 
氯分子也象氢分子一样,分子里共用电子对处在两个原子核的中间。 
氯气是一种非金属单质。
在通常情况下,氯气呈黄绿色,\qty{1}{atm} 下,冷却到 \qty{-34.6}{\celsius},变成液氯,液氯继续冷却到 \qty{-101}{\celsius},变成固态氯。
\end{minipage}\hfill
\begin{minipage}{0.43\linewidth}\centering
\begin{figurehere}
  \includegraphics{2-1.pdf}
  \caption{氯气分子}\label{fig:2-1}
\end{figurehere}
\end{minipage}
\footnotetext[1]{氯原子很小,它的原子半径,即氯分子中两个原子核间距离的一半,是 \qty{0.99e-10}{m}。}

\begin{Experiment}
  展示一瓶氯气,瓶后衬一张白纸,以便清晰地观察到氯气的颜色。
\end{Experiment}

氯气有毒,有剧烈的刺激性,吸入少量氯气会使鼻和喉头的粘膜受到刺激,引起胸部疼痛和咳嗽;吸入大量氯气会中毒致死。
实验室里,闻氯气的时候,必须十分小心,应该用手轻轻地在瓶口扇动,仅使极少量的氯气飘进鼻孔(\cref{fig:2-2})。

我们已经知道氯原子的最外电子层上有 7 个电子,因而在化学反应中容易结合一个电子,使最外电子层达到 8 个电子的稳定结构。
氯气的化学性质很活泼,它是一种活泼的非金属。
\begin{figure}
  \begin{minipage}[b]{0.48\linewidth}\centering
    \caption{闻氯气的方法}\label{fig:2-2}
  \end{minipage}
  \begin{minipage}[b]{0.48\linewidth}\centering
    \caption{铜在氯气里燃烧}\label{fig:2-3}
  \end{minipage}
\end{figure}

\subsubsection{氯气跟金属的反应}
氯气跟金属钠的反应很剧烈,这在初中化学里已经观察过。
氯气不但跟钠等活泼金属直接化合,而且还能跟受热的铜等某些不活泼的金属起反应。
\begin{Experiment}
  把一束细铜丝灼热后,立刻放进盛有氯 气的集气瓶里(\cref{fig:2-3}),观察发生的现象。把少量的水注入集气瓶里,用毛玻璃片把瓶口盖住,振荡。观察溶液的颜色。
\end{Experiment}
可以看到红热的铜丝在氯气里燃烧起来,集气瓶里充满棕色的烟,这是氯化铜晶体颗粒。
这个反应可以用化学方程式表示如下:
\[ \ce{ Cu + Cl2 \xlongequal{\triangle} CuCl2 } \]

氯化铜溶解在水里,成为绿色的氯化铜溶液。

\subsubsection{氯气跟非金属的反应}
\begin{Experiment}*
把新收集的一瓶氯气和一瓶氢气(氢气和氯气可以分别收集在透明或半透明的塑料制的集气瓶里),口对口地对着,抽去瓶口间的较璃片,上下颠倒几次,使氯气和氢气充分混和。
拿一瓶氯、氢混和气体作试验;用塑料片盖好,在离瓶约 \qty{10}{cm} 处点燃镁条,当发生的强光照射混和气体时,可以观察到因瓶里的氯气跟氢气迅速化合而发生的爆炸,把塑料片向上弹起(\cref{fig:2-4})。
\tcblower
\begin{figurehere}
  \caption{氯气和氢气化合}\label{fig:2-4}
\end{figurehere}
\end{Experiment}
氯气跟氢气起反应,生成氯化氢气体,放出大量的热,以致发生爆炸。
\[ \ce{H2}\,\text{(气)} + \ce{Cl2}\,\text{(气)} \xlongequal{\text{光照}} \ce{2HCl}\,\text{(气)} +\qty{44.1}{kCal} \]

\begin{Experiment}*
把红磷放在燃烧匙里,点燃后插入盛有氯气的集气瓶里,磷就燃烧起来(\cref{fig:2-5})。观察发生的现象。
\tcblower
\begin{figurehere}
  \caption{磷在氯气里燃烧}\label{fig:2-5}
\end{figurehere}
\end{Experiment}
氯气跟磷起反应,生成三氯化磷和五氯化磷。出现的白色烟雾是三氯化磷和五氯化磷的混和物。
\begin{gather*}
  \ce{ 2P +3Cl2 \xlongequal{\text{点燃}} 2PCl3 }\\
  \ce{ PCl3 +Cl2 \xlongequal{\quad} PCl5 }
\end{gather*}

三氯化磷是无色液体,是重要的化工原料,用来制造许多磷的化合物,如敌百虫等多种农药。

\subsubsection{氯气跟水的反应}
氯气溶解于水,在常温下,1 体积的水能够溶解约 2 体积的氯气。
氯气的水溶液叫做 “氯水”。
溶解的氯气能够跟水起反应,生成盐酸和次氯酸(\ce{HClO})。
\[ \ce{Cl2 + H2O \xlongequal{\quad} HCl + HClO} \]

次氯酸不稳定,容易分解,放出氧气。当氯水受自光照射时,次氯酸的分解加速了。

\begin{Experiment}*[righthand ratio=0.6]
  当日光照射到如\cref{fig:2-6} 盛有氯水的装置时,不久就见到有气泡逸出。
  \tcblower
  \begin{figurehere}
    \caption{氯水被分解}\label{fig:2-6}
  \end{figurehere}
\end{Experiment}
次氯酸是一种强氧化剂,能杀死水里的病菌,所以自来水常用氯气(\qty{1}{L} 水里约通入 \qty{0.002}{g} 氯气)来杀菌消毒。次氯酸能使染料和有机色质褪色,可用作漂白剂。

\begin{Experiment}
  取干燥的和湿润的有色布条各一条,放在如\cref{fig:2-7} 的装置里,观察发生的现象。
  \tcblower
  \begin{figurehere}
    \caption{次氯酸使色布褪色}\label{fig:2-7}
  \end{figurehere}
\end{Experiment}
\subsubsection{氯气跟碱的反应}
氯气跟氢氧化钠等碱类都能较快地发生反应,所以制氯气时可以用碱液吸收剩余的氯气。
\[ \ce{ 2NaOH + Cl2 \xlongequal{\quad} NaCl + NaClO + H2O } \]

由于次氯酸盐比次氯酸稳定,容易保存。
工业上就用氯气和消石灰制成漂白粉,漂白粉的有效成分是次氯酸钙。
制漂白粉的反应可以用化学方程式简单表示如下:
\[ \ce{ 2Ca(OH)2 + 2Cl2 \xlongequal{\quad} Ca(ClO)2 +CaCl2 +2H2O } \]

漂白粉应用于漂白的时候,使次氯酸钙跟稀酸或空气里的二氧化碳和水蒸气起反应,就生成次氯酸。
\begin{gather*}
  \ce{ Ca(ClO)2 + 2HCl \xlongequal{\quad} CaCl2 +2HClO } \\
  \ce{ Ca(ClO)2 + CO2 +H2O \xlongequal{\quad} CaCO3 v + 2 HClO }
\end{gather*}
\subsection{氯气的用途}
氯气除用于消毒、制造盐酸和漂白粉外,还用于制造多种农药,制造氯仿等有机溶剂,所以氯气是一种重要的化工原料。

\subsection{氯气的试验室制法}
在实验室里,氯气可以用浓盐酸跟二氧化锰起反应来制取。
\begin{Experiment}
  象\cref{fig:2-8} 所示那样把装置连接好,检查气密性。在烧瓶里加入少量二氧化锰粉末,从分液漏斗慢慢地注入密度为 \qty{1.19}{g/cm^3} 的浓盐酸。缓缓加热,使反应加速,氯气就均匀地放出。用向上排空气法收集,多余的氧气用氢氧化钠吸收。
  \tcblower
  \begin{figurehere}
    \caption{实验室制取氯气}\label{fig:2-8}
  \end{figurehere}
\end{Experiment}

这个反应可以用化学方程式表示如下:
\[ \ce{ 4HCl + MnO2 \xlongequal{\triangle} MnCl2 + 2H2O + Cl2 ^}\]

\begin{Practice}[习题]
  \begin{question}
    \item 下列说法里哪一条是正确的?
    \begin{tasks}
      \task 氯原子跟氯离子的性质是一样的。
      \task 氯离子比氯原子多一个电子。
      \task 氯离子呈黄绿色。
    \end{tasks}
    \item 新制备的氯水和长久搁置的氯水在成分上有什么不同?
    \item 写出氯气跟锌、铝、铁反应的化学方程式。
    \item 固态的磷跟氯气起反应,生成 \qty{1}{mol} 气态三氧化磷,放出 \qty{73.2}{kCal} 的热;气态三氯化磷再跟氧气起反应,生成  \qty{1}{mol} 气态五氧化磷,放出 \qty{22.1}{kCal} 的热。写出这两个热化学方程式。
    \item 取含 78\% \ce{MnO2} 的软锰矿 \qty{150}{g},跟足量浓盐酸起反应,可以制得氯气多少克?
  \end{question}
\end{Practice}
\section{氯化氢和盐酸}
\subsection{氯化氢}
\begin{Experiment}
  把少量食盐放在烧瓶里(\cref{fig:2-9})。通过分液漏斗注入浓硫酸,同时加热。把氯化氢收集在干燥的容器里。一部分氯化氢用水收集。
  \tcblower
  \begin{figurehere}
    \caption{实验室制取氯化氢}\label{fig:2-9}
  \end{figurehere}
\end{Experiment}

食盐跟浓硫酸起反应,不加热或稍微加热,就生成硫酸氢钠和氯化氢。
\[ \ce{NaCl + H2SO4 \text{(浓)} \xlongequal{\quad} NaHSO4 + HCl ^} \]
在 \qtyrange{500}{600}{\celsius} 的条件下,继续起反应而生成硫酸钠和氯化氢。
\[ \ce{NaHSO4 + NaCl \xlongequal{\triangle} Na2SO4 + HCl ^} \]

总的化学方程式可以表示如下:
\[ \ce{2NaCl + H2SO4 \xlongequal{\triangle} Na2SO4 +2HCl ^} \]

氯化氢是没有颜色而有刺激性的气体。
它易溶于水,在 \qty{0}{\celsius} 时,1 体积的水大约能溶解 500 体积的氯化氢。

\begin{Experiment}*[righthand ratio=0.6]
  在干燥的圆底烧瓶里装满氯化氢,用带有玻璃管和滴管(滴管里预先吸入水)的塞子塞紧瓶口。立即倒置烧瓶,使玻璃管放进盛着石蕊溶液的烧杯里。压缩滴管的胶头,出水几滴。烧杯里的溶液即由玻璃管喷入烧瓶,形成美丽的喷泉(\cref{fig:2-10})。
  \tcblower
  \begin{figurehere}
    \caption{氯化氢在水里溶解}\label{fig:2-10}
  \end{figurehere}
\end{Experiment}

\subsection{盐酸和金属氯化物}
氯化氢的水溶液呈酸性,叫氢氯酸,习惯上又叫盐酸。
盐酸的性质我们在初中化学里已经学过,它能够使酸碱指示剂变色,能够跟金属活动性顺序中氢以前的金属起置换反应,能够跟碱起中和反应,能够跟盐起复分解反应而生成不溶性 的或挥发性的物质。
它跟金属、碱或盐反应,生成金属氯化物。

金属氯化物在自然界里分布很广,也广泛地应用于日常生活中、工农业生产上等等。
重要的有氯化钠、氯化钾、氯化镁、氯化锌等。
在这里只介绍氯化钠。

氯化钠俗名食盐,它对于人和高等动物的正常生理活动是不可缺少的。
我们每天要吃一点食盐,来补充从尿、汗水里所排泄掉的氯化钠。
食盐在自然界里分布很广。
海水里含有丰富的食盐。
由于地壳的变化,食盐也蕴藏在盐湖、盐井和盐矿中。
我国有极为丰富的食盐资源,盛产海盐、井盐、池盐和岩盐。

海水和盐湖等都蕴藏着丰富的资源。
除生产食盐外,它们的综合利用能够制得钾肥和许多别的盐类,制得溴等产品。
这些产品是农业和许多工业不可缺少的原料。

从海水晒盐或用从盐井汲出的卤水煮盐,都是为了把水蒸发掉,使食盐溶液达到饱和,继续蒸发,食盐不断成晶体析出。
这样得到的食盐晶体还含有较多的杂质,常叫粗盐。
粗盐经过再结晶,就得到精盐。

纯净的氯化钠的晶体呈立方形,在 \qty{801}{\celsius} 熔化,在 \qty{1413}{\celsius} 沸腾。
纯净的氯化钠在空气里不潮解,粗盐因含有氯化镁、氯化钙等杂质,易潮解。

食盐的用途很广。
日常生活里用于调味和腌渍蔬菜、鱼肉、蛋类等等。
医疗上用的生理盐水是 0.9\% 的食盐水。
食盐是重要的化工原料,用于制取金属钠、氯气、氢氧化钠、纯碱等等化工产品。

\begin{Practice}[习题]
  \begin{question}
    \item \qty{11.2}{L} 氧气和 \qty{11.2}{L} 氢气起反应,生成多少升的氯化氢气体(气体体积都按标准状况计),把生成的氧化氢都溶解在 \qty{328.5}{g} 水里形成密度为 \qty{1.047}{g/cm^3} 的盐酸,计算这种盐酸的摩尔浓度。
    \item 写出盐酸跟下列物质起反应的化学方程式。
    \begin{tasks}(3)
      \task \ce{Mg}
      \task \ce{MgO}
      \task \ce{Mg(OH)2}
      \task \ce{Mg(HCO3)2}
      \task \ce{MgCO3}
    \end{tasks} 
    \item \qty{11.7}{g} 氯化钠跟 \qty{10}{g} 浓度为 98\% 的硫酸反应,微热时生成多少克氟化氢?继续加热到 \qty{600}{\celsius} 时,又生成多少克的氯化氢?
    \item \qty{1}{g} 的锌和 \qty{1}{g} 的铁分别跟足量的稀盐酸起反应,各生成多少升的氢气(在标准状况下)?
    \item \qty{1.5}{mL} 密度为 \qty{1.028}{g/cm^3} 的盐酸(6\% \ce{HCl})跟足量的硝酸银溶液起反应,计算生成的氯化银的质量。
    \item 为什么在晒盐的时候,日晒风吹都有利于食盐晶体的析出?
    \item 为什么制食盐的时候,不宜采用降低溶液温度的方法?
  \end{question}
\end{Practice}

\section{氧化—还原反应}
我们在初中化学里,已经学习过氧化—还原反应,知道氧化、还原并不限于得氧或失氧的反应,而是可以用正负化合价升降的观点来分析氧化-还原反应。
在反应过程里,物质所含元素化合价升高的反应是氧化反应;物质所含元素化合价降低的反应是还原反应。
所含元素化合价升高的物质是还原剂,所含元素化合价降低的物质是氧化剂。
我们现在就用这个观点来分析在这章里学过的几个氧化—还原反应。

我们在初中化学里已经知道,氯化钠是由氯离子和钠离子构成的。
钠原子失去 1 个电子,成为钠离子,氯原子得到 1 个电子,成为氯离子。
在这里,可以从电子得失的观点进一步分析几个氧化-还原反应。


我们知道,氯气跟钠或铜起反应,分别生成了氯化钠和氯化铜。
在反应过程里,铜原子也象钠原子一样失去了电子成为铜离子。
在下面的化学方程式里,“$e$” 表示电子,并用箭头表明同一元素的原子得到或失去电子的情况。


\ce{NaCl} 里,钠是 $+1$ 价,氯是 $-1$ 价。
在反应过程里,钠原子失去 1 个电子,钠从 0 价变到 $+1$ 价,化合价升高;氯原子得到 1 个电子,氟从 0 价变到 $-1$ 价,化合价降低。
同样地,铜原子失去 2 个电子,铜从 0 价升高到 $+2$ 价,化合价升高;氯原子得到 1 个电子,氯从 0 价降低到 $-1$ 价,化合价降低。
元素化合价升高是由于失去电子,升高的价数也就是失去的电子数。
元素化合价降低是由于得到电子,降低的价数也就是得到的电子数。
元素的化合价升降的原因就是它们的原子失去或得到电子的缘故。
所以,我们可以给氧化—还原反应下一个更确切的定义。

\emph{物质失去电子的反应就是}\Concept{氧化反应},\emph{物质得到电子的反应就是}\Concept{还原反应}。

在氧化—还原反应里,一种物质失去电子,必然同时有别的物质得到电子;一种物质失去电子的数目总是跟别的物质得到电子的数目相等。
\emph{得到电子的物质是}\Concept{氧化剂},\emph{失去电子的物质是}\Concept{还原剂}。

在下面的化学方程式里,用箭头表示不同元素间电子得失的情况。


但是,也有一些反应,如上面讲到的氢气跟氢气生成氯化氢的反应,生成的氯化氢是共价化合物。
我们在初中化学里已经知道,氯化氢分子里的共用电子对是偏向于氯原子,偏离于氢原子的。


在上述电子转移过程里,就没有那么显著,也就是并没有完全失去或完全得到电子,而是共用电子对的偏移。
这样的反应也属于氧化—还原反应。在这个反应里,氯气是氧化剂,氢气是还原剂。

氧化—还原反应里,电子转移(得失或偏移)、正负化合价升降的关系,可以用\cref{fig:2-11} 表示。
\begin{figure}
  \caption{氧化—还原反应中电子得失、化合价变化的关系简图}\label{fig:2-11}
\end{figure}

\begin{example}
  分析镁跟稀盐酸的反应里镁元素和氢元素在反应前后的电子得失、化合价升降和氧化、还原的关系。哪一种物质是氧化剂,哪一种物质是还原剂?
\end{example}
\begin{solution}
  写出反应的化学方程式,并用箭头表示镁元素和氢元素在反应前后的电子得失等。
  \[ \]
  答: \ce{Mg} 是还原剂,\ce{HCl} 是氧化剂。
\end{solution}

根据对许多例子的分析,可以说凡是没有电子转移、也就是没有正负化合价升降的反应,就不属于氧化—还原反应。

在工农业生产、科学技术和日常生活里,我们经常会碰到许多氧化—还原反应。
所以氧化—还原反应是很重要的一类化学反应。

\begin{Practice}[习题]
  \begin{question}
    \item 说明在下列反应里的氧化、还原、化合价升降的关系。
    \begin{tasks}
      \task \ce{ 2P + 3Cl2 $\xlongequal{\quad}$ 2PCl3}
      \task \ce{ C + O2 $\xlongequal{\quad}$ CO2}
      \task \ce{ 2Sb + 5Cl2 $\xlongequal{\quad}$ 2SbCl5}
      \task \ce{ 2KClO3 $\xlongequal{\quad}$ 2KCl + 3O2 }
    \end{tasks}
    \item 你怎样理解氧化—还原反应跟电子得失的关系?举例说明。
    \item 分析下列化学反应中化合价变化的关系,由此说明反应中的电子得失。
    \begin{tasks}
      \task \ce{ 2Mg + O2 $\xlongequal{\quad}$ 2MgO }
      \task \ce{ Zn + 2HCl $\xlongequal{\quad}$ ZnCl2 + H2 ^}
    \end{tasks}
    \item 分析下列氧化-还原反应里电子的转移情况,哪种物质是氧化剂,哪种物质是还原剂?
    \begin{tasks}
      \task \ce{ Zn + H2SO4 $\xlongequal{\quad}$ ZnSO4 + H2 ^}
      \task \ce{ Fe + 2HCl $\xlongequal{\quad}$ FeCl2 + H2 ^}
    \end{tasks}
    \item 写出氯气分别跟钾、钙起反应的化学方程式。在各个反应里分别指出不同元素间电子转移的关系。在各个反应里, 哪种物质是氧化剂,哪种物质是还原剂?
  \end{question}
\end{Practice}

\section{卤族元素}
我们已经学过氯元素的单质和一些重要的化合物。
现在学习氟、溴、碘等元素,并把它们跟氯元素相比较,它们在性质上、原子结构上有哪些相似和不同的地方。

\subsection{卤素的原子结构和它们的单质的物理性质}
卤素在自然界里都以化合态存在,它们的单质可由人工制得。
卤素的单质都是双原子的分子。
\cref{tab:2-1} 列出各元素的原子结构和单质的物理性质。

\begin{table}
  \caption{卤素的原子结构和单质的物理性质}\label{tab:2-1}
\end{table}

从\cref{tab:2-1} 可以看出,氟、氯、溴、碘的原子的最外电子层的电子数是相同的,都是 7 个电子,但电子层数不同。
因此,它们的原子半径或离子半径\footnote{离子半径是根据阴阳离子的核间距推算出来的。阴离子的半径比它们的原子半径大,阳离子的半径比它们的原子半径小。}都随着电子层数的增多而增大(\cref{fig:2-12}),它们的离子都因得到了一个电子,离子半径比相应的原子半径增大了。

\begin{figure}
  \caption{卤素的原子和离子大小示意图(数据单位是 \qty{e-10}{m})}\label{fig:2-12}
\end{figure}

从\cref{tab:2-1} 还可以看出,卤素的物理性质有较大的差别。
如在常温,氟、氯是气体,溴是液体,碘是固体,它们的沸点、熔点都逐渐升高,颜色由淡黄绿色到紫黑色,逐渐转深。

\begin{Experiment}
  打开盛溴的瓶的盖子,有什么现象发生?观察液态和气态的溴。
\end{Experiment}

可以观察到,液态溴容易挥发成溴蒸气。

\begin{Experiment}*[righthand ratio=0.5]
  观察碘的颜色、状态和光泽。把少量的碘晶体放在烧杯里,烧杯上放盛冷水的烧瓶,稍稍加热(\cref{fig:2-13})。观察发生的现象。
  % \tcblower
  % \begin{figurehere}
  %   \caption{碘的升华}\label{fig:2-13}
  % \end{figurehere}
\end{Experiment}

可以观察到,碘在常压下加热,不经过熔化就直接变成紫色蒸气,蒸气遇冷,重新凝成固体。
这种固态物质不经过转变成液态而直接变成气态的现象叫做\Concept{升华}。
\begin{figure}
  \begin{minipage}[b]{0.48\linewidth}\centering
    \caption{碘的升华}\label{fig:2-13}
  \end{minipage}
  \begin{minipage}[b]{0.48\linewidth}\centering
    \caption{溴在不同溶剂里的溶解}\label{fig:2-14}
  \end{minipage}
\end{figure}

\begin{Experiment}
  把水注入盛着少量溴的试管,振荡,水溶液的颜色显橙色(图 2-14,I)。把上部橙色溶液倒在另一个试管里,再注入少量无色的汽油(或苯或四氯化碳)(图 2-14,II),用力振荡,静置一会儿。观察油层和水溶液的颜色。
\end{Experiment}

\begin{Experiment}
  把水、酒精分别注入两个试管,各约占小半试管,并各投入少量的碘的晶体,振荡。比较碘在两种液体里的溶解性。把碘的水溶液注入另一空试管,再注入少量无色汽油(或苯或四氯化碳),振荡,静置一会儿,观察油层和溶液的颜色。
\end{Experiment}

可以观察到溴和碘都比较容易溶解于汽油、苯、四氯化碳、酒精等有机溶剂中。医疗上用的碘酒,即是碘的酒精溶液。

\subsection{卤素的单质的化学性质}
我们知道,氯的化学性质很活泼,它的原子的最外电子层是 7 个电子,在化学反应中容易得到一个电子而成为 8 个电子的稳定结构。
氟、溴、碘的原子的最外电子层也都是 7 个电子,因而它们的化学性质跟氯有很大的相似性。
\subsubsection{卤素都能跟金属起反应生成卤化物}


氟、溴、碘都能象氯一样跟钠等金属起反应。自然界里,也存在着许多种的金属跟卤素的化合物,如氟化钙、氯化钠、 氯化镁、溴化钾、碘化钾等等卤化物。

\subsubsection{卤素都能跟氢气起反应,生成卤化氢}
\begin{Experiment}
  在一个铁坩埚里放锌粉 \qty{0.5}{g},加入碘粉 \qty{0.5}{g},混和后,加水 1~2 滴作为催化剂,观察发生的现象。
\end{Experiment}
氟的性质比氯更活泼,氟气跟氢气的反应不需光照,在暗处就能剧烈化合,并发生爆炸。
\[ \ce{H2}\,\text{(气)} + \ce{F2}\,\text{(气)} \xlongequal{\quad} \ce{2HF}\,\text{(气)} + \qty{128.4}{kCal}\]

溴的性质不如氯活泼,溴跟氢气的反应在达到 \qty{500}{\celsius} 时即较慢地进行。
\[ \ce{H2}\,\text{(气)} + \ce{Br2}\,\text{(气)} \xlongequal{\quad} \ce{2HBr}\,\text{(气)} + \qty{17.3}{kCal}\]

碘的性质比溴更不活泼,碘跟氢气的反应在不断加热条件下缓慢地进行,生成的碘化氢很不稳定,同时发生分解。

\[ \ce{H2}\,\text{(气)} + \ce{I2}\,\text{(固)} \xlongequal{\triangle} \ce{2HI}\,\text{(气)} - \qty{12.4}{kCal}\]

卤素也能跟磷等非金属起反应。

\subsubsection{卤素跟水的反应}
氟遇水发生剧烈的反应,生成氟化氢和氧气。
\[ \ce{2H2O + 2F2 \xlongequal{\quad} 4HF + O2 ^}\]

溴跟水的反应比氯气跟水的反应更弱一些,碘跟水只有很微弱的反应。
\subsubsection{卤素各单质的活动性比较}
\begin{Experiment}
  把少量氯水分别注入盛着溴化钠溶液和碘化钾溶液的两个试管里,用力振荡后,再注入少量无色汽油 (或四氯化碳)。振荡。观察油层和溶液颜色的变化。
\end{Experiment}
\begin{Experiment}
  把少量溴水注入盛着碘化钾溶液的试管里,用力振荡。观察溶液颜色的变化。
\end{Experiment}

溶液颜色的变化,说明氯可以把溴或碘从它们的化合物里置换出来,溴可以把碘从它的化合物中置换出来。
\begin{gather*}
  \ce{ 2NaBr +Cl2 \xlongequal{\quad} 2NaCl +Br2 } \\ 
  \ce{ 2KI +Cl2 \xlongequal{\quad} 2KCl +I2 } \\ 
  \ce{ 2KI +Br2 \xlongequal{\quad} 2KCBr +I2 } 
\end{gather*}

由此可以证明,在氯、溴、碘这三种元素里,氯比溴活泼,溴又比碘活泼。
实验证明,氟的性质比氯、溴、碘更活泼,能把氯等从它们的卤化物中置换出来。

氟的性质特别活泼,它甚至还能够跟惰性气体中的氙、氪等起反应,生成氙和氪的氟化物\footnote{氙和氪的氟化物的结构比较特殊,不宜应用通常的正负化合价来解释。}:\ce{XeF2}、\ce{XeF4}、\ce{XeF6}、\ce{KrF2}等。
它们在常温下都是白色固体。

\subsubsection{碘跟淀粉的反应}
碘遇淀粉变蓝色。利用碘的这个特性,可以鉴定碘的存在。
\begin{Experiment}
  在试管里注入少量淀粉溶液,滴入几滴碘水,溶液显示出特殊的蓝色。
\end{Experiment}

从卤素的化学性质可以看出,它们有很多相似的地方,但也有差别(\cref{tab:2-2})。
卤素的原子,最外电子层都有 7 个电子,结合外来电子的能力很强,所以卤素是活泼的非金属元素。
卤素容易得到电子而被还原,它们本身是强氧化剂。
但是,氟、氯、溴、碘各原子的核电荷数不同、核外电子层数不同,原子和离子的大小也都不同,各原子核对外层电子的引力也有所不同。
原子的大小对非金属的活动性有很密切的关系。
氟的原子较小,外层电子受到核的引力最强,它得到电子的能力很强,非金属性最活泼,所以合成的氟化氢最稳定,合成的时候反应放热,并最剧烈。
碘的原子较大,最外层电子受到核的引力较弱,它得到电子的能力也较弱,非金属性也较弱,生成的碘化氢不稳定,合成的时候要吸热。
氯和溴的非金属性是介乎其间的,氯比溴又活泼一些。
总的看来,卤素是活泼的非金属元素,它们的活动性又随着核电荷数和电子层数的增加、原子半径的增大而减弱。

\begin{table}
  \caption{卤素的单质的化学性质比较}\label{tab:2-2}
\end{table}

\begin{Theorem}{讨论}
  从哪些方面可以比较氟、氯、溴、碘的性质的相似和不同之处?
\end{Theorem}



\subsection{卤素的几种化合物}
\subsubsection{氟化氢和氟化钙}
氟化钙俗名萤石,是自然界里存在相当广泛的氟的化合物。

使浓硫酸跟萤石在铅皿中起反应,就制得氟化氢。
\[ \ce{ CaF2 + H2SO4 \xlongequal{\quad} CaSO4 + 2HF ^ }\]

氟化氢也象氯化氢那样,在空气里呈现白雾。
氟化氢有剧毒。
氟化氢溶解在水里就是氢氟酸。

氟化氢用于雕刻玻璃和制造塑料、橡胶、药品等,还用于制备单质氟和提炼铀。
氟化氢还用于制造氟化钠等氟化物。
氟化钠是一种用来杀灭地下害虫的农药。

\subsubsection{溴化银和碘化银}
\begin{Experiment}
把少量硝酸银溶液分别滴入盛着溴化钠溶液和碘化钾溶液的两个试管。
在盛溴化钠溶液的试管里有浅黄色的溴化银沉淀生成,在盛碘化钾溶液的试管里有黄色的碘化银沉淀生成。
在两个试管里各加入少量稀硝酸,生成的溴化银、碘化银沉淀都不溶解。
\end{Experiment}
\begin{gather*}
  \ce{ NaBr + AgNO3 \xlongequal{\quad} AgBr v + NaNO3}\\
  \ce{ KI + AgNO3 \xlongequal{\quad} AgI v + KNO3 }
\end{gather*}

溴化银和碘化银都有感光性,在光的照射下会起分解反应。例如:
\[ \ce{ 2AgBr \xlongequal{\text{光照}} 2Ag + Br2} \]

照相用的感光片,就是在暗室里用溴化银的明胶凝胶,均匀地涂在胶卷或玻璃片上制成的。
照相时利用溴化银有感光性的原理,使底片感光后,再用还原剂(显影剂)和定影剂处理,得到明暗程度跟实物相反的底片。使感光片通过底片曝光,再经显影和定影,就得到明暗程度跟实物一致的照片。

碘化银可用于人工降雨,用小火箭、高射炮等工具把磨成很细粉末的碘化银发射到几千米的高空,使空气里的水蒸气凝聚成雨。

\begin{Practice}[习题]
  \begin{question}
    \item 写出氟、溴、碘跟金属钠的反应的化学方程式。
    \item 在三个试管里,分别盛着氯化钠、溴化钠、碘化钾的溶液。各加入一些氯水,发生什么反应?再加入一些溶剂汽油,振荡。出现什么现象?
    \item 从哪些化学性质可以证明氟是卤素中最活泼的元素?
    \item 日光照射在下列物质上,各有什么现象发生?为什么?写出化学方程式。
    \begin{tasks}(3)
      \task 氯水
      \task 氯气和氢气的混和物
      \task 溴化银
    \end{tasks} 
    \item 怎样鉴别单质碘和碘离子?
    \item 现有二氧化锰、氯化钾、溴化钾、浓硫酸和水五种物质,怎样从这些物质来制取盐酸、氯气和溴。写出反应的化学方程式。
    \item 氟化钙跟浓硫酸起反应,制得氟化氢,写出反应的化学方程式。为什么这个反应的装置不能用玻璃仪器?
  \end{question}
\end{Practice}

\section*{内容提要}
卤素是一族非金属元素。
它们的原子结构的共同之点是最外电子层都有 7 个电子,在化学反应中容易得到电子;差别之处是核电荷数不同、电子层数不同,原子半径也不同,由此形成了卤族元素既相似又有着差别的性质。
它们的化学性质主要是强的非金属性,它们的单质都是强氧化剂。
卤素里,氟原子很小,非金属性很强,氯、溴、碘随着原子的增大而非金属性减弱。

卤素的化学性质包括:
\begin{itemize}
  \item 跟金属的反应——生成金属卤化物。
  \item 跟氢气的反应——生成卤化氢。
  \item 跟水的反应——生成氢卤酸、次卤酸。
  \item 跟氢氧化物的反应——生成金属卤化物等。
  \item 跟卤化物的反应——把较不活泼的卤素从它们的卤化物溶液里置换出来。
\end{itemize}

物质失去电子的反应是氧化反应,物质得到电子的反应是还原反应。氧化和还原反应必然同时发生。
\begin{Review}
  \begin{question}
    \item 下列说法是否正确,如不正确,加以改正。
    \begin{tasks}
      \task 氯酸钾里有氧气,所以加热时有氧气放出。
      \task 卤素各单质都可以成为氧化剂。
      \task \qty{1}{mol} 液态 \ce{HCl} 在标准状况时约占 \qty{22.4}{L}。
    \end{tasks}
    \item 能不能在下列条件下收集氯气?说明原因,并写出可能发生的反应的化学方程式。
    \begin{tasks}(3)
      \task 排水收集
      \task 排氢氧化钾溶液收集
      \task 排碘化钾溶液收集
    \end{tasks}
    \item 在用氯酸钾制氧气和浓盐酸制氯气时都要用到二氧化锰,二氧化锰的作用是否相同,分别是起了什么作用?
    \item 实验室里制备氢气、氯气、二氧化碳都要用到盐酸。盐酸在制备这三种气体的反应里,各起什么作用?
    \item 从电子得失的观点来看,在已经学过的化学反应类型里,置换反应都属于氧化—还原反应,一部分化合反应和分解反应属于氧化—还原反应,复分解反应都不属于氧化—还原反应。你认为这个结论合理吗? 举例说明。
    \item 一包白色固体,可能是 \ce{CaCl2}、\ce{Na2CO3}、\ce{NaI} 三种物质之一,也可能是两者或三者的混和物。当把白色固体溶解于水时,发现有白色沉淀。过滤后,滤液是无色的。
    \begin{tasks}
      \task 把滤纸上的沉淀,移到试管里,加入盐酸,有气体产生。把气体通入澄清石灰水,石灰水显浑浊。
      \task 把滤液分成两部分。向一部分滤液加入几滴硝酸银溶液,出现白色沉淀。再加稀硝酸,白色沉淀不溶解。这白色沉淀光照后,逐渐呈黑色。向另一部分滤液里加几滴氯水,再注入少许四氯化碳,振荡,四氯化碳层不显紫色。
    \end{tasks}
    试分析并判断这包白色固体里含有哪些物质。写出反应的化学方程式。
    \item 使足量浓硫酸跟 \qty{11.7}{g} 氯化钠混和微热,反应生成的氯化氢通入 \qty{45}{g} 10\% 氢氧化钠溶液,向最后的溶液滴入几滴石蕊试液,将显什么颜色。
    \item 浓盐酸跟二氧化锰起反应,生成的氯气能从碘化钠溶液里置换出 \qty{1.27}{g} 碘。计算至少需用多少摩尔的 \ce{HCl} 和 \ce{MnO2} 。
    \item 把滤纸用淀粉和碘化钾的溶液浸泡,晾干后就是实验室常用的淀粉碘化钾试纸。这种试纸润湿后遇到氯气会发生什么变化?为什么?
  \end{question}
\end{Review}
\chapter{硫\texorpdfstring{\quad}{ }硫酸}
硫是一种重要的非金属元素。
硫的原子结构和性质跟我们已经学过的氧很相似,它们的原子的最外电子层都有 6 个电子,氧(\ce{O})、硫(\ce{S})和另外三种原子结构和性质相似的元素硒(\ce{Se})、碲(\ce{Te})、钋(\ce{Po})\footnote{硒音\pinyin{xi1},碲音\pinyin{di4},钋音\pinyin{po1}。},统称为氧族元素。
本章主要介绍硫及其化合物的知识。

\section{硫}
在自然界,游离态的天然硫, 存在于火山喷口附近或地壳的岩层里。
由于天然硫的存在,人类从远古时代起就知道硫了。
以化合态存在的硫分布很广,主要是硫化物和硫酸盐,如硫铁矿(\ce{FeS2}),黄铜矿(\ce{CuFeS2}),石膏(\ce{CaSO4.2H2O}),芒硝(\ce{Na2SO4.10H2O}),等等。
硫的化合物也常存在于火山喷 出的气体中和矿泉水里。
煤和石油里都含有少量硫。
硫还是某些蛋白质的组成元素,是生物生长所需要的一种元素。

\subsection{硫的物理性质}
硫通常是一种淡黄色的晶体,俗称硫黄。
它的密度大约是水的两倍。
硫很脆,容易研成粉末,不溶于水,微溶于酒精,容易溶于二硫化碳。
硫的熔点是 \qty{112.8}{\celsius},沸点是 \qty{444.6}{\celsius}。

\subsection{硫的化学性质}
硫的化学性质比较活泼,跟氧相似,容易跟金属、氢气和其它非金属发生反应。

\subsubsection{硫跟金属的反应}
\begin{Experiment}*[righthand ratio=0.7]
给盛着硫粉的大试管加热。加热到硫沸腾产生蒸气时,用坩埚钳夹住一束擦亮的细铜丝伸入管口(\cref{fig:3-1}),观察发生的现象。
  \tcblower 
  \begin{figurehere}
    \caption{铜在硫蒸气里燃烧}\label{fig:3-1}
  \end{figurehere}
\end{Experiment}
铜丝在硫蒸气里燃烧,生成黑色的硫化亚铜。
\[ \ce{ 2Cu + S \xlongequal{\triangle} Cu2S}\]

\begin{Experiment}
把少量硫粉和铁粉的混和物装在试管里,加热到红热,立即把酒精灯移开。
反应里放出的热,就能使反应继续进行。观察发生的现象。
\end{Experiment}

硫跟铁起反应,生成黑色的硫化亚铁。
\[ \ce{ Fe + S \xlongequal{\triangle} FeS} \]

硫还能跟其它金属起反应。硫跟金属的化合物,叫做金属硫化物。

\subsubsection{硫跟非金属的反应}
在初中化学里,我们已经知道硫能跟氧气发生反应,生成二氧化硫,并放出大量的热。
\[ \ce{S}\,\text{(固)} + \ce{O2}\,\text{(气)}\ce{ \xlongequal{\quad} SO2}\,\text{(气)} + \qty{71}{kCal}\]

此外,硫还能跟其它非金属发生反应。例如,硫的蒸气能跟氢气直接化合而生成硫化氢气体:
\[ \ce{S + H2 \xlongequal{\triangle} H2S }\]

\subsection{硫的用途}
硫的用途很广。
硫主要用来制造硫酸。
硫也是生产橡胶制品的重要原料。
硫还可用于制造黑色火药、焰火、火柴等。
硫又是制造某些农药(如石灰硫黄合剂)的原料。
医疗上,硫还可用来制硫黄软膏医治某些皮肤病,等等。
\begin{Practice}[习题]
  \begin{question}
    \item 写出硫跟氢气、硫跟氧气反应的化学方程式。
    \item 试计算 \qty{0.32}{kg} 的硫燃烧生成二氧化硫,放出多少热量。
    \item \qty{21}{g} 铁粉跟 \qty{8}{g} 硫粉混和加热可生成硫化亚铁多少克,哪一种物质过剩,剩余多少?
  \end{question}
\end{Practice}
\section{硫的氢化物和氧化物}
\subsection{硫的氢化物——硫化氢(\texorpdfstring{\ce{H2S}}{H2S})}
\subsubsection{硫化氢的实验室制法}
在实验室里,硫化氢通常是由硫化亚铁跟稀盐酸或稀硫酸反应而制得的。
\[ \ce {FeS +2HCl \xlongequal{\quad} FeCl2 + H2S ^}\]
\[ \ce {FeS +H2SO4 \xlongequal{\quad} FeSO4 + H2S ^}\]

这个反应可以在启普发生器里进行。

\subsubsection{硫化氢的性质}
硫化氢是一种没有颜色而有臭鸡蛋气味的气体。
它的密度比空气略大。
硫化氢有剧毒,是一种大气污染物。
空气里如果含有微量的硫化氢,就会使人感到头痛、头晕和恶心。
吸入较多的硫化氢,会使人昏迷甚至死亡。
因此,制取或使用硫化氢时,必须在密闭系统或通风橱中进行。

硫化氢能溶于水。
在常温、常压下,1 体积的水能溶解 2.6 体积的硫化氢。

在较高温度时,硫化氢分解成氢气和硫。
\[ \ce{H2S \xlongequal{\triangle} H2 + S}\]

硫化氢是一种可燃性气体。
\begin{Experiment}
在导管口用火点燃硫化氢气体,观察硫化氢完全燃烧时火焰的颜色。
然后把干燥的烧杯罩在火焰的上方,观察烧杯内壁附有什么物质。小心地闻一下气味。
\end{Experiment}

在空气充足条件下,硫化氢能完全燃烧而发生淡蓝色的火焰,并生成水和二氧化硫。
\[ \ce{ 2H2S + 3O2 \xlongequal{\text{点燃}} 2H2O +2SO2} \]

\begin{Experiment}
在导管口用火点燃硫化氢气体,用一个蒸发皿(或玻璃片),使蒸发皿底靠近硫化氢的火焰,观察蒸发皿底发生的现象。
\end{Experiment}
我们可以看到,蒸发皿底部附有黄色的粉末。这是硫化氢不完全燃烧时析出的单质硫。
\[ \ce{ 2H2S + O2 \xlongequal{\quad} 2H2O + 2S}  \]

如果在一个集气瓶里,使硫化氢跟二氧化硫两种气体充分混和。不久,在瓶壁上就有黄色的粉末一一硫的生成。
\[ \ce{ SO2 +2H2S \xlongequal{\quad} 2H2O + 3S} \]

由此可见,硫化氢具有还原性。
硫化氢里的硫是 $-2$ 价,它能够失去电子而变成游离态的单质硫或高价硫的化合物。

硫化氢的水溶液能够使石蕊试液变为浅红色,它是一种酸,叫做氢硫酸,当这种酸受热时,硫化氢又从水里逸出。
氢硫酸是一种弱酸,它具有酸的通性。

\subsection{硫的氧化物}
硫的氧化物中最重要的是二氧化硫和三氧化硫。

\subsubsection{二氧化硫(\ce{SO2})}
二氧化硫是没有颜色而有刺激性气味的有毒气体。
它的密度比空气大,容易液化(沸点是 \qty{-10}{\celsius}),易溶于水,在常温、 常压下,1 体积水大约能溶解 40 体积二氧化硫。

二氧化硫是酸性氧化物,它跟水化合而生成亚硫酸(\ce{H2SO3})。
因此,二氧化硫又叫做亚硫酐。
\[ \ce{ SO2 + H2O \xlongequal{\quad} H2SO3 }\]

亚硫酸很不稳定,容易分解成二氧化硫和水。
\[ \ce{ H2SO3 \xlongequal{\quad} H2O + SO2 ^} \]

通常把向生成物方向进行的反应叫做\Concept{正反应},向反应物方向进行的反应叫做\Concept{逆反应}。
象这种\emph{在同一条件下,既能向正反应方向进行,同时又能向逆反应方向进行的反应,叫做}\Concept{可逆反应}。
在化学方程式里,用两个方向相反的箭头代替等号来表示可逆反应。
\[ \ce{ SO2 + H2O <=> H2SO3 }\]

二氧化硫在适当的温度并有催化剂存在的条件下,可以被氧气氧化而生成三氧化硫。
\[ \ce{ 2SO2 +O2} \xlongequal[\triangle]{\text{催化剂}} \ce{2SO3} \]

\begin{Experiment}
把二氧化硫气体通入盛有品红溶液的试管里,观察品红溶液颜色的变化。把试管加热,再观察溶液发生的变化。
\end{Experiment}

\begin{Reading}[]{补充阅读}
亚硫酸钠是亚硫酸(\ce{H2SO3})的钠盐。亚硫酸钠溶液跟硫反应,生成硫代硫酸钠(\ce{Na2S2O3})。
\[ \ce{ Na2SO3 +S \xlongequal{\triangle} Na2S2O3} \]

硫代硫酸钠是硫代硫酸(\ce{H2S2O3})的钠盐。
硫代硫酸可以看作是硫酸分子中的一个氧原子被硫原子代换后所生成的酸。
带有五个结晶水的硫代硫酸钠(\ce{Na2S2O3.5H2O}),俗称大苏打或海波。
它是无色晶体,溶于水,在照相业中常用作定影剂,用于溶解照相底片或感光纸上尚未感光的溴化银。
\end{Reading}

\subsubsection{三氧化硫(\ce{SO3})}

\section{硫酸的工业制法——接触法}
\subsection{接触法制造硫酸的反应原理和生产过程}
\subsection{尾气中二氧化硫的回收和环境保护}
\begin{Practice}[习题]
  \begin{question}
    \item 
    \item 
    \item 
    \begin{tasks}
      \task
      \task
      \task
    \end{tasks}
    \item 
  \end{question}
\end{Practice}
\section{硫酸\texorpdfstring{\quad}{ }硫酸盐}
\subsection{硫酸}
\subsection{硫酸盐}
\subsection{硫酸根离子的检验}
\begin{Practice}[习题]
  \begin{question}
    \item 
    \begin{tasks}
      \task
      \task
      \task
      \task
      \task
    \end{tasks}
    \item 
    \item 
    \item 
    \item 
  \end{question}
\end{Practice}
\section{离子反应\texorpdfstring{\quad}{ }离子方程式}
\subsection{离子反应\texorpdfstring{\quad}{ }离子方程式}
\subsection{离子反应发生的条件}
\begin{Practice}[习题]
  \begin{question}
    \item 
    \begin{tasks}
      \task
      \task
      \task
      \task
      \task
      \task
    \end{tasks}
    \item 
    \item 
    \begin{tasks}
      \task
      \task
      \task
      \task
      \task
    \end{tasks} 
    \item 
    \item 
  \end{question}
\end{Practice}
\section{氧族元素}
\begin{Practice}[习题]
  \begin{question}
    \item 
    \item 
  \end{question}
\end{Practice}
\section*{内容提要}
\subsection{氧族元素}
\subsection{硫的化学性质}
\subsection{硫的重要氧化物}
\subsection{硫酸}
\subsection{离子反应和离子方程式}
\begin{Review}
  \begin{question}
    \item
    \begin{tasks}(5)
      \task 氧气
      \task 氯气
      \task 氯化氢
      \task 硫化氢
      \task 二氧化硫
    \end{tasks}
    \item 浓硫酸和稀硫酸的性质有哪些不同的地方?
    \item 回答下列问题:
    \begin{tasks}
      \task 
      \task 
      \task 
    \end{tasks}
    \item
    \item
    \item
    \item
    \item 在下列反应里,二氧化硫是氧化剂还是还原剂,为什么?
    \begin{tasks}
      \task \ce{2H2S + SO2 $\xlongequal{\quad}$ 3S +2H2O}
      \task \ce{Br2 + SO2 +2H2O $\xlongequal{\quad}$ H2SO4 + 2HBr}
    \end{tasks}
    \item 写出下列物质变化的化学方程式(其中的离子反应还要写出相应的离子方程式):
    \[ \ce{FeS2} \to \ce{SO2} \to \ce{SO3} \to \ce{H2SO4} \to \ce{CuSO4} \to \ce{Cu} \]
  \end{question}
\end{Review}
\chapter{碱金属}
我们已经学习了卤族和氧族元素的知识,对非金属元素有了一些认识。
在这一章里,我们将要学习一族叫做碱金属的金属元素。
我们已经知道锂和钠的原子结构在最外电子层都只有 1 个电子。
具有相似结构的还有钾等四种元素。
碱金属包括锂、钠、钾、铷、铯、钫六种元素,因为它们的氧化物的水化物是可溶于水的碱,所以统称为碱金属。
本章主要介绍钠及其化合物的知识。

\section{钠}
\subsection{钠的物理性质}
\begin{Experiment}
  取一块金属钠,用刀切去一端的外皮,观察钠的颜色。
\end{Experiment}
金属钠很软,可以用刀切割。切开外皮后,可以看到钠的 “真面目”呈银白色,具有美丽的光泽。

钠是热和电的良导体,密度 \qty{0.97}{g/cm^3},比水还轻,能浮在水面上,熔点 \qty{97.81}{\celsius},沸点 \qty{882.9}{\celsius}。
\subsection{钠的化学性质}
钠的化学性质非常活泼。

\subsubsection{钠跟氧气的反应}
\begin{Experiment}
  用刀切开一小块钠,观察在光亮的断面上所发生的变化。把小块钠放在燃烧匙里加热,观察发生的变化。
\end{Experiment}
钠很容易氧化,在常温下就能够跟空气里的氧气化合而生成氧化物。
切开的光亮的金属钠断面很快地发暗,主要是因为生成了一薄层氧化物的缘故。
钠受热以后能够在空气里着火燃烧,在纯净的氧气里燃烧得更为剧烈,燃烧时发出黄色的火焰。
在钠跟氧气化合的过程中,可以生成氧化钠,氧化钠不稳定,继续氧化,生成过氧化钠。
过氧化钠比较稳定,所以钠在空气里燃烧,生成的是过氧化钠。
\[ \ce{2Na + O2 \xlongequal{\quad} Na2O2} \]

\subsubsection{钠跟硫等非金属的反应}
钠除了能跟氯气直接化合外,还能跟很多其它非金属直接化合,如跟硫化合时甚至发生爆炸,生成硫化钠。
\[ \ce{2Na + S \xlongequal{\quad} Na2S} \]

\subsubsection{钠跟水的反应}
钠跟水能起剧烈的反应。
\begin{Experiment}*[righthand ratio=0.5]
向一个盛有水的烧杯里,滴入几滴酚酞溶液。然后把一小块钠(约等于 1/2 豌豆那么大小)投入烧杯里。注意观察钠跟水起反应的情形和溶液颜色的变化。再用铝箔包好一小块钠,并在铝箔上刺些小孔,用镊子夹住,放在试管口下面,用排水法收集气体(\cref{fig:4-1})。小心地取出试管,移近火焰,检验试管里是不是收集了氢气。
  \tcblower
  \begin{figurehere}
    \caption{钠跟水起反应}\label{fig:4-1}
  \end{figurehere}
\end{Experiment}
投入烧杯里的钠比水轻,浮在水面上。
钠跟水起反应放出的热,立刻使钠熔成一个闪亮的小球。
小球向各个方向迅速游动,并逐渐缩小,最后完全消失。
钠跟水起反应后,使滴有酚酞溶液的水由无色变为红色。
这个现象说明有别的物质生成,这种生成物就是氢氧化钠。
试管里收集到的气体是氢气。
\[ \ce{2Na + 2H2O \xlongequal{\quad} 2NaOH +H2 ^}\]

由于钠很容易跟空气里的氧气或水起反应,所以通常保存在煤油里,跟空气和水隔绝。

\subsection{钠的存在}
钠的性质很活泼,所以它在自然界里不能以游离态存在,只能以化合态存在。
钠的化合物在自然界里分布很广,主要以氯化钠的形式存在,也以硫酸钠、碳酸钠和硝酸钠等形式存在。

\subsection{钠的制备和用途}
工业上,可以把直流电通入熔融的氯化钠来制取钠。

钠可以用来制取过氧化钠等化合物。
钠和钾的合金(含 50\%~80\% 钾)在室温下呈液态,是原子反应堆的导热剂。
钠是一种很强的还原剂,可以把钛、锆、铌、钽等金属从它们的熔融卤化物里还原出来。
钠也应用在电光源上。
高压钠灯发出的黄光射程远,透雾能力强,对道路平面的照度比高压水银灯高几倍。

\begin{Practice}[习题]
  \begin{question}
    \item 金属的应该怎样保存?为什么?
    \item 使 \qty{0.2}{mol} 钠跟水起反应,能生成多少升的氢气(标准状况)?
    \item 钠着火时,应选用下列哪种物质和器材灭火?为什么?
    \begin{tasks}(3)
      \task 水
      \task 泡沫灭火器
      \task 干粉灭火器
    \end{tasks}
  \end{question}
\end{Practice}

\section{钠的化合物}
\subsection{钠的氧化物}
钠的氧化物有氧化钠和过氧化钠等。
氧化钠是白色的固体,跟水起剧烈的反应,生成氢氧化钠。
\[ \ce{ Na2O + H2O \xlongequal{\quad} 2NaOH }\]

过氧化钠是淡黄色的固体,也能跟水起反应,生成氢氧化钠和氧气。
\begin{Experiment}
把水滴入盛有过氧化钠固体的试管,用带火星的木条放在管口,检验有没有氧气放出。
\end{Experiment}
\[ \ce{ 2Na2O2 + 2H2O \xlongequal{\quad} 4NaOH + O2 ^ }\]

过氧化钠是强氧化剂,可以用来漂白织物、麦秆、羽毛等等。

过氧化钠跟二氧化碳起反应,生成碳酸钠和氧气。
\[ \ce{ 2Na2O2 + 2CO2 \xlongequal{\quad} 2Na2CO3 + O2 ^ }\]

因此,它用在呼吸面具上和潜水艇里作为氧气的来源。

\subsection{钠的其他重要化合物}
我们在初中学过一种重要的钠的化合物——氢氧化钠。
下面简单介绍几种重要的钠盐。
\subsubsection{硫酸钠}

硫酸钠晶体俗名芒硝(\ce{Na2SO4.10H2O})。
硫酸钠是制玻璃和造纸(制浆)的重要原料,也用在染色、纺织、制水玻璃等工业上,在医药上用作缓泻剂。
自然界里的硫酸钠主要分布在盐湖和海水里。
我国盛产芒硝。

\subsubsection{碳酸钠和碳酸氢钠}
碳酸钠(\ce{Na2CO3})俗名纯碱或苏打,是白色粉末。
碳酸钠通常情况含有结晶水 (\ce{Na2CO3.10H2O})。
在空气里碳酸钠晶体很容易失去结晶水,表面失去光泽而逐渐发暗,并渐渐碎裂成粉末。
失水以后的碳酸钠叫做无水碳酸钠。
碳酸氢钠(\ce{NaHCO3})俗名小苏打,是一种细小的白色晶体。
碳酸钠较碳酸氢钠容易溶解于水。

碳酸钠和碳酸氢钠遇到盐酸都能放出二氧化碳。
\begin{gather*}
\ce{Na2CO3 + 2HCl \xlongequal{\quad} 2NaCl + H2O + CO2 ^} \\
\ce{NaHCO3 + HCl \xlongequal{\quad} NaCl + H2O + CO2 ^} 
\end{gather*}
\begin{Experiment}
把少量盐酸分别加入盛着碳酸钠和碳酸氢钠的两个试管里。
比较它们放出二氧化碳的快慢程度。
\end{Experiment}
碳酸氢钠遇到盐酸放出二氧化碳的作用,要比碳酸钠剧烈得多。

碳酸钠很稳定,受热很难分解,碳酸氢钠却不很稳定,受热容易分解。
\begin{Experiment}*[righthand ratio=0.5]
用\cref{fig:4-2} 的装置,把碳酸钠放入试管里,约占试管容积的 1/6,并往烧杯里倒入石灰水。
加热,观察澄清的石灰水是否起变化。
把试管拿掉,换上一个放入同样容积碳酸氢钠的试管。
再加热,观察澄清的石灰水所起的变化。
\tcblower
\begin{figurehere}
  \caption{鉴别碳酸钠和碳酸氢钠}\label{fig:4-2}
\end{figurehere}
\end{Experiment}
碳酸钠受热没有变化,而碳酸氢钠受热分解,放出二氧化碳。
\[ \ce{ 2NaHCO3 \xlongequal{\triangle} Na2CO3 + H2O + CO2 ^} \]

这个反应可以用来鉴别碳酸钠和碳酸氢钠。

碳酸钠是化学工业的重要产品之一,有很多用途。
它广泛地用在玻璃、制皂、造纸、纺织等工业上,也可以用来制造其它钠的化合物。
日常生活里也常用它作洗涤剂。
碳酸氢钠是焙制糕点所用的发酵粉的主要成分之一。
在医疗上,它是治疗胃酸过多的一种药剂。

碳酸钠有天然产出的。碱性土壤里和某些盐湖里常含有碳酸钠。
我国内蒙古自治区一带的盐湖就出产大量的天然碱。
\begin{Practice}[习题]
  \begin{question}
    \item 在呼吸面具里有时用到过氧化钠,这利用了它的什么性质?
    \item 怎样断定某种碳酸钠粉末里是否含有碳酸氢钠?怎样把混在碳酸钠里的碳酸氢钠除去?
    \item 写出下列各物质间转化的化学方程式(其中离子反应还要写出相应的离子方程式)。
    \[ \ce{Na} \to \ce{NaOH} \to \ce{NaCl} \to \ce{Na2SO4}\]
    \item 空气里通常含有 0.05\% \ce{CO2}(质量百分比),计算 \qty{10}{g} 的过氧化钠能够吸收多少升空气(标准状况)里的二氧化碳。
    \item 加热 \qty{410}{g} 小苏打到再没有气体放出时,剩余的物质是什么?它的质量是多少克?
    \item 把碳酸钠和碳酸氢钠的混和物 \qty{146}{g} 加热到质量不再继续减少为止。剩下的残渣的质量是 \qty{137}{g}。计算这混和物里含有百分之几的碳酸钠?
  \end{question}
\end{Practice}

\section{碱金属元素}
\subsection{碱金属元素的原子结构和碱金属的物理性质}
碱金属元素在自然界里都以化合态存在,它们的单质由人工制得。
碱金属除铯略带金色光泽外,都呈银白色,碱金属都比较柔软,有展性,它们的密度较小,熔点较低,铯在气温稍高的时候,就呈液态。
它们的导热、导电的性能都很强。
碱金属,特别是锂、钠、钾,是金属中比较轻的。
\cref{tab:4-1} 列出各元素的原子结构和物理性质。

从\cref{tab:4-1} 可以看出,锂、钠、钾、铷、铯的原子的最外电子层的电子数是相同的,都是一个电子。
这个电子对于原子的大小是有影响的,一旦这个电子失去而变成离子,离子就显著地比原子小了。
这可以从\cref{fig:4-3} 清楚地看到。
\begin{table}
  \caption{碱金属的原子结构和物理性质}\label{tab:4-1}
\end{table}
\begin{figure}
  \caption{碱金属的原子和离子的大小示意图(数据单位是 \qty{e-10}{m})}\label{fig:4-3}
\end{figure}

碱金属原子的原子半径\footnote{锂、钠、钾等金属的原子半径是指固态金属里两个临近原子核间的距离之半。}或离子半径一般都随着电子层数的增多而增大,这是跟卤素和氧族元素的原子的变化相一致的。
碱金属的熔点、沸点一般随着原子的电子层数的增加而降低。

\subsection{焰色反应}
我们在炒菜的时候,偶有食盐或食盐水溅在煤气火焰或煤火上,火焰就呈现黄色。
火焰呈现颜色的现象应用在科学实验上,可以检验一些金属或金属化合物。
多种金属或它们的化合物在灼烧时使火焰呈特殊的颜色,这在化学上叫做\Concept{焰色反应}。
\begin{Experiment}*
把装在玻璃棒上的铂丝(也可用光洁无锈的铁丝或镍、铬、钨丝)放在酒精灯火焰(最好用煤气灯,火焰的本身颜色较微弱)里灼烧,等到跟原来的火焰颜色相同的时候,用铂丝蘸碳酸钠溶液,放在火焰上,就可以看到火焰呈黄色(\cref{fig:4-4})。
每次试完后都要用稀盐酸洗净铂丝,在火焰上灼烧到没有什么颜色,再分别蘸碳酸钾溶液、碳酸锂溶液作试验,观察火焰的颜色。
在观察钾的火焰颜色的时候,要透过蓝色的钴玻璃去观察,这样就可以滤去黄色的光,避免碳酸钾里钠的杂质所造成的干扰。
  \tcblower
  \begin{figurehere}
    \caption{焰色反应试验的操作}\label{fig:4-4}
  \end{figurehere}
\end{Experiment}
碱金属和它们的化合物都能使火焰呈现出不同的颜色,即呈现焰色反应。
此外,钙、锶、钡等金属也能呈现焰色反应。
根据焰色反应所呈现的特殊颜色,可以测定金属或金属离子的存在。
下面列出各金属或金属离子的焰色反应的颜色:
\begin{description}
  \item[锂] 紫红色
  \item[钠] 黄色
  \item[钾] 浅紫色(透过蓝色钴玻璃)
  \item[铷] 紫色
  \item[钙] 砖红色
  \item[锶] 洋红色
  \item[钡] 黄绿色
  \item[铜] 绿色
\end{description}

在节日晚上燃放的五彩缤纷的焰火,其中就有碱金属和锶、钡等金属的化合物所呈现的各种鲜艳色彩。

\subsection{碱金属的化学性质}
我们知道,钠的化学性质很活泼。
它的原子的最外电子层是一个电子,在化学反应中容易失去。
锂、钾、铷、铯等原子的最外电子层都是一个电子,都容易失去,因此它们的化学性质都很活泼。
失去电子是氧化反应,所以碱金属是强还原剂。

\subsubsection{跟非金属的反应}
碱金属跟卤素的反应,有的是很剧烈的,这我们已经知道了。

其它碱金属都象钠一样能跟氧气起反应。锂跟氧气起反应,生成氧化锂。
\[ \ce{ 4Li+ O2 \xlongequal{\quad} 2 Li2O} \]

钾、铷等跟氧气起反应,生成比过氧化物更复杂的氧化物。

碱金属能够跟大多数的非金属起反应,表现出很强的金属性。

\subsubsection{跟水的反应}
碱金属都跟水起反应,生成氢氧化物并放出氢气。这类氢氧化物都能使酚酞溶液变红色。钾跟水的反应比钠更剧烈,常使生成的氢气燃烧,并发生轻微爆炸。
\begin{Experiment}
从煤油里取出一块金属钾,放在干燥玻璃片上,用滤纸吸干煤油,切取象绿豆大小那样的一块钾,放在装冷水的烧杯里,迅速用玻璃片盖好,以免因轻微爆炸而飞溅出液体来。
反应完成后滴入几滴酚酞溶液,观察溶液颜色的变化。
\end{Experiment}

\[ \ce{2K + 2H2O \xlongequal{\quad} 2KOH +H2 ^} \]

这个反应就是钾原子失去一个电子,水里的氢离子获得一个电子成为氢原子,氢原子构成氢分子。

在这几种碱金属中,由于原子的电子层数不同,核对层数越多的电子的吸引力越小,电子就越容易失去。
随着原子的电子层数增加,原子半径的增大,碱金属的活动性增强。
以钠和钾为例,钾跟氧气、跟水的反应都比钠刷烈,这些事实都可说明原子结构跟性质的关系。

\subsection{锂、钾、铷、铯的用途}
碱金属在生产和现代科学技术上都有一定的用途。
锂用以制备有机化学工业上的催化剂、多种合金、高强度玻璃等。
锂还用于制热核反应的材料氚。
钾的化合物象 \ce{KCl}、\ce{K2SO4} 等是重要的肥料。
铷和铯因在普通光的照射下能够放出电子,用于制光电管等。

钾的许多重要化合物,如氯化钾、硫酸钾、碳酸钾等都是钾肥。
我们在初中化学里已学过钾肥的初步知识。
土壤里钾的含量并不少,但大部分以钾的矿物形式存在,例如,正长石、 白云母\footnote{正长石:\ce{KAlSi3O8};白云母:\ce{KH2Al3Si3O12}。}等等。
这些矿物难溶于水,作物不能利用,只有在长期风化 (在土壤里受到空气、水分、酸的作用) 过程中,才能逐步转化为作物可以吸收的水溶性的钾的化合物。
因此,土壤里的钾常常不能满足作物生长的需要,人们往往要施用钾肥加以补充。

通常施用的钾肥主要是各种钾盐,如氯化钾、硫酸钾、碳酸钾(草木灰的主要成分)等。
这些钾盐都易溶于水,在溶液里钾以离子形式存在,易被作物吸收,所以,这些钾肥都是速效的。
但必须注意的是,由于它们易溶于水,在施用时要防止雨水淋失。

在科学种田、夺取高产的过程中,施用钾肥时,要因地制宜,注意氮、磷、钾三种肥料的合理配合。

\begin{Practice}[习题]
  \begin{question}
    \item 试比较钠和钾的物理性质和化学性质。
    \item 在卤族元素和碱金属元素中,哪一族元素的原子比它们相应的离子小,哪一族元素的原子比它们相应的离子大?试举例说明。
    \item 写出下列反应的化学方程式。
    \begin{tasks}
      \task \ce{ K2O +  H2O -> }
      \task \ce{ K2O2 + H2O -> }
      \task \ce{ Li2O + H2O -> }
    \end{tasks}
    \item 用电子得失的观点来说明下列氧化—还原反应。
    \begin{tasks}
      \task \ce{ 2K +  Cl2 $\xlongequal{\quad}$ 2KCl }
      \task \ce{ 2K + 2H2O $\xlongequal{\quad}$ 2KOH + H2 ^ }
    \end{tasks}
    \item 把 \qty{4}{g} 氢氧化钠溶解在水里,制成溶液。这溶液能跟多少毫升的密度为 \qty{1.19}{g/cm^3} 的盐酸起反应?
  \end{question}
\end{Practice}
\section*{内容提要}
碱金属是一族金属元素,它们的原子结构的共同之点是次外层有 8 个电子(锂是 2 个)和最外电子层都只有一个电子,在化学反应中容易失去电子,因此,它们的化学性质基本相似;差别之处是核电荷数不同,电子层数不同,原子半径也不同,因而碱金属元素的性质既相似又有差别。

碱金属的化学性质主要是强的金属性,随着原子半径的增大而金属性增强。
它们的单质都是强还原剂。

碱金属的化学性质:
\begin{itemize}
  \item 跟卤素的反应——生成卤化物。
  \item 跟氧气的反应——生成氧化物、过氧化物等等。
  \item 跟水的反应——生成氢氧化物,放出氢气。
\end{itemize}

碱金属和它们的化合物能使火焰呈现不同的颜色,即呈现焰色反应。
根据焰色反应所呈现的特殊颜色,可以判断某些金属或金属离子的存在。

\begin{Review}
  \begin{question}
    \item 回答下列问题:
    \begin{tasks}
      \task 使用金属钠
      \task 使用金属钠
      \task 使用金属钠
      \task 使用金属钠
    \end{tasks}
    \item 
    \item 选择正确的答案填写在括号里。
    \begin{enumerate}[itemindent=1.7em]
      \item 下列哪些物质可用来制氧气:\hfill(\hspace{2em})
      \begin{tasks}(5)
        \task \ce{Na2O2}
        \task \ce{CaCO3}
        \task \ce{KClO3}
        \task \ce{H2SO4}
        \task \ce{H2O}
      \end{tasks}
      \item 钠离子\hfill(\hspace{2em})
      \begin{tasks}(2)
        \task 遇水放出氢气,
        \task 要保存在煤油里,
        \task 比钠原子多一个电子,
        \task 在无色火焰上灼烧显黄色。
      \end{tasks}
      \item 下列哪些物质放在水里后,溶液显碱性:\hfill(\hspace{2em})
      \begin{tasks}(4)
        \task \ce{Na2O2}
        \task \ce{NaCl}
        \task \ce{CuO}
        \task \ce{K}
      \end{tasks}
    \end{enumerate}
    \item 写出下列物质转化的化学方程式(其中的离子反应还要写出相应的离子方程式)。
    \[ \ce{Na} \to \ce{Na2O2} \to \ce{NaOH} \to \ce{Na2CO3} \to \ce{CaCO3} \to \ce{Ca(HCO3)2}\]
  \end{question}
\end{Review}
\chapter{原子结构\texorpdfstring{\quad}{ }元素周期律}
到目前为止,我们已经学习了氧、惰性气体、氢、碳、卤族、氧族、碱金属等元素和它们的一些化合物,知道元素、化合物的性质跟它们的结构有密切的关系。
只有了解了它们的结构,才能深刻地认识它们的性质和变化规律。
所以,我们要在初中学习的物质结构初步知识的基础上,进一步学习物质结构的知识。
本章我们先来学习有关原子结构和反映元素内在联系的元素周期律的知识。
至于物质结构的其它知识,以后将要逐步学习。

\section{原子核}
\subsection{原子核}
原子是由居于原子中心的带正电的原子核和核外带负电的电子构成的。
由于原子核带的电量跟该外电子的电量相等而电性相反,因此,原子作为一个整体不显电性。
原子很小,而原子核更小,它的半径约是原子的万分之一,它的体积只占原子体积的几千亿分之一。
原子核由质子和中子构成。
每个质子带一个单位正电荷,中子呈电中性,因此,核电荷数由质子数决定。
核电荷数的符号为 $Z$。
\[ \text{核电荷数}(Z) = \text{核内质子数} = \text{核外电子数}\]

质子的质量为 \qty{1.6726e-27}{kg},中子的质量稍大些,为 \qty{1.6748e-27}{kg},电子的质量很小,仅约为质子质量的 1/1836,所以,原子的质量主要集中在原子核上。
由于质子、 中子的质量很小,计算不方便,因此,通常用它们的相对质量。

通过科学实验测得,作为原子量标准的那种碳原子的质量是 \qty{1.9927e-26}{kg},它的 1/12 为 \qty{1.6606e-27}{kg}。
质子和中子对它的相对质量分别为 1.007 和 1.008,取近似整数值为 1。
显然,如果忽略电子的质量,将原子核内所有的质子和中子的相对质量取近似整数值加起来,所得的数值,叫做质量数,用符号 $A$ 表示。中子数用符号 $N$ 表示。则
\[ \text{质量数}(A)= \text{质子数}(Z)+\text{中子数}(N)\]

因此,只要知道上述三个数值中的任意两个,就可以推算出另一个数值来。
例如,知道硫原子的核电荷数为 16,质量数为 32,则
\[\text{硫原子的中子数}=A-Z= 32-16=16\]

归纳起来,如以 \ce{^$A$_$Z$ $X$} 代表一个质量数为 $A$、质子数为 $Z$ 的原子,那么,组成原子的粒子间的关系可以表示如下:
\[ \text{原子}\; \ce{^$A$_$Z$ $X$} 
     \left\{ \begin{array}{l}
       \text{原子核} \left\{
        \begin{array}{ll}
          \text{质子} & Z \text{个}\\
          \text{中子} & (A-Z) \text{个}\\
        \end{array}
       \right.\\
       \text{核外电子}\quad Z \text{个}    
    \end{array}
     \right.
\]

\subsection{同位素}
我们已经知道,具有相同核电荷数 (即质子数) 的同一类原子叫做元素。
也就是说,同种元素的原子的质子数相同,那么,它们的中子数是否相同呢? 
科学研究证明,不一定相同。
例如,氢元素的原子都含 1 个质子,但有的氢原子不含中子,有的氢原子含 1 个中子,还有的氢原子含 2 个中子:
\begin{itemize}[label={},leftmargin=2em]
  \item 不含中子的氢原子叫做氕;
  \item 含 1 个中子的氢原子叫做氘,就是重氢;
  \item 含 2 个中子的氢原子叫做氚\footnote{氕音\pinyin{pie1},氘音\pinyin{dao1},氚音\pinyin{chuan1}。},就是超重氢。
\end{itemize}

为了便于区别,将氕记为 \ce{^1_1H},氘记为 \ce{^2_1H}(或 \ce{D}),氚记为  \ce{^3_1H}(或 \ce{T})。
元素符号的左下角记核电荷数,左上角记质量数。

人们把原子里具有相同的质子数和不同的中子数的同一元素的原子互称同位素。
许多元素都有同位素。上述 \ce{^1_1H}、\ce{^2_1H}、\ce{^3_1H} 是氢的三种同位素,\ce{^2_1H}、\ce{^3_1H} 是制造氢弹的材料。
铀元素有 \ce{^234_92U}、\ce{^235_92U}、\ce{^238_92U} 等多种同位素,\ce{^235_92U} 是制造原子弹的材料和核反应堆的燃料。
碳元素有 \ce{^12_6C}、\ce{^13_6C} 和 \ce{^14_6C} 等几种同位素,而 \ce{^12_6C} 就是我们将它的质量的 1/12 当做原子量标准的那种碳原子。
同一元素的各种同位素虽然质量数不同,但它们的化学性质几乎完全相同。
在天然存在的某种元素里,不论是游离态还是化合态,各种同位素所占的原子百分比一般是不变的。
我们平常所说的某种元素的原子量,是按各种天然同位素原子所占的一定百分比算出来的平均值。
例如,元素氯是 \ce{^35_17Cl} 和  \ce{^37_17Cl} 两种同位素的混和物,从下列数据即可计算出氯元素的原子量:

\begin{tabular}{ccc}
  符号 & 同位素的原子量 & 在自然界各同位素原子的百分组成 \\
  \ce{^35_17Cl} & 34.969 & 75.77 \\
  \ce{^37_17Cl} & 36.966 & 24.23 \\
\end{tabular}
\[ 36.969\times 0.7577 + 36.966 \times 0.2423 = 35.453\]
即氯的原子量为 35.453。

同理,根据同位素的质量数,也可以算出近似原子量。
\begin{Practice}[习题]
  \begin{question}
    \item 下列说法是否正确? 如有错误,加以改正。
    \begin{tasks}
      \task 石墨和金刚石是由碳元素组成的两种同位素。
      \task 人们已经知道了 107 种元素,就是说人们已经知道了 107 种原子。
    \end{tasks}
    \item 指出下列各原子中质子、中子、电子的数目各是多少:
    \begin{gather*}
      \ce{^12_6C},\quad\ce{^13_6C},\quad\ce{^16_8O},\quad\ce{^17_8O},\quad\ce{^18_8O},\quad\ce{^19_9F},\quad\ce{^24_12Mg}\\
      \ce{^39_19K},\quad\ce{^40_19K},\quad\ce{^41_19K},\quad\ce{^40_20Ca},\quad\ce{^42_20Ca}
    \end{gather*}
    \item 氧有三种天然同位素,它们的同位素原子量和各同位素原子的百分组成数据分列如下:\par
    \begin{tabular}{ccc}
      \ce{^16_8O} & 15.994915 & 99.759\% \\
      \ce{^17_8O} & 16.999133 &  0.037\% \\
      \ce{^18_8O} & 17.99916  &  0.204\% \\
    \end{tabular}

    \noindent 计算氧元素的原子量。
    \item 镁有三种天然同位素: \ce{^24_12Mg}(占 78.7\%),\ce{^25_12Mg}(占 10.13\%),\ce{^26_12Mg}(占 11.17\%),计算镁元素的近似原子量。
  \end{question}
\end{Practice}

\section{核外电子的运动状态}
电子带负电荷,质量很小,仅 \qty{9.1095e-31}{kg}。
它在原子这样大小的空间(直径约 \qty{e-10}{m})内运动,速度很快,接近光速(\qty{3e8}{m/s})。
电子的运动情形跟质量大、速度小的普通物体是否相同?
有没有特殊的规律?
现在我们就来进行研究。

\subsection{电子云}
我们在生活中见到汽车在公路上奔驰,用仪器观察到人造卫星按一定轨道围绕地球旋转,都可以测定或根据一定的数据计算出它们在某一时刻所在的位置,并描画出它们的运动轨迹。
但是,核外电子的运动规律就跟上述普通物体不同。
核外电子的运动没有上述那样确定的轨道,我们不能测定或计算出它在某一时刻所在的位置,也不能描画它的运动轨迹。
我们在描述核外电子运动时,只能指出它在原子核外空间某处出现机会的多少。
电子在核外空间一定范围内出现,好象带负电荷的云雾笼罩在原子核周围,所以我们形象地称它为“电子云”。
为了便于理解,我们用给氢原子照像的比喻来加以说明。
我们知道,氢原子核外有一个电子。
为了在一瞬间找到电子在氢原子核外的确定位置,我们假想有一架特殊的照相机,可以用它来给氢原子照相。
先给某个氢原子拍五张照片,得到如\cref{fig:5-1} 所示的不同的图象。
\begin{figure}
  \begin{minipage}{0.19\linewidth}\centering
    \subcaption{}\label{fig:5-1a}
  \end{minipage}
  \begin{minipage}{0.19\linewidth}\centering
    \subcaption{}\label{fig:5-1b}
  \end{minipage}
  \begin{minipage}{0.19\linewidth}\centering
    \subcaption{}\label{fig:5-1c}
  \end{minipage}
  \begin{minipage}{0.19\linewidth}\centering
    \subcaption{}\label{fig:5-1d}
  \end{minipage}
  \begin{minipage}{0.19\linewidth}\centering
    \subcaption{}\label{fig:5-1e}
  \end{minipage}
  \caption{氢原子的五次瞬间照相}\label{fig:5-1}
\end{figure}
\cref{fig:5-1} 里 $\oplus$ 表示原子核,一个小黑点表示电子在这里出现过一次。
然后继续给氢原子拍上成千上万张照片,并把这些照片一一对比研究,这样,我们就获得一个印象: 电子好象是在氢原子核外作毫无规律的运动,一会儿在这里出现,一会儿在那里出现。
如果我们将这些照片叠印,就会看到如\cref{fig:5-2} 所示的图象。
图象说明,对氢原子的照片叠印张数越多,就越能使人形成一团电子云雾笼罩原子核的印象,而这团 “电子云雾” 呈球形对称,在离核越近处密度越大,离核越远处密度越小。
也就是说,在离核越近处单位体积的空间中电子出现的机会越多,离核越远处单位体积的空间中电子出现的机会越少。
实际上,\cref{fig:5-2d} 就是在通常状况下氢原子的电子云示意图。
\begin{figure}
  \begin{minipage}{0.24\linewidth}\centering
    \subcaption{5 张照片}\label{fig:5-2a}
  \end{minipage}
  \begin{minipage}{0.24\linewidth}\centering
    \subcaption{20 张照片}\label{fig:5-2b}
  \end{minipage}
  \begin{minipage}{0.24\linewidth}\centering
    \subcaption{100 张照片}\label{fig:5-2c}
  \end{minipage}
  \begin{minipage}{0.24\linewidth}\centering
    \subcaption{10000 张照片}\label{fig:5-2d}
  \end{minipage}
  \caption{将若干张氢原子瞬间照相叠印的结果}\label{fig:5-2}
\end{figure}

\subsection{核外电子的运动状态}
\subsubsection{电子层}
我们已经知道,在含有多个电子的原子里,电子的能量并不相同。
能量低的,通常在离核近的区域运动;能量高的,通常在离核远的区域运动。
根据电子的能量差异和通常运动的区域离核的远近不同,可以将核外电子分成不同的电子层。

我们怎么知道含有多个电子的原子里核外电子的能量并不相同呢?
根据对元素电离能数据的分析,可以初步得到这个结论。

什么是电离能?从气态原子(或气态阳离子)中去掉电子,把它变成气态阳离子(或更高价的气态阳离子),需要克服核电荷的引力而消耗能量,这个能量叫做\Concept{电离能},符号为 $I$,单位常用电子伏特\footnote{电子伏特(\unit{eV})是一个电子在真空中通过 \qty{1}{V} 电位差所获得的动能,它是一种描述微观粒子运动的能量单位。$\qty{1}{eV}=\qty{1.6022e-19}{J}$}。

从元素的气态原子去掉一个电子成为 $+1$ 价气态阳离子所需消耗的能量,称为第一电离能($I_1$),从 $+1$ 价气态阳离子再去掉一个电子成为 $+2$ 价气态阳离子所需消耗的能量,叫做第二电离能($I_2$);依次类推。

\cref{tab:5-1} 列出了几种元素电离能的数据。
\begin{table}
  \caption{几种元素的电离能(\unit{eV})}\label{tab:5-1}
  \begin{tblr}{
    colspec={cc*{9}{X[r]}},hline{2}=0.8pt,row{1}={m,c},
    hline{3}={4-4}{1.2pt},vline{4}={2-2}{1.2pt},
    hline{4}={5-5}{1.2pt},vline{5}={3-3}{1.2pt},
    hline{5}={6-6}{1.2pt},vline{6}={4-4}{1.2pt},
    hline{6}={7-7}{1.2pt},vline{7}={5-5}{1.2pt},
    hline{7}={8-8}{1.2pt},vline{8}={6-6}{1.2pt},
    hline{8}={9-9}{1.2pt},vline{9}={7-7}{1.2pt},
    vline{10}={8-8}{1.2pt}
    }
    {核电\\荷数} & {元素\\符号} & $I_1$ & $I_2$ & $I_3$ & $I_4$ & $I_5$ & $I_6$ & $I_7$ & $I_8$ & $I_9$ \\
    3 & \ce{Li} &  5.4 & 75.6 & 122.4 & &&&&&\\ 
    4 & \ce{Be} &  9.3 & 18.2 & 153.9 & 217.7 &&&&&\\ 
    5 & \ce{B}  &  8.3 & 25.1 &  37.9 & 259.3 & 340.1 &&&&\\ 
    6 & \ce{C}  & 11.3 & 24.4 &  47.9 &  64.5 & 392.0 & 489.8 &&&\\ 
    7 & \ce{N}  & 14.5 & 29.6 &  47.4 &  77.5 &  97.9 & 551.9 & 666.8 &&\\ 
    8 & \ce{O}  & 13.6 & 35.1 &  54.9 &  77.4 & 113.9 & 138.1 & 739.1 & 871.1 &\\ 
    9 & \ce{F}  & 17.4 & 35.0 &  62.6 &  87.1 & 114.2 & 157.1 & 185.1 & 953.6 & 1102 \\ 
  \end{tblr}
\end{table}

从表上数据可见,元素的第二电离能大于第一电离能,第三电离能大于第二电离能,依次类推,即 $I_1<I_2<I_3<\cdots$。 
这是容易理解的,因为从 $+1$ 价气态阳离子中去掉一个电子需克服的电性引力比从中性原子去掉一个电子要大,消耗的能量要多。
同理,从 $+2$ 价气态阳离子中去掉一个电子,需克服的电性引力,比从 $+1$ 价气态阳离子中去掉一个电子更大,消耗的能量更多。
因此,一个原子的电离能是依次增大,甚至是成倍增长的,但增大的倍数并不相同。
有的增大得不多,有的增大得很多。我们在\cref{tab:5-1} 上将增大倍数很多的电离能数据前面和下面标上粗线,以示区别。
下面就来分析这些数据。

\ce{Li},原子核外有 3 个电子。$I_3$ 比 $I_2$ 增大不到一倍,但 $I_2$ 比 $I_1$ 却增大了十几倍。这说明什么问题? 
说明这 3 个电子可分为两组,两组能量有差异。
$I_1$ 比 $I_2$、$I_3$ 小得多,说明有一个电子能量较高,通常在离核较远的区域运动,容易被去掉。
另外两个电子能量较低,通常在离核较近的区域运动。

\ce{Be},原子核外有 4 个电子。按照如上的分析,$I_2$ 比 $I_1$,$I_4$ 比 $I_3$ 均增大不到一倍,但 $I_3$ 比 $I_2$ 却增大了好几倍。
因此可以认为有两个电子能量较低,通常在离核较近的区域运动;另外两个电子能量较高,通常在离核较远的区域运动。

分析 \ce{B}、\ce{C}、\ce{N}、\ce{O}、\ce{F} 等元素的电离能的数据,将会发现它们的核外电子都分两组,第一组是两个电子,能量较低,通常在离核较近的区域运动;第二组分别是 3、4、5、6、7 个电子,能量较高,通常在离核较远的区域运动。

如果分析其它元素的电离能数据,也会得出相似的结论。 
可见,在含多个电子的原子中,电子是分层排布的。

\subsubsection{电子亚层和电子云形状}
科学研究发现,在同一电子层中,电子的能量还稍有差别,电子云的形状也不相同。
根据这个差别,又可以把一个电子层分成一个或几个亚层,分别用 $s$、$p$、$d$、$f$ 等符号\footnote{$s$、$p$、$d$、$f$ 是光谱学上的符号}表示。
K 层只包含一个亚层,即 $s$ 亚层;L 层包含两个亚层,即 $s$ 亚层和 $p$ 亚层;M 层包括三个亚层,即 $s$、$p$、$d$ 亚层;N 层包括四个亚层,即 $s$、$p$、$d$、$f$ 亚层。
不同亚层的电子云形状不同。
$s$ 亚层的电子云是以原子核为中心的球形,$p$ 亚层的电子云是 纺锤形,$d$ 亚层、$f$ 亚层的电子云形状比较复杂,这里就不介绍了。

在同一个电子层里,亚层电子的能量是按 $s$、$p$、$d$、$f$ 的次序递增的。
为了清楚地表示某个电子处于核外哪个电子层和亚层(自然同时也表示它的能量高低和电子云的形状),可将电子层的序数 $n$ 标在亚层符号的前面。
如处于 K 层的 $s$ 亚层的电子标为 $1s$;处于 L 层的 $s$ 亚层和 $p$ 亚层的电子标为 $2s$ 和 $2p$ ;处于 M 层的 $d$ 亚层的电子标为 $3d$;处于 N 层的 $f$ 亚层的电子标为 $4f$。\cref{fig:5-3a} 就是氢的 $1s$ 电子云。
\begin{figure}
  \begin{minipage}{0.32\linewidth}\centering
    \subcaption{}\label{fig:5-3a}
  \end{minipage}
  \begin{minipage}{0.32\linewidth}\centering
    \subcaption{}\label{fig:5-3b}
  \end{minipage}
  \begin{minipage}{0.32\linewidth}\centering
    \subcaption{}\label{fig:5-3c}
  \end{minipage}
  \caption{氢原子的 $1s$ 电子云}\label{fig:5-3}
\end{figure}

\cref{fig:5-3b} 虚线表示的球壳称为电子云的界面。
在界面内电子出现的机会最多,界面外电子出现的机会很少。
通常也用电子云界面图来表示电子云。
\cref{fig:5-3c} 是氢原子 \({1s}\) 电子云的界面图,它把表示电子出现机会的小黑点略去了。

\subsubsection{电子云的伸展方向}
电子云不仅有确定的形状,而且有一定的伸展方向。$s$ 电子云是球形对称的,在空间各个方向上伸展的程度相同。
$2p$ 电子云如\cref{fig:5-4} 所示,在空间可以有三种互相垂直的伸展方向。
$d$ 电子云可以有五种伸展方向,$f$ 电子云可以有七种伸展方向。
\begin{figure}
  \begin{minipage}{0.32\linewidth}\centering
    \subcaption{}\label{fig:5-4a}
  \end{minipage}
  \begin{minipage}{0.32\linewidth}\centering
    \subcaption{}\label{fig:5-4b}
  \end{minipage}
  \begin{minipage}{0.32\linewidth}\centering
    \subcaption{}\label{fig:5-4c}
  \end{minipage}
  \caption{$2p$ 电子云的三种伸展方向}\label{fig:5-4}
\end{figure}

如果把在一定的电子层上,具有一定的形状和伸展方向的电子云所占据的空间称为一个轨道,那么 $s$、$p$、$d$、$f$ 四个亚层就分别有 1、3、5、7 个轨道。这样,各电子层可能有的最多轨道数如下:

\begin{tabular}{rlr}
  电子层($n$)& 亚层 & 轨道数 \\
  $n=1$  & $s$                & $1=1^2$ \\
  $n=2$  & $s$、$p$           & $1+3=4=2^2$ \\
  $n=3$  & $s$、$p$、$d$      & $1+3+5=9=3^2$ \\
  $n=4$  & $s$、$p$、$d$、$f$ & $1+3+5+7=16=4^2$ \\
    $n$  &                    & $n^2$ \\
\end{tabular}

\noindent 即每个电子层可能有的最多轨道数应为 $n^2$。

\subsubsection{电子的自旋}
电子不仅在核外空间不停地运动,而且还作自旋运动。
电子自旋有两种状态,相当于顺时针和逆时针两种方向。
平常我们用向上箭头 $\uparrow$ 和向下箭头 $\downarrow$ 来表示不同的自旋状态。

通过以上的叙述我们可以看出,电子在原子核外的运动状态是相当复杂的,必须由它所处的电子层、电子亚层、电子云的空间伸展方向和自旋状态四个方面来决定。前三个方面跟电子在核外空间的位置有关,体现了电子在核外空间的运动状态,确定了电子的轨道。
因此,当我们要说明一个电子的运动状态时,必须同时指明它处于什么轨道和哪一种自旋状态。

\begin{Practice}[习题]
  \begin{question}
    \item 你如何理解电子层、电子亚层和电子云这三个概念?氢原子核外只有一个电子,为什么要用电子云来描述它的运动?
    \item 什么叫电离能?为什么根据元素电离能的变化可以判断核外电子是分层排布的?
    \item 原子核外电子的运动状态,必须从哪几个方面来进行描述?
    \item $1s$、$2p$、$3d$、$4f$ 各表示什么意思?
    \item $d$ 亚层和 $f$ 亚层各有多少轨道?在同一轨道上运动的电子可以有几种不同的运动状态?
    \item $2p_x$、$2p_y$、$2p_z$ 各表示什么意思?
  \end{question}
\end{Practice}

\section{原子核外电子的排布}
上一节我们学习了原子核外电子的运动状态,了解电子是分层排布的,而电子层又可分为几个电子亚层。现在,我们就来进一步讨论原子核外电子的排布规律。

\subsection[泡利不相容原理]{泡利\protect\footnote{泡利(Pauli,1900--1958),奥地利物理学家。}不相容原理}
我们先来讨论锂的核外电子排布。
锂原子有 3 个电子。 这 3 个电子是都在一个轨道上,还是分别在几个轨道上呢?
实验证明,有 2 个在 $1s$ 轨道上,1 个在 $2s$ 轨道上。
在 $1s$ 轨道上的 2 个电子自旋方向是平行的,还是相反的?
实验证明,是相反的。
在 $2s$ 轨道上的那个电子虽然自旋方向跟 $1s$ 轨道上的一个电子相同,但它们分别处于两个不同的轨道。
其它元素核外电子排布有类似的情况。

从以上建立在实验基础上的讨论中我们看到,在原子核外电子的排布中,排在同一轨道上的两个电子,自旋方向就相反; 而自旋方向相同的电子,必然处于不同的轨道上。
我们知道,一个轨道是由电子层、电子亚层和电子云的伸展方向三方面确定的,因此,可以得出一个结论: 在同一个原子里; 没有运动状态四个方面完全相同的电子存在。
这个结论是泡利提出来的,叫做\Concept{泡利不相容原理}。

根据这个原理,我们可以推算出各电子层可以容纳的最多电子数。
我们知道,每个电子层可能有的最多轨道数为 $n^2$,而每个轨道又只能容纳 2 个电子,因此,各电子层可能容纳的电子总数就是 $2n^2$。
现将 1~4 电子层可容纳电子的最大数目列于\cref{tab:5-2} 中。
\begin{table}
  \caption{1~4 电子层可容纳电子的最大数目}\label{tab:5-2}
  \begin{tblr}{colspec={X[6,c]*{10}{X[c]}}}
    电子层($n$)& \SetCell[c=1]{m,c}{K\\(1)}& \SetCell[c=2]{m,c}{L\\(2) }& & \SetCell[c=3]{m,c}{M\\(3)} & & & \SetCell[c=4]{m,c}{N\\(4)} & & & \\ 
    电子亚层     & $s$ & $s$ & $p$  & $s$ & $p$ & $d$ & $s$ & $p$ & $d$ & $f$ \\ 
    亚层中的轨道数     & 1 & 1 & 3  & 1 & 3 & 5 & 1 & 3 & 5 & 7 \\ 
    亚层中的电子数     & 2 & 2 & 6  & 2 & 6 & 10 & 2 & 6 & 10 & 14 \\ 
    每个电子层中可容纳电子的最大数目     & 2 & \SetCell[c=2]{m,c}8 &  & \SetCell[c=3]{m,c}18 &  &  & \SetCell[c=4]{m,c}32 &  &  &  \\ 
  \end{tblr}
\end{table}

\subsection{能量最低原理}
生活常识告诉我们,水总是由高处向低处流,山上的石头可以自动地向山下滚。
这是由于物体处于高势能状态时不如低势能状态稳定。
同理,在核外电子的排布中,通常状况下电子也总是尽先占有能量最低的轨道,只有当这些轨道占满后,电子才依次进入能量较高的轨道。
这个规律叫做\Concept{能量最低原理}。

那么,哪些轨道的能量高,哪些轨道的能量低呢?

我们知道,不同电子层具有不同的能量,而每个电子层中不同亚层的能量也不相同。
为了表示原子中各电子层和亚层电子能量的差异,人们把原子中不同电子层和亚层的电子按能量高低排成顺序,象台阶一样,叫做能级。
例如,$1s$ 能级,$2s$ 能级,$2p$ 能级,等等。
在一个原子中,离核越近、 $n$ 越小的电子层能量越低。
在同一电子层中,各亚层的能量是按 $s$、$p$、$d$、$f$ 的次序增高的。
因此,我们可以认为 $2s$ 能级高于 $1s$ 能级,$2p$ 能级高于 $2s$ 能级,等等。
可是对于那些核外电子数较多的元素来说,情况就比较复杂了。
为什么呢? 因为多电子原子的各个电子之间存在着排斥力,在研究某个外层电子的运动状态时,必须同时考虑到核对它的吸引力及其它电子对它的排斥力。
由于其它电子的存在,往往减弱了原子核对外层电子的吸引力,从而使多电子原子的电子所处的能级产生了交错现象。
\cref{fig:5-5} 是多电子原子电子的近似能级图,图上一个方框代表一个轨道。
\begin{figure}
  \caption{多电子原子电子的近似能级图}\label{fig:5-5}
\end{figure}

从\cref{fig:5-5} 可以看到,从第三电子层起就有能级交错现象,例如,$3d$ 电子的能量似乎应该低于 $4s$,而实际上 $E_{3d}>E_{4s}$。
按照能量最低原理,电子在进入核外电子层时,不是排完了 $3p$ 就排 $3d$,而是先排 $4s$。排完了 $4s$,才排 $3d$。

应用多电子原子电子的近似能级图,并根据能量最低原理,就可以确定电子排入各轨道的次序,如\cref{fig:5-6} 所示。
\begin{figure}
  \caption{电子填入轨道的顺序}\label{fig:5-6}
\end{figure}

\subsection[洪特规则]{洪特\protect\footnote{洪特(Hund,1896--1997),德国物理学家。}规则}
我们运用泡利不相容原理和能量最低原理,再来讨论碳、氮、氧三种元素原子的核外电子的排布情况。

碳元素的核电荷数为 6,即核外有 6 个电子。
根据上述两个原理,核外电子首先在 $1s$ 轨道排入两个自旋方向相反的电子,然后另 2 个自旋方向相反的电子排入 $2s$ 轨道,还剩 2 个电子,应排入 $2p$ 轨道。
$2p$ 轨道有 3 个,它们是以自旋方向相反的方式排入一个 $2p$ 轨道,还是以自旋方向相同的方式排入两个 $2p$ 轨道呢?
人们从科学实验中总结出的叫做\Concept{洪特规则}的一条规律回答了这个问题,这个规则指出,在同一亚层中的各个轨道(如 3 个 $p$ 轨道,或 5 个 $d$ 轨道,或 7 个 $f$ 轨道)上,电子的排布尽可能分占不同的轨道,而且自旋方向相同,这样排布整个原子的能量最低。
因此,碳、氮、氧三元素原子的电子层排布应该如\cref{fig:5-7} 所示。图中 \(\left| \frac{1s}{ \uparrow \downarrow }\right| \frac{2s}{\left( { \uparrow \downarrow }\right) \left| \uparrow \right| \uparrow }\frac{2p}{ \uparrow \uparrow }\) 叫做轨道表示式,一个方框表示一个轨道; 式子 $1s^22s^22p^2$ 叫做电子排布式,式中右上角的数字表示该轨道中电子的数目,如 $1s^2$ 表示在 $1s$ 轨道上有两个电子。
\begin{figure}
  \caption{碳、氮、氧原子的电子层排布}\label{fig:5-7}
\end{figure}

根据上述三个原理和多电子原子电子的近似能级图,我们将核电荷数为 1~36 的元素原子的核外电子的排布情况列入\cref{tab:5-3} 中。

\begin{table}
  \caption{核电荷数为 1~36 的元素的电子层排布}\label{tab:5-3}
  % \begin{tblr}{colspec={},}
  % \end{tblr}
\end{table}

从表上可以看出,核电荷数为 24 的元素 \ce{Cr},核电荷数为 29 的元素 \ce{Cu},它们的电子层结构并没有完全按照前述规律排布,\ce{Cr} 和 \ce{Cu} 在排了 $3p^6$ 后似应排成 $3d^44s^2$ 和 $3d^94s^2$,但实验数据表明应排成 $3d^54s^1$ 和 $3d^{10}4s^1$; 其它元素的电子层排布也有类似的情况。
根据这种情况,人们又归纳出一条规律,就是对于同一电子亚层,当电子排布为全充满、半充满或全空时,是比较稳定的。
即
\begin{description}
  \item[全充满] $p^6$ 或 $d^{10}$ 或 $f^{14}$ 
  \item[半充满] $p^6$ 或 $d^5$ 或 $f^7$ 
  \item[全 空] $p^6$ 或 $d^0$ 或 $f^0$ 
\end{description}
这是洪特规则的一种特例。上述 \ce{Cr}、\ce{Cu} 的电子层排布,就是属于 $d$ 轨道半充满、全充满时比较稳定的例子。

这里需要指出,核外电子的排布情况是通过实验测定的。
上面讲的泡利不相容原理、能量最低原理和洪特规则三条原理,是从大量事实中概括出来的,它们能帮助我们了解元素原子核外电子排布的规律,但不能用它们来解释有关电子排布的所有问题。
因此,这些原理只具有相对近似的意义。

\begin{Practice}[习题]
  \begin{question}
    \item 解释下列符号各代表什么意义:
    \item 用 $E$ 代表能量,把下列轨道按能量由低到高的顺序排列起来:
    \item N 电子层有哪几种轨道?轨道数共是多少?分别写出这些轨道的符号。
    \item 填空
    \item 解释下列事实:
    \begin{tasks}
      \task 核电荷数为 19 的元素 \ce{K} 的电子层排布为什么是 $1s^22s^22p^63s^23p^64s^1$,而不是 $1s^22s^22p^63s^23p^63d^1$?
      \task 核电荷数为 24 的元素 \ce{Cr} 的电子层排布为什么是 $1s^22s^22p^63s^23p^63d^54s^1$,而不是 $1s^22s^22p^63s^23p^63d^44s^2$?
    \end{tasks}
    \item 某元素 $2p$ 亚层上有 3 个电子; 这 3 个电子应该是按方式排布,还是按方式排布? 为什么?
    \item 某种元素的电子排布式是 $1s^22s^22p^63s^23p^63d^{10}4s^24p^6$ 说明它的原子Σ外有多少个电子层,各电子层有多少个电子,该元素的原子总共有多少个电子,核电荷数是几。
    \item 用电子排布式表示铝(核电荷数为 13)、氯(核电荷数为 17)、铁(核电荷数为 26)、铜(核电荷数为 29)和氪(核电荷数为 36)的电子层排布。
  \end{question}
\end{Practice}

\section{元素周期律}
从学初中化学到现在,我们已经学习了惰性气体、卤族、 氧族、碱金属几个元素族的知识,了解到一个自然族内的元素性质相似,而族跟族之间元素的性质不同。
这说明元素之间的关系存在着一定的规律。

为了认识元素间的这种规律性,我们将核电荷数为 1~18 的元素的核外电子排布、原子半径、第一电离能和主要化合价列成表(\cref{tab:5-4})来加以讨论。
为了方便,人们按核电荷数由小到大的顺序给元素编号,这种序号,叫做该元素的原子序数。
显然,原子序数在数值上与这种原子的核电荷数相等。
\cref{tab:5-4} 就是按原子序数的顺序编排的。
\begin{sidewaystable}
  \centering\small
  \caption{元素性质随着核外电子周期性排布而呈周期性的变化}\label{tab:5-4}
  \begin{tblr}{colspec={X[c]*{18}{c}},colsep=2pt,rowsep=10pt,row{4,6}={font=\scriptsize},cell{4,6}{1}={}{font=\small}}
    原子序数 & 1  & 2  & 3 & 4 & 5 & 6 & 7 & 8 & 9 & 10 & 11 & 12 & 13 & 14 & 15 & 16 & 17 & 18 \\
    元素名称 & 氢 & 氦 &锂 &铍 &硼 &碳 &氮 &氧 &氟 & 氖 & 钠 & 镁 & 铝 & 硅 & 磷 & 硫 & 氯 & 氩\\
    元素符号 & \ce{H}  & \ce{He}  & \ce{Li} & \ce{Be} & \ce{B} & \ce{C} & \ce{N} & \ce{O} & \ce{F} & \ce{Ne} & \ce{Na} & \ce{Mg} & \ce{Al} & \ce{Si} & \ce{P} & \ce{S} & \ce{Cl} & \ce{Ar} \\
    {最外层电\\子的排布} & $1s^1$  & $1s^2$  & $2s^1$ & $2s^2$ & $2s^22p^1$ & $2s^22p^2$ & $2s^22p^3$ & $2s^22p^4$ & $2s^22p^5$ & $2s^22p^6$ & $3s^1$ & $3s^2$ & $3s^23p^1$ & $3s^23p^2$ & $3s^23p^3$ & $3s^23p^4$ & $3s^23p^5$ & $3s^23p^6$ \\
    {原子半径\\(\qty{e-10}{m})} & 0.37  & 1.22  & 1.52 & 0.89 & 0.82 & 0.77 & 0.75 & 0.74 & 0.71 & 1.60 & 1.86 & 1.60 & 1.43 & 1.17 & 1.10 & 1.02 & 0.99 & 1.01 \\
    {第一\\电离能\\(\unit{eV})} & 13.595  & 24.481  & 5.39 & 9.32 & 8.296 & 11.256 & 14.53 & 13.614 & 17.418 & 21.559 & 5.138 & 7.644 & 5.984 & 8.149 & 10.484 & 10.357 & 13.01 & 15.755 \\
    化合价 & $+1$  & $0$  & $+1$ & $+2$ & $+3$ & {$+4$ \\ $-4$} & {$+5$ \\ $-3$} & $-2$ & $-1$ & $0$ & $+1$ & $+2$ & $+3$ & {$+4$ \\ $-4$} & {$+5$ \\ $-3$} & {$+6$ \\ $-2$} & {$+7$ \\ $-1$} & $0$ \\
  \end{tblr}
\end{sidewaystable}

\subsection{核外电子排布的周期性}
我们来看表 5-4 中原子序数 1~18 的元素原子电子层排布的情况。原子序数从 1~2 的元素,即从氢到氦,有一个电子层,电子层排布由 $1s^1$ 到 $1s^2$,电子由 1 个增到 2 个,达到稳定结构。
原子序数从 3~10 的元素,即从锂到氖,有两个电子层,最外电子层排布由 $2s^1$ 到 $2s^22p^6$,最外层电子从 1 个递增到 8 个,达到稳定结构。
原子序数从 11~18 的元素,即从钠到氩,有三个电子层,最外电子层排布从 $3s^1$ 到 $3s^23p^6$,最外层电子也从 1 个递增到 8 个,达到稳定结构。
如果我们对 18 号以后的元素继续研究下去,同样可以发现,每隔一定数目的元素,也会重复出现原子最外层电子数从 1 个递增到 8 个的情况。也就是说,随着原子序数的递增,元素原子的最外层电子排布呈周期性的变化。

\subsection{原子半径的周期性变化}
从\cref{tab:5-4} 可以看出,由碱金属元素锂到卤素氟,随着原子序数的递增,原子半径由 \qty{1.52e-10}{m} 递减到 \qty{0.71e-10}{m},即原子半径由大逐渐变小。
再由碱金属元素钠到卤素氯,随着原子序数的递增,原子半径由 \qty{1.86e-10}{m} 递减到 \qty{0.99e-10}{m},原子半径也是由大逐渐变小。
如果把所有的元素按原子序数递增的顺序排列起来,将会发现随着原子序数的递增,元素的原子半径发生周期性的变化\footnote{惰性气体元素原子半径跟邻近的非金属元素相比显得特别大,这是由于它们测定的根据跟其它元素不同。},\cref{fig:5-8} 表示碱金属等 7 个族和惰性气体元素的原子半径的周期性变化。
\begin{figure}
  \caption{元素原子半径的周期性变化}\label{fig:5-8}
\end{figure}


\subsection{第一电离能的周期性变化}
元素电离能的数值反映了元素原子失去电子的难易程度,元素的电离能越小,它的原子越容易失去电子。
因此,元素的第一电离能就是该元素的金属活动性的一种衡量尺度。

研究原子序数 1~18 的元素的第一电离能,我们将会发现,由氢到氦,由锂到氖,由钠到氩,第一电离能的变化趋势都是由小到大。
如果继续研究 18 号以后的元素,也会得出相同的结论。将 1~18 号元素的第一电离能数据绘成曲线图(\cref{fig:5-9}),从图上可以形象地看到,元素的第一电离能随着原子序数的递增,呈现周期性的变化。
\begin{figure}
  \caption{元素第一电离能的周期性变化}\label{fig:5-9}
\end{figure}
\begin{Theorem}{讨论}
  如何解释氨、铍、氖、镁几种元素的第一电离能比它们的相邻元素为高?
\end{Theorem}

\subsection{元素主要化合价的周期性变化}
从\cref{fig:5-4} 可以看到,第 11 号元素到第 18 号元素,在极大程度上重复着第 3 号元素到第 10 号元素所表现的化合价的变化——正价从 $+1$(\ce{Na})逐渐递变到 $+7$(\ce{Cl}),从中部的元素开始有负价,负价是从 $-4$(\ce{Si})递变到 $-1$(\ce{Cl})。
如果研究第 18 号元素以后的元素的化合价,同样可以看到与前面 18 种元素相似的变化。也就是说,元素的化合价随着原子序数的递增而起着周期性的变化。

原子半径、第一电离能和元素主要化合价,都是元素的重要性质。
通过上述的研究,我们可以引出这样一条规律,就是\emph{元素的性质随着元素原子序数的递增而呈周期性的变化}。 
这个规律叫做\Concept{元素周期律}。

元素性质的周期性变化是元素原子的核外电子排布的周期性变化的必然结果。

\begin{Practice}[习题]
  \begin{question}
    \item 随着原子序数的递增,原子半径有什么变化?
    \item 随着原子序数的递增,元素的第一 电离能和化合价各有什么变化?
    \item 用原子结构的观点说明为什么元素性质随原子序数的递增呈周期性的变化?
  \end{question}
\end{Practice}
\section{元素周期表}
根据元素周期律,把现在已知的 107 种元素\footnote{截至 2019 年,已知的元素种类已达 118 种。}中电子层数目相同的各种元素,按原子序数递增的顺序从左到右排成横行,再把不同横行中最外电子层的电子数相同的元素按电子层数递增的顺序由上而下排成纵行\footnote{严格说来,是把外围电子相似的元素按电子层数递增的顺序由上到下排成纵行。}。
这样得到一个表,叫做元素周期表(见附录: 元素周期表)。
元素周期表是元素周期律的具体表现形式,它反映了元素之间相互联系的规律。
下面我们就来学习元素周期表的有关知识。

\subsection{元素周期表的结构}
\subsubsection{周期}
元素周期表有 7 个横行,也就是 7 个周期。
具有相同的电子层数而又按照原子序数递增的顺序排列的一系列元素,称为一个周期。
周期的序数就是该周期元素原子具有的电子层数。

各周期里元素的数目不一定相同,第一周期只有 2 种元素; 第二、三周期各有 8 种元素; 第四、五周期各有 18 种元素; 第六周期有 32 种元素。
我们把含有元素较少的第一、二、 三周期叫短周期,把含有元素较多的四、五、六周期叫长周期。
第七周期到现在为止只发现了 21 种元素,还没有填满,叫不完全周期。

除第一周期外,同一周期中,从左到右,各元素原子最外电子层的电子数都是从 1 个逐步增加到 8 个。
除第一周期从气态元素氢开始,第七周期尚未填满外,每一周期的元素都是从活泼的金属元素——碱金属开始,逐渐过渡到活泼的非金属元素——卤素,最后以惰性气体结束。

第六周期中 57 号元素镧 \ce{La} 到 71 号元素镥 \ce{Lu},共 15 种元素,它们的电子层结构和性质非常相似,总称镧系元素。
为了使表的结构紧凑,将镧系元素放在周期表的同一格里,并按原子序数递增的顺序,把它们另列在表的下方,实际上还是各占一格。

第七周期 89 号元素锕 \ce{Ac} 至 103 号元素铹 \ce{Lr},共 15 种元素,它们彼此的电子层结构和性质也十分相似,总称锕系元素,同样把它们放在周期表的同一格里,并按原子序数递增的顺序另列在表下方镧系元素的下面。
锕系元素中铀后面的元素多数是人工进行核反应制得的元素,叫做超铀元素。

\subsubsection{族}
周期表有 18 个纵行。除第 8、9、10 三个纵行叫做第 \MyRoman{8} 族元素外,其余 15 个纵行,每个纵行标作一族。
族可分主族和副族。
由短周期元素和长周期元素共同构成的族,叫做主族;
完全由长周期元素构成的族,叫做副族。
主族元素在族的序数(习惯用罗马数字表示)后面标一 A 字,如 \MyRoman{1}A、 \MyRoman{2}A……,副族元素标一 B 字,如 \MyRoman{1}B、\MyRoman{2}B……。
惰性气体元素化学性质非常不活泼,在通常状况下难以发生化学反应,把它们的化合价看作为 0 ,因而叫做 0 族。
因此. 在整个周期表里,有 7 个主族,7 个副族,1 个第 \MyRoman{8} 族,1 个 0 族,共 16 个族。

\begin{Reading}{元素周期表根据原子的电子层结构分区}
根据原子的电子层结构的特征,元素周期表可划分为四个区(\cref{fig:5-10})。
\begin{figurehere}
  \begin{minipage}{\linewidth}\centering
    \caption{元素周期表根据原子的电子层结构分区}\label{fig:5-10}
  \end{minipage}
\end{figurehere}
\begin{enumerate}
  \item $s$ 区包括 \MyRoman{1}A 和 \MyRoman{2}A 两个主族,最外层只有 1~2 个 $s$ 电子。
  \item $p$ 区包括 \MyRoman{3}A~\MyRoman{7}A 五个主族和 0 族,最外层除了 2 个 $s$ 电子之外,有 1~6 个 $p$ 电子(\ce{He} 例外,无 $p$ 电子)。
  \item $d$ 区包括 \MyRoman{1}B~\MyRoman{7}B 七个副族和第 \MyRoman{8} 族,均属过渡元素。
  最外层有 2 个 $s$ 电子(个别为 1 个,\ce{Pd} 例外,无 $5s$ 电子),次外层有 1~10 个 $d$ 电子。 $d$ 区元素原子电子层的这种结构特征可用通式 $(n-1)d^xns^2$ 表示,$n$ 是周期数,$x$ 是 1~10 的正整数。
  这种表示原子电子层结构特征的式子,又叫原子的特征电子构型(也称外围电子)。
  $s$ 区的特征电子构型是 $ns^x$,$x$ 是 1~2 的正整数;$p$ 区的特征电子构型是 $ns^2np^x$,$x$ 是 1~6 的正整数。
  \item $f$ 区包括镧系和钠系,最外层有 2 个 8 电子,次外层有 2 个 8 电子和 6 个 $p$ 电子(个别有 $d$ 电子),例数第三层有 1~14 个 $f$ 电子。特征电子构型一般是 $(n-2)f^xns^2$,$x$ 是 1~14 的正整数。$f$ 区元素也是过渡元素。
\end{enumerate}

$s$ 区元素都是活泼的金属(氢除外),它们起化学反应时总是失去最外层的 $s$ 电子而成为 $+1$ 价或 $+2$ 价的阳离子。

$p$ 区元素除惰性气体外,有金属,也有非金属。
它们在起化学反应时只有最外层的 $s$ 亚层或 $p$ 亚层的电子发生得失或偏移,不牵涉内层电子。
$ns^2np^8$ 是惰性气体的特征电子构型(\ce{He} 的特征电子构型为 $1s^2$ ),是一种稳定结构,在通常状况下难以发生电子得失或偏移。

$d$ 区元素都是金属,它们在发生化学反应时,不仅有最外层的 $s$ 电子,而且可以有部分或全部次外层的 $d$ 电子失去或偏移。

$f$ 区元素也都是金属。
它们在发生化学反应时,不仅有最外层的 $s$ 电子,次外层的 $d$ 电子,而且可以有倒数第三层的部分或全部 $f$ 电子失去或偏移。
\end{Reading}

\subsection{元素的性质跟原子结构的关系}
\subsubsection{原子结构跟元素的金属性和非金属性的关系}
在同一周期中,各元素的原子核外电子层数虽然相同,但从左到右,核电荷依次增多,原子半径逐渐减小,电离能趋于增大,失电子能力逐渐减弱,得电子能力逐渐增强,因此,金属性逐渐减弱,非金属性逐渐增强。
从同周期元素化学性质变化情况的研究可以证实这个结论是正确的。

一般说来,我们可以从元素的单质跟水或酸反应置换出氢的难易,元素氧化物的水化物(氧化物间接或直接跟水生成的化合物)——氢氧化物的碱性强弱,来判断元素金属性的强弱; 可以从元素氧化物的水化物的酸性强弱,或从跟氢气生成气态氢化物的难易,来判断元素非金属性的强弱。
下面以第三周期元素为例,来研究同周期元素金属性和非金属性的递变。

我们知道,第 11 号元素钠的单质能跟冷水剧烈反应,放出氢气,生成的氢氧化钠是一种强碱。

第 12 号元素镁,它的单质跟水起反应的情况怎样呢?
\begin{Experiment}
  取两段镁带,用砂纸擦去氧化膜,放于试管中,加 \qty{3}{mL} 水,往水中滴 2 滴无色酚酞试液,观察现象。然后加热试管至水沸腾,观察现象。
\end{Experiment}

实验表明,镁不易跟冷水作用,但加热时能跟沸水起反应,产生大量气泡,反应后的溶液使无色酚酞试液变红。这个反应的化学方程式如下:
\[ \ce{ Mg + 2H2O \xlongequal{\quad} Mg(OH)2 +H2 ^} \]

镁能从水中置换出氢,说明它是一种活泼金属。
但它只容易跟沸水起反应,所生成的氢氧化镁的碱性也比氢氧化钠弱,说明它的金属活动性不如钠强。

现在我们来研究第 13 号元素铝的一些性质。
\begin{Experiment}
取一小片铝和一小段镁带,用砂纸擦去氧化膜,分别放入两个试管中,再各加入 \qty{2}{mL} $1M$ 盐酸,观察现象。
\end{Experiment}

实验表明,镁、铝都能跟盐酸起反应,置换出氢气,反应的化学方程式如下:
\begin{gather*}
  \ce{ Mg + 2HCl \xlongequal{\quad} MgCl2 + H2 ^}\\
  \ce{ 2Al + 6HCl \xlongequal{\quad} 2AlCl3 + 3H2 ^}
\end{gather*}
但铝跟酸的反应不如镁跟酸的反应剧烈。也就是说,铝的金属活动性不如镁强。

我们在初中已经知道,铝的氧化物 \ce{Al2O3} 既能跟酸反应,又能跟碱反应,是一种两性氧化物。那么,它的对应水化物氢氧化铝的酸碱性又怎样呢?
\begin{Experiment}
取少量 $1M$ 的三氯化铝溶液注入试管中,加入 $3M$ 的氢氧化钠溶液到产生大量的氢氧化铝白色絮状沉淀为止。
将氢氧化铝沉淀分盛在两个试管中,然后在两个试管中分别加入 $3M$ 的硫酸和 $6M$ 的氢氧化钠溶液,观察现象。
\end{Experiment}
我们看到,两个试管中的白色沉淀都消失了。这说明,氢氧化铝既能跟酸反应,又能跟碱反应。

上述反应的化学方程式如下:
\begin{gather*}
  \ce{ AlCl3 + 3NaOH \xlongequal{\quad} 3NaCl + Al(OH)3 v  } \\ 
  \ce{ 2Al(OH)3 + 3H2SO4 \xlongequal{\quad} Al2(SO4)3 + 6H2O}\\ 
  \ce{ H3AlO3 + NaOH \xlongequal{\quad} NaAlO2 + 2H2O}
\end{gather*}
当 \ce{Al(OH)3} 跟碱起反应时,它的分子式还可以写成 \ce{H3AlO3} 的形式。

象氢氧化铝这样既能跟酸起反应,又能跟碱起反应的氢氧化物,叫做两性氢氧化物。氢氧化铝既然呈两性,就说明铝已表现出一定的非金属性。

第 14 号元素硅是非金属。
硅的氧化物 \ce{SiO2} 是酸性氧化物,它的对应水化物是硅酸(\ce{H4SiO4})。
硅酸是一种很弱的酸。 
硅只有在高温下才能跟氢气起反应生成气态氢化物 \ce{SiH4}。

第 15 号元素磷是非金属,它的最高价氧化物是 \ce{P2O5},\ce{P2O5} 的对应水化物是磷酸(\ce{H3PO4}),属于中强酸。
磷的蒸气和氢气能起反应生成气态氢化物 \ce{PH3},但相当困难。

第 16 号元素硫是比较活泼的非金属,它的最高价氧化物是 \ce{SO3},\ce{SO3} 的对应水化物是硫酸。
硫酸是一种强酸。
在加热时硫能跟氢气化合生成气态氢化物硫化氢。

第 17 号元素氯是很活泼的非金属,它的最高价氧化物是 \ce{Cl2O7},\ce{Cl2O7} 的对应水化物是高氯酸(\ce{HClO4}),它是已知酸中最强的一种酸。
氯气跟氢气在光照或点燃时就能发生爆炸而化合,生成气态氢化物氯化氢。

第 18 号元素氩是一种惰性气体。

综上所述,可以得出如下结论:
\begin{center}
  \begin{tikzpicture}
    \foreach \x[count=\i] in {Na,Mg,Al,Si,P,S,Cl}
      { \node at (\i,0.6) [below] {\ce{\x}}; }
    \draw[->](0,0)--(8,0)node[midway,below]{金属性逐渐减弱,非金属性逐渐增强};
  \end{tikzpicture}
\end{center}

对其它周期元素的化学性质进行逐一的探讨,也会得到类似的结论。

在同一主族的元素中,由于从上到下电子层数增多,原子半径增大,电离能一般趋于减小,失电子能力逐渐增强,得电子能力逐渐减弱,所以元素的金属性逐渐增强,非金属性逐渐减弱。
这可以从碱金属元素和卤素的化学性质的递变中得到证明。
我们知道,碱金属元素的金属性是从上到下逐渐增强,卤素的非金属性是从上到下逐渐减弱的。

副族元素化学性质的变化规律比较复杂,这里就不讨论了。

我们还可以在周期表上对金属元素和非金属元素进行分区(\cref{tab:5-5})。
如果沿着周期表中硼、硅、砷、碲、破跟铝、锗、锑、 钋之间划一条虚线,虚线的左面是金属元素,右面是非金属元素。
左下方是金属性最强的元素,右上方是非金属性最强的元素。
由于金属性、非金属性没有严格的界线,位于分界线附近的元素,既表现某些金属性质,又表现某些非金属性质。

\begin{table}
  \caption{主族元素金属性和非金属性的递变}\label{tab:5-5}
  \begin{tblr}{
      colspec={c*{8}{X[c]}},
      hlines={0pt},vlines={0pt},rowsep=2pt,
      hline{1,Z}={1.5pt},vline{1,Z}={1.5pt},
      hline{2}={0.8pt},vline{2}={0.8pt},
      vline{4}={3-3}{dashed,1.2pt},hline{4}={4-4}{dashed,1.2pt},
      vline{5}={4-4}{dashed,1.2pt},hline{5}={5-5}{dashed,1.2pt},
      vline{6}={5-5}{dashed,1.2pt},hline{6}={6-6}{dashed,1.2pt},
      vline{7}={6-6}{dashed,1.2pt},hline{7}={7-7}{dashed,1.2pt},
      vline{8}={7-7}{dashed,1.2pt},hline{8}={8-8}{dashed,1.2pt}
    }
    \diagbox{周期}{族}& \MyRoman{1}A & \MyRoman{2}A & \MyRoman{3}A & \MyRoman{4}A & \MyRoman{5}A & \MyRoman{6}A & \MyRoman{7}A & \\
    1 &\SetCell[r=7]{m,r}{\tikz \draw[->](0,4)--(0,0)node[midway,left]{\parbox{1em}{金\\属\\性\\逐\\渐\\增\\强}};} 
    & \SetCell[c=6]{h,c}{\tikz \draw[-stealth](0,0)--(9,0) node[midway,above]{非金属性逐渐增强};} &&&&&
    &\SetCell[r=7]{m,l}{\tikz \draw[<-](0,4)--(0,0)node[midway,right]{\parbox{1em}{非\\金\\属\\性\\逐\\渐\\增\\强}};}\\
    2 & & &\ce{B}  & & & & &  \\
    3 & & &\ce{Al} & \ce{Si} &  & & &  \\
    4 & & &        & \ce{Ge} & \ce{As} & & & \\
    5 & & &        &         & \ce{Sb} & \ce{Te} & & \\
    6 & & &        &         &         & \ce{Po} & \ce{At} & \\
    7 & & \SetCell[c=6]{f,c}{\tikz \draw[stealth-](0,0)--(9,0) node[midway,below]{金属性逐渐增强};}&&&&&&\\
  \end{tblr}
\end{table}

\subsubsection{原子结构跟化合价的关系}
元素的化合价跟原子的电子层结构有密切关系,特别是跟最外层电子的数目有关,因此,元素原子的最外层电子,称为价电子。
有些元素的化合价跟它们原子的次外层或倒数第三层的部分电子有关,这部分电子也叫价电子。

在周期表中,主族元素的最高正化合价等于它所在族的序数,因为它们的最外层电子数,即价电子数,跟族的序数相当。
非金属元素的最高正化合价和它的负化合价绝对值的和等于 8。
因为非金属元素的最高正化合价,等于原子所失去或偏移的最外层上的电子数; 而它的负化合价,则等于原子最外层达到 8 个电子稳定结构所需得到的电子数。

副族和第 \MyRoman{8} 族元素的化合价比较复杂,它们原子次外层 $d$ 亚层或倒数第三层 $f$ 亚层上的电子不很稳定,在适当的条件下,和最外层电子一样,也可失去。
它们失去电子的最大数目一般说来跟它们的族的序数相当。

从以上的学习中我们可以知道,元素的性质是由原子结构决定的; 元素在周期表中的位置反映了那个元素的原子结构和一定的性质。
所以,元素性质、原子结构和该元素在周期表中的位置三者有着密切的关系。
我们可以根据元素在周期表中的位置,推论它的原子结构和一定的性质; 反过来,根据元素的原子结构,也可以推论它在周期表中的位置。

\begin{example}
  已知某元素在第四周期 \MyRoman{6}A 族,试写出它的电子排布式,指出它是金属元素还是非金属元素,最高正化合价是多少,最高价氧化物的水化物是酸还是碱。
\end{example}
\begin{solution}
设该元素为 $X$。已知 $X$ 在第四周期,因此,它的原子核外有 4 个电子层。

又知 $X$ 属 \MyRoman{6}A 族,即它的原子的最外层电子数是 6 个,因此它的电子排布式是 $1s^22s^22p^63s^23p^63d^{10}4s^24p^4$。

根据电子排布式判断,它在化学反应中易得到 2 个电子,形成 8 电子稳定结构,这时表现为 $-2$ 价,因此它是非金属元素。它的最高正化合价为 $+6$,最高价氧化物 \ce{$X$O_{$i$}} 的水化物为 \ce{H2$X$O4},是一种酸。
\end{solution}

\begin{example}
已知某元素原子序数为 32,试指出它属于哪一周期,哪一族,是什么元素。
\end{example}
\begin{solution}
  该元素原子序数为 32,即核外有 32 个电子。已知前 4 个周期共有 $2+8+8+18=36$ 个元素,第 36 号元素是惰性气体 \ce{Kr},它的最外层电子是 $4s^24p^6$,而该元素比  \ce{Kr} 少 $36-32=4$ 个电子,它的电子排布式为:$1s^22s^22p^63s^23p^63d^{10}4s^24p^2$。
  
根据电子排布式判断,它属于第四周期,\MyRoman{4}A 族。查周期表,知道该元素是锗 \ce{Ge}。
\end{solution}

注意,解答这类问题时,必须先经过分析推理,找出所求元素在周期表中的位置,然后查周期表得出它的名称。
决不能根据题设的原子序数等数据,直接查周期表得出答案。

\begin{Practice}[习题]
  \begin{question}
    \item 用原子结构的知识,说明元素周期表里的周期和族是按什么划分的?什么叫主族?什么叫副族?
    \item 对于同周期和同主族元素来说,元素的金属性和非金属性是怎样逆变的?在元素周期表上金属元素和非金属元素是怎样分区的?
    \item 有某元素 $A$,它的最高氧化物的分子式是 \ce{$A$O3},气态氢化物里含氢 2.489\%,这是什么元素?
    \item 某元素 $B$ 的最高正化合价和负化合价的绝对值相等,该元素在气态氢化物中占 87.5\%,问该元素的原子量是多少,它是什么元素。
    \item 根据元素在周期表中的位置,判断下列各组化合物的水溶液,哪个酸性较强? 哪个碱性较强?
    \begin{tasks}(2)
      \task \ce{H2CO3} 和 \ce{H3BO3}(硼酸)
      \task \ce{H3PO4} 和 \ce{HNO3}
      \task \ce{Ca(OH)2} 和 \ce{Mg(OH)2}
      \task \ce{Al(OH)3} 和 \ce{Mg(OH)2}
    \end{tasks}
    \item 已知三种元素的原子序数是 11、33 和 35,不看元素周期表,确定它们各处在哪一周期,哪一族,并说明你是如何推断的?
    \item 填空:
    
    % \begin{tblr}{}
    % \end{tblr}
    \item 甲元素原子的核电荷数为 17,乙元素的正二价离子跟氢原子(原子序数为 18)的电子层结构相同。试回答下列问题:
    \begin{tasks}
      \task 甲元素在周期表里位于第\_\_\_周期,第\_\_\_主族,电子排布式是\_\_\_,元素符号是\_\_\_,它的最高价氧化物对应的水化物分子式是\_\_\_,属于无机物的\_\_\_类。
      \task 乙元素在周期表里位于第\_\_\_周期,第\_\_\_主族,电子排布式是\_\_\_,元素符号是\_\_\_,它的最高价氧化物对应的水化物分子式是\_\_\_,属于无机物的\_\_\_ 类。
      \task 比较碘跟甲元素的非金属性哪个强,乙元素跟钾的金属性哪个强。
    \end{tasks}
  \end{question}
\end{Practice}

\section{元素周期律的发现和意义}
从十八世纪中叶到十九世纪中叶这一百年间,随着生产和科学技术的发展,新的元素不断地被发现。
到 1869 年,人们已经知道了 63 种元素。
对于这些元素的物理、化学性质的研究,也已积累了不少的资料。
但是它们还是杂乱无章、无甚头绪的材料,不便于进一步研究和使用。
因此,人们产生了整理和概括这些感性材料,将元素进行分类,寻找它们内在联系的迫切要求,元素周期律就是在这个时代背景下,经过许多人的努力,最后由俄国化学家门捷列夫发现的。
\begin{Reading}[]{补充阅读}
1829 年,德国人德贝莱纳(D\"obereiner,1780--1849)根据元素性质的相似性提出了“三素组”学说。当时它归纳出五个“三素组”,即

\[ \ce{Li},\ce{Na},\ce{K}\quad\ce{Ca},\ce{Sr},\ce{Ba}\quad\ce{P},\ce{As}, \ce{Sb}\quad\ce{S},\ce{Se},\ce{Te}\quad \ce{Cl},\ce{Br},\ce{I}\]

当时已经知道了 54 种元素,而他却只能将 15 种元素归纳入三素组,不能揭示其它大部分元素间的关系,因此,三素组学说没能引起人们的重视。

此后,又有许多人对元素分类作过研究,比较突出的有迈尔的《六元素表》和纽兰兹的《八音律表》。

1864 年,德国人迈尔(Meyer,1830--1895)发表了《六元素表》,在表中对于性质相似的元素六个、六个地进行了分族,但他已归纳成族的元素尚不及当时已知元素的一半。

1865 年,英国人纽兰兹(Newlands,1837--1898)把当时已知的元素按原子量的大小顺序排列,发现从任意一个元素算起,每到第八个元素就和第一个元素的性质相近,犹如八度音阶一样。
他把这个规律叫做“八音律”。
可是他按八音律排的元素表很多地方却是混乱的。
原因是他没充分估计到当时的原子量测定值可能有错误,而是机械地按原子量大小往下排;同时他也没考虑到还有未被发现的元素,没有留下空位。
显然他这样作不能把元素内在联系的规律揭示出来。

对于这项将元素进行科学分类,寻找它们内在联系规律的重要工作,1869 年俄国化学家门捷列夫({\rcmu Менделеев},1834--1907)获得了成功。
他在批判继承前人工作的基础上,对大量实验事实进行了订正、分析和概括,总结出一条规律:元素(以及由它所形成的单质和化合物)的性质随着原子量的递增而呈周期性的变化。
这就是元素周期律。
他还根据元素周期律编制了第一个元素周期表,把已经发现的 63 种元素全部列入表里,从而初步完成了使元素系统化的任务。
他还在表中留下空位,预言了类似硼、铝、硅的未知元素(门捷列夫叫它类硼、类铝和类硅,即以后发现的钪、镓、锗)的性质,并指出当时测定的某些元素原子量的数值有错误,而他在周期表中也没有机械地完全按照原子量数值的顺序排列。
若干年后,他的预言都得到了证实。
门捷列夫工作的成功,引起了科学界的震动。
人们为了纪念他的功绩,就把元素周期律和周期表称为门捷列夫元素周期律和门捷列夫元素周期表。
但是由于时代的局限,门捷列夫仍然未能认识到造成元素性质周期性变化的根本原因。

二十世纪以来,随着科学技术的发展,人们对于原子的结构有了更深刻的认识。
人们发现,引起元素性质周期性变化的本质原因不是原子量的递增,而是核电荷数(原子序数)的递增,也就是核外电子排布的周期性变化。
这样才把元素周期律修正为现在的形式,同时对于元素周期表也作了许多改进。
\end{Reading}

元素周期律的发现,对于化学科学的发展,有很大的影响。

元素周期表是学习和研究化学的一种重要工具。
元素周期表是元素周期律的具体表现,它反映了元素之间的内在联系,是对元素的一种很好的自然分类。
我们可以利用元素的性质、它在周期表中的位置和它的原子结构三者之间的密切关系,来指导我们对化学的学习和研究。

过去,门捷列夫曾用它预言未知元素并得到了证实;此后,人们在周期律、周期表的指导下,对元素的性质进行系统地研究,对物质结构理论的发展起了一定的推动作用。
不仅如此,元素周期律和周期表对于新元素的合成、预测它的原子结构和性质提供了线索。

元素周期律对于工农业生产也有一定的指导作用。
由于在周期表中位置靠近的元素性质相近,这就启发了人们在周期表中一定的区域内寻找新的物质。
例如通常用来制造农药的元素,如氟、氯、硫、磷、砷等在周期表里占有一定区域。
对这个区域里的元素进行充分的研究,有助于制造出新品种的农药。
又例如要找半导体材料,可以在周期表里金属和非金属的接界处去找,如硅、锗、硒等就是。
我们还可以在过渡元素中去寻找催化剂和耐高温、耐腐蚀的合金材料等。

元素周期律的重要意义还在于它从自然科学上有力地论证了事物变化的量变引起质变的规律性。
\section*{内容提要}
\setcounter{subsection}{0}
\subsection{原子结构}
\begin{enumerate}
  \item 组成原子的粒子间的关系如下:
  \[ \text{原子}\; \ce{^$A$_$Z$ $X$} 
     \left\{ \begin{array}{l}
       \text{原子核} \left\{
        \begin{array}{ll}
          \text{质子} & Z \text{个}\\
          \text{中子} & (A-Z) \text{个}\\
        \end{array}
       \right.\\
       \text{核外电子}\quad Z \text{个}    
    \end{array}
     \right.
  \]
  \item 具有相同质子数和不同中子数的同一元素的原子互称同位素。
  \item 电子在核外空间作高速的运动,好象带负电荷的云雾笼罩在原子核的周围,我们形象地称它为“电子云”。
  \item 从气态原子(或气态阳离子)中去掉电子,把它变成气. 态阳离子(或更高价的气态阳离子),所需消耗的能量叫做电离能。根据元素电离能的变化,可以判断电子是分层排布的。
  \item 一个电子的运动状态由它所处的电子层、电子亚层、 电子云的空间伸展方向和自旋状态四个方面来决定。
  \begin{enumerate}
    \item 电子层\quad 根据电子的能量差别和通常运动的区域离核的远近不同,可以将核外电子分成不同的电子层。
    \item 电子亚层\quad 在同一电子层中,根据电子能量的差别和电子云形状的不同,可以分为 $s$、$p$、$d$、$f$ 等几个亚层。
    \item 电子云的伸展方向\quad $s$ 电子云是球形对称的,$p$ 电子云有 3 种伸展方向,$d$ 电子云有 5 种伸展方向,$f$ 电子云有 7 种伸展方向。
    \item 电子的自旋\quad 电子的自旋有两种状态,相当于顺时针和逆时针两种方向。
  \end{enumerate}
  \item 在一定电子层上、具有一定形状和伸展方向的电子云所占据的空间称为一个轨道。
  \item 核外电子排布遵循以下规律:
  \begin{description}[style=nextline,leftmargin=10em]
    \item[泡利不相容原理] 在同一个原子中,不可能有运动状态完全相同的两个电子存在。
    \item[能量最低原理] 核外电子总是尽先占有能量最低的轨道。
    \item[洪特规则] 在同一亚层的各个轨道上,电子的排布将尽可能分占不同的轨道,而且自旋方向相同。
  \end{description}
\end{enumerate}
\subsection{元素周期律和周期表}
\begin{enumerate}
  \item 元素的性质随着元素原子序数的递增而呈周期性的变化。这就是元素周期律。
  \item 周期表中具有相同电子层数而又按照原子序数递增的顺序排列的一系列元素,叫做一个周期。周期表中每个纵行叫做一个族(第 \MyRoman{8} 族包括三个纵行)。
  \item 在同一周期中,从左到右(惰性气体除外),元素的金属性减弱,非金属性增强。在同一主族中,从上到下,元素的金属性增强,非金属性减弱。
  \item 主族元素的最高正化合价等于它所在的族的序数;非金属元素的最高正化合价和它的负化合价绝对值的和等于 8。
  \item 元素周期律的发现,对于化学科学的发展有很大的影响。元素周期表是学习和研究化学的一种重要工具,对于工农业生产也有一定的指导作用。
\end{enumerate}
\begin{Review}
  \begin{question}
    \item 下列各种事实跟原子结构的哪一部分有关?
    \begin{tasks}
      \task 元素在周期表中的排列顺序;
      \task 原子量的大小;
      \task 元素具有同位素;
      \task 元素的化学性质;
      \task 元素的化合价;
      \task 元素在周期表里处于哪个周期;
      \task 主族元素在周期表里处于哪个族。
    \end{tasks}
    \item 自然界里 \ce{^14_7N} 占 99.635\% ,\ce{^16_7N} 占 0.365\%,求氮元素的近似原子量。
    \item 为什么各个电子层所能容纳的最多电子数为 $2n^2$?
    \item 某元素原子的电子排布式是 $1s^22s^22p^63s^23p^63d^{10}4s^2$,说明这个元素的原子核外有多少个电子层?每个电子层有多少个轨道,有多少个电子?
    \item 已知下列元素原子的最外电子层结构(内层已填满)为:
    \[ 3s^1, \quad 4s^24p^1, \quad 3s^23p^3\]
    它们各属于第几周期?第几族?最高正化合价是多少?
    \item 已知 \ce{K}、\ce{Ca} 分别属于第 4 周期 \MyRoman{1}A 族和 \MyRoman{2}A 族,\ce{Ar} 是第 3 周期 0 族。
    \begin{tasks}
      \task 写出 \ce{K}、\ce{Ca}、\ce{Ar} 的电子排布式;
      \task 比较三个元素的电子层结构特征;
      \task 已知 \ce{K}、\ce{Ca} 的电离能(单位:\unit{eV})为:

      \begin{tabular}{cccc}
                 & $I_1$ & $I_1$  & $I_1$  \\
         \ce{K}  & 4.341 & 31.63  & 45.72  \\
         \ce{Ca} & 6.113 & 11.87  & 50.91  \\
      \end{tabular}
      
      试从电子层结构的观点及电离能的数据,说明在化学反应中 \ce{K} 表现为 $+1$ 价、\ce{Ca} 表现为 $+2$ 价的原因。
    \end{tasks}
    \item 设计实验证明钠、镁、铝的金属性依次减弱,并写出实验步骤、现象和有关的化学方程式。
    \item 某元素 $A$ 的气态氢化物 \ce{$A$H3} 中含 \ce{H} 为 17.65\%,又知该元素的原子核中有 7 个中子,试求:
    \begin{tasks}
      \task 该元素的原子量;
      \task 它在周期表的位置及元素名称。
    \end{tasks}
    (提示:可以粗略地从原子量推知质量数,下同。)
    \item 某元素 $B$ \qty{0.9}{g} 和稀盐酸反应生成 \ce{$B$Cl3},置换出 \qty{1.12}{L} 氢气(标准状况),$B$ 的原子核里有 14 个中子,根据计算结果,写出 $B$ 的电子排布式,说明它是什么元素。
  \end{question}
\end{Review}

\begin{Exercise}*[总复习题]
  \begin{question}
    \item 填空
    \begin{tasks}
      \task \qty{0.5}{mol} 水含\CJKunderline[hidden]{\ \num{3.01e23}\ }个水分子,共含有\CJKunderline[hidden]{\ \num{9.03e23}\ }个原子,它的质量是\CJKunderline[hidden]{\ \num{9}\ }\unit{g}。
      \task 燃烧 \qty{8}{g} 硫粉可以放出 \qty{17.7}{kCal} 的热量,该反应的热化学方程式是\CJKunderline[hidden]{该反应的热化学方程式该反应的热化学方程式}。
      \task 把 4 体积二氧化硫跟 3 体积氧气在一定条件下起反应,反应后的混和气体是\CJKunderline[hidden]{\ \num{3.01e23}\ }体积,其中,三氧化硫与氧气的摩尔数之比是\CJKunderline[hidden]{\ \num{3.01e23}\ }。
      \task \qty{0.38}{g} 某卤素单质在标准状况下的体积是 \qty{120}{mL},这种卤素单质的分子量是\CJKunderline[hidden]{\ \num{3.01e23}\ },这是\CJKunderline[hidden]{\ \num{3.01e23}\ }。
      \task
    \end{tasks}
    \item 选择正确的答案填写在括号里。
    \begin{enumerate}[label=(\arabic*),leftmargin=1.7em]
      \item 下列物质(或指溶液中的溶质)含分子数最多的是\hfill(\qquad )
      \begin{tasks}(2)
        \task \qty{22.4}{L} 氢气(标准状况)
        \task \num{3.0e23} 个氧分子,
        \task \qty{9e-3}{kg} 水,
        \task $2M$ 盐酸 \qty{600}{mL},
        \task 98\% 浓硫酸(密度为 \qty{1.84}{g/cm^3})\qty{100}{mL}。
      \end{tasks}
      \item 下列物质(或指溶液中的溶质)含分子数最多的是\hfill(\qquad )
    \end{enumerate}
    \item 
    \item 
    \item 
    \item 
    \item 
    \item 
    \item 
    \item 
    \item 
    \item 
    \item 
    \item 
    \item 
    \item 
    \item 
    \item 
    \item 
    \item 
  \end{question}
\end{Exercise}
% \chapter*{学生实验}\markboth{学生实验}{学生实验}
\stepcounter{chapter}
\ctexset{section/name={实验,}}
\addcontentsline{toc}{chapter}{学生实验}

\section{化学实验基本操作}
\section{配置一定摩尔浓度的溶液}
\section{重结晶法提纯硫酸铜\texorpdfstring{\quad}{ }测定硫酸铜晶体里结晶水的含量}
\section{氯、溴、碘的性质}
\section{硫酸的性质\texorpdfstring{\quad}{ }硫酸根离子的检验}
\section{实验习题}
\section{碱金属及其化合物的性质}
\section{同周期、同主族元素性质的递变}
\section{实验习题}
\section*{选作实验\texorpdfstring{\quad}{ }阿伏伽德罗常数的测定}
\appendix
% \chapter{酸、碱和盐的溶解性表(\qty{20}{\celsius})}
{\noindent\small
\begin{tblr}{colspec={c*{9}{X[c]}},colsep=3pt,rowsep=0pt}
  \diagbox{阳离子}{阴离子} & \ce{OH-} & \ce{NO_3^-}& \ce{Cl-}& \ce{SO_4^2-}& \ce{S^2-}& \ce{SO_3^2-}& \ce{CO_3^2-}& \ce{SiO_3^2-}& \ce{PO_4^2-}\\
  \ce{H+}     &        & 溶、挥 & 溶、挥 & 溶 & 溶、挥 & 溶、挥 & 溶、挥 & 微 & 溶 \\
  \ce{NH_4^+} & 溶、挥 & 溶     & 溶     & 溶 & 溶     & 溶     & 溶     & 溶 & 溶 \\
  \ce{K+}     & 溶     & 溶     & 溶     & 溶 & 溶     & 溶     & 溶     & 溶 & 溶 \\
  \ce{Na+}    & 溶     & 溶     & 溶     & 溶 & 溶     & 溶     & 溶     & 溶 & 溶 \\
  \ce{Ba^2+}  & 溶     & 溶     & 溶     & 不 & —      & 不     & 不     & 不 & 不 \\
  \ce{Ca^2+}  & 微     & 溶     & 溶     & 微 & —      & 不     & 不     & 不 & 不 \\
  \ce{Mg^2+}  & 不     & 溶     & 溶     & 溶 & —      & 微     & 微     & 不 & 不 \\
  \ce{AL^3+}  & 不     & 溶     & 溶     & 溶 & —      & —      & —      & 不 & 不 \\
  \ce{Mn^2+}  & 不     & 溶     & 溶     & 溶 & 不     & 不     & 不     & 不 & 不 \\
  \ce{Zn^2+}  & 不     & 溶     & 溶     & 溶 & 不     & 不     & 不     & 不 & 不 \\
  \ce{Cr^3+}  & 不     & 溶     & 溶     & 溶 & —      & —      & —      & 不 & 不 \\
  \ce{Fe^2+}  & 不     & 溶     & 溶     & 溶 & 不     & 不     & 不     & 不 & 不 \\
  \ce{Fe^3+}  & 不     & 溶     & 溶     & 溶 & —      & —      & 不     & 不 & 不 \\
  \ce{Sn^2+}  & 不     & 溶     & 溶     & 溶 & 不     & —      & —      & —  & 不 \\
  \ce{Pb^2+}  & 不     & 溶     & 微     & 不 & 不     & 不     & 不     & 不 & 不 \\
  \ce{Bi^3+}  & 不     & 溶     & —      & 溶 & 不     & 不     & 不     & —  & 不 \\
  \ce{Cu^2+}  & 不     & 溶     & 溶     & 溶 & 不     & 不     & 不     & 不 & 不 \\
  \ce{Hg+}    & —      & 溶     & 不     & 溶 & 不     & 不     & 不     & —  & 不 \\
  \ce{Hgi^2+} & —      & 溶     & 溶     & 溶 & 不     & 不     & 不     & —  & 不 \\
  \ce{Ag+}    & —      & 溶     & 不     & 溶 & 不     & 不     & 不     & 不 & 不 \\
\end{tblr}}

{\smallskip\noindent\footnotesize
说明: “溶”表示那种物质可溶于水,“不”表示不溶于水,“微”表示微溶于水,“挥”表示挥发性,“—”表示那种物质不存在或遇到水就分解了。
} % ok
\end{document}